\chapter{Standard library}\label{chap:standardlibrary}
The standard library of MC consists of three seperate parts:

\begin{description}[align=right,labelwidth=2cm]
   \item [StandardLibrary] contains all the basic Library items
   \item [BasicMonads] contains all the basic monads
\end{description}

We will now look at the standard library and explain the evolution it went through during the assignment.
For readability only the neccesary parts of the code is shown.
The complete can be found on~\cite{}.

\section{Prelude}
\emph{Prelude} contains some basic definitions making basic programming possible.
We will go through \emph{prelude} in parts, the complete file can be found here~\ref{}.

\begin{lstlisting}
import System

Data "unit" -> Unit
\end{lstlisting}

This first part shows how MC imports the .NET \emph{System} library and how \verb|Unit| is defined within MC.
The \emph{System} library will be used later on in the code.
\verb|Unit|, or \verb|unit| on term level, defines the empty or null value in MC.

\begin{lstlisting}
TypeAlias #a => "*" => #b => #c
'a * 'b => tuple<'a 'b>
Data 'a -> "," -> 'b -> 'a * 'b    #> 5
\end{lstlisting}
Next we have the declaration and definition of a tuple on both term and type level.

\begin{lstlisting}
TypeAlias #a => "|" => #b => pipe<#a #b>
Data "Left" -> 'a -> 'a | 'b       #> 5
Data "Right" -> 'b -> 'a | 'b      #> 5

Data "then" -> Then
Data "else" -> Else

Func "if" -> Boolean^System -> Then -> 'a -> Else -> 'a -> 'a
if True^builtin then f else g -> f
if False^builtin then f else g -> g
\end{lstlisting}

\section{Match}
\emph{Match} works with the pipe operator.
It takes a variable and matches it on either \verb|Left| or \verb|Right| and executes the function specified.

\begin{lstlisting}
import prelude

Data "with" -> With

TypeFunc "match" => #a => Module
match ('a | 'b) => Module ('a | 'b) {
  TypeFunc "Head" => #a
  Head => 'a

  TypeFunc "Tail" => #a
  Tail => 'b

  Func "do" -> 'a -> With -> (Head -> 'b) -> (Tail -> 'b) -> 'b
  do (Left x) with f g -> f x

  do y with g h -> res
  --------------------
  do (Right y) with f (g h) -> res

  do (Right y) with f g -> g y
}
\end{lstlisting}
\section{Number}
\emph{Number} was created to give the user a generic interface to create numbers.
It can be used to create integers, floats or a newly created number type.
However because of the import system of MC it can directly import the integer and float types with all their functions from DotNet.
It can still be used for self defined number types.




During development \emph{number} was used to create integers and floats.
This was before the .NET import system was in place and it was thus needed.
Currently it is only needed for user defined number types.

\section{Record}
\section{Monad}
\section{TryableMonad}
