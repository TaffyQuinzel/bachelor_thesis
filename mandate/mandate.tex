% \documentclass[a4paper,twoside,openright]{report}
\documentclass[a4paper]{article}
% \documentclass[landscape,a4paper]{report}
\usepackage[driver=xetex,a4paper]{geometry} % propper margins on a4 paper
% \usepackage[driver=xetex,a4paper,margin=25mm]{geometry} % propper margins on a4 paper
% \usepackage[landscape,driver=xetex,a4paper,margin=25mm]{geometry} % propper margins on a4 paper

\usepackage[hidelinks]{hyperref} % make contents and references clickable in pdf
\usepackage[english]{babel,isodate}\isodate
\usepackage{fancyhdr}
\usepackage{emptypage}
\usepackage{amsmath}

\usepackage{pstricks-add}
%% \usepackage[official]{eurosym}
\usepackage[nottoc,numbib]{tocbibind}

\usepackage{appendix}

\usepackage{graphicx}
\usepackage{caption}
\graphicspath{{./plaatjes/}}
\DeclareGraphicsExtensions{.png}

% used for typeproofs
\usepackage{amsfonts}
\usepackage{amssymb}
\usepackage{bussproofs}

\usepackage[utf8]{inputenc}

% font stuff
\usepackage{fontspec}
\setmainfont[
   BoldFont={Source Sans Pro Black},
   AutoFakeSlant=0.3
]{Source Serif Pro}
\setmonofont{Source Code Pro}

\usepackage{url}
\usepackage{multicol}
\usepackage[section]{placeins}

\usepackage{enumerate}
\usepackage{enumitem}

\usepackage{listings}
\usepackage{amsmath}
\usepackage{xcolor}
\lstset{%
   basicstyle=\footnotesize\ttfamily,
   frame=single,
   breaklines=true,
   postbreak=\raisebox{0ex}[0ex][0ex]{\ensuremath{\color{red}\hookrightarrow\space}},
   keepspaces=true
}

% draft marker
\usepackage{draftwatermark}

\newcommand*\writer{Louis van der Burg}
% \renewcommand{\chaptermark}[1]{%
% \markboth{#1}{}}

%header & footer
\pagestyle{fancy}
\fancyhf{}
\fancyhead[LE,RO]{\nouppercase\leftmark}
% \fancyhead[RE,LO]{\partname\ \thepart}
\fancyhead[RE,LO]{\writer}
\fancyfoot[CE,CO]{\thepage}
% \fancyfoot[LE,RO]{\thepage}

%title page stuff
\fancypagestyle{plain}{%
   \fancyhf{}\fancyfoot[LE,RO]{\thepage}%
   \fancyfoot[CE,CO]{\writer}
\renewcommand{\headrulewidth}{0pt}}

\newcommand\BrText[2]{%
   \par\smallskip
   \noindent\makebox[\textwidth][r]{$\text{#1}\left\{
      \begin{minipage}{\textwidth}
         \lstset{#2}
      \end{minipage}
      \right.\nulldelimiterspace=0pt$}\par\smallskip
   }


\begin{document}

\title{Debugging and expanding Meta-Casanova}
\author{\writer}

\begin{titlepage}
   \input{voorpagina}
\end{titlepage}

\begin{abstract}
\end{abstract}

\section{Introduction}
\subsection{Working title}
Debugging and expanding Meta-Casanova.

\subsection{Motive}
\textbf{expand with there are games, C is used for games, C has problems, MC will solve these problems.}
Video game development is difficult~\cite{blow2004game}.
To make the game development easier \emph{Casanova} was created~\cite{}.

MetaCasanova (from now on called \emph{MC}) comes forth from the language Casanova, hence the name.

Casanova is a language made for building games.
It uses higher order types to make the programming of complex constructs easier.
These constucts are often used in game development.
A few of these constructs will be explained in detail from chapter~\ref{chap:standardlibrary} onward.

Because of the higher order types the compiler for Casanova became complex.
The compiler for Casanova worked, but was still in development.
So when a bug was found in the compiler, fixing it was becoming more and more time consuming.
This frustrated the developers so much a new language was created.

MC was this new language.
MC serves as the language in which the Casanova will be written as a library

At the beginning of the assignment MC was still in development.
The syntax was nearly complete, but untested, and the standard library of MC was just beginning to take shape.
The assignment was created to debug and expand the language and further refine the standard library.
C

\bibliographystyle{ieeetr}
\bibliography{biblio}

\begin{appendices}
   \section{Stakeholders}
% \renewcommand{\abstractname}{Stakeholders}
% \begin{abstract}
   % \hfill
   % \vfill
   % \columnbreak
% \begin{multicols}{2}

   \begin{tabular}
      { l l }
      \textbf{The graduate} & \\
      Name & Louis van der Burg \\
      Student number & 0806963 \\
      E-mail address & louis.burg@hotmail.com \\
      Telephone number & +31 6 14 85 55 79 \\
      & \\
      \textbf{Client} & \\
      Name company & Kennis Centrum Creating 010 \\
      Name company supervisor & Sunil Choenni \\
      E-mail address & h.choenni@hr.nl \\
      Telephone number & +31 6 48 10 03 01  \\
      Function & Lector \\
      Visitors address & Wijnhaven 103 verdieping 6 \\
      Website company & www.creating010.com \\
   % \end{tabular}
   % \begin{tabular}
      % { l l }
      & \\
      \textbf{Company supervisor} & \\
      Name company & Kennis Centrum Creating 010 \\
      Name company supervisor & Sunil Choenni \\
      E-mail address & h.choenni@hr.nl \\
      Telephone number & +31 6 48 10 03 01  \\
      Function & Lector \\
      Visitors address & Wijnhaven 103 verdieping 6 \\
      Website company & www.creating010.com \\
      & \\
      \textbf{School supervisors} & \\
      Name examinator (first teacher) & Giuseppe Magiorre \\
      E-mail address & +31 6 41 78 12 23 \\
      Telephone number & g.maggiore@hr.nl \\
      & \\
      Name assessor (second teacher) & Hans Manni \\
      E-mail address & j.p.manni@hr.nl \\
      Telephone number & \\
      & \\
      \textbf{School co\"ordinator} & \\
      Name graduate coördinator INF/TI & Aad van Raamt \\
      Telephone number & 010 7944993 \\
      E-mail address & A.van.Raamt@HRO.NL \\
   \end{tabular}
% \end{multicols}
% \end{abstract}

\end{appendices}
\end{document}
