\begin{abstract}
   Testing the Meta-Compiler language from a practical perspective.

   The programming language Meta-Compiler (from now on abbreviated as MC) has as purpose simplifying the creation of compilers.
   However the latest implementation of the language has not been tested in the field yet. And we want to know how it fares in the real world.

   Or simply put: "How good is the programming language MC from a practical perspective?"

   We will set up testing criteria and put MC against these criteria.
   The testing will take place in two stages:
   First we will see how MC holds up in theory;
   Then we will test MC by writing a test program with a certain complexity.

   The main research field is compilers and the assignment is executed within Kenniscentrum Creating
   010.
   There will be made use of describing and evaluating MC and the test program.

   The end result will consist of a report and a test program.
\end{abstract}

\chapter{Introduction}
\section{Working title}
   Testing the Meta-Compiler language from a practical perspective.

\section{Motive}\label{sec:motivemandate}
Making compilers for programming languages is a difficult task, as every language is different and needs different features.
From this problem MC came into existence.
The language is made to build compilers.
The process of making compilers is hard, MC strives to make it less hard.
MC also gives the complex code of compilers a clear structure.

However MC has not yet been tested in the field yet, which brings us to the goal of the assignment.

\section{Importance}
The result of this paper will give some insight of the practicality of MC and how far it is from its end goal.
This will be used in the further development and improvement of MC.

\section{Goal}\label{sec:goalsmandate}
The main goal is to test MC with a practical application.
This is done by creating a test application which has the following steps of progressing difficulty:

\begin{enumerate}
   \item For a minimum passing grade of 6: \newline
      A small interactive and graphical application.
   \item For an 8: \newline
      Within the application for a 6, add traversable and changeable data structures.
   \item And for a 10: \newline
      Within the application for an 8, add coroutines.
\end{enumerate}

After which the result is reflected upon.

\section{Problem statement}
Currently there are several working versions of MC available, but none of these implement the entire language.
We want to have a full version of MC which works as desired.
That is to say, it matches the design specifications.
However the last documented version is implemented in the latest working version of MC.
These versions are already proven to be adequate\cite{giuseppe2015mc}.

The newest development version is the one that will be used for this assignment.
This version has no working compiler yet and is therefor easily adapted to bugs found during testing.
The new version also implements new features not present in the previous versions.
Because this newer version is still in development we want to detect any flaws and mishaps in the design during this phase.
This makes otherwise fatal flaws fixable.

The usability of the new version is also not tested yet and we want to see how MC compares to other programming languages.
For this reason we will be testing the current development version of MC.

\section{Research question and sub-questions}
From this problem statement follows the following research question:

\textit{How good is the programming language MC in comparison to other languages?} \newline
Which is divided into these subquestions:
\begin{enumerate}
   \item What constitutes a good programming language?
   \item What is MC?
   \item How does MC work?
   \item How does MC differ with other languages?
   \item Which purpose does MC have?
   \item How practical is MC in its use?
\end{enumerate}

This will also translate to the actual assignments and goals as described in section \ref{sec:goalsmandate}.

\section{The client}\label{sec:clientmandate}
The graduation assignment will be carried out at Kenniscentrum Creating 010.
The company is located in Rotterdam.
\textit{Kenniscentrum Creating 010 is a transdisciplinary design-inclusive Research Center enabling citizens, students and creative industry making the future of Rotterdam}\cite{creating2016home}.

\section{Working environment and tasks}\label{sec:workenvmandate}
During the internship the student will work as a part of the existing research team, whom do research in the Casanova and MC languages.
The student will be doing research and programming and will be supported and helped by the entire research team.
Every two weeks the student will present his progress and receive feedback.


\chapter{Method}\label{ch:methodmandate}
\section{Research methods}
During the internship there will be made use of different research methods.
This stems from the wide scope of the project and the fact that there are multiple research questions which require different approaches.

At the start there will be a preliminary research concerning MC.
This will consist of describing and comparing MC and the ways it works.

After this there will be a basic definition of the test program which will we written to further test MC.

When this is done there will be a evaluation of the test program and MC.

\section{Gathering information}
The information needed will be gathered from the research team involved, via personal interviews, and via published papers concerning the material.

\section{Validation of results}
The validity of the results will be secured by regularly discussing with the client and the supervisor.
Further more there will be programs will be double-checked by the supervisor and the other members of the research team.

\section{Validity and trustworthiness of sources}
The sources used will be published papers only and relative to the subject matter.
Other sources, such as personal communications will be with people who are approved by the supervisor.

\section{Project methods}
During the internship the LEAN-software development method will be used\cite{ries2011lean}.
Using this method gives the ability to quickly iterate through versions and gather knowledge more easily through these iterations.

\section{Risk analyses}
\begin{center}
   \begin{tabular}
      {| p{0.2\textwidth} | p{0.25\textwidth} | l | p{0.3\textwidth} |}
      \hline
      \textbf{Risk} & \textbf{Effect} & \textbf{Possibility} & \textbf{Counter measure}
      \\ \hline
      Quick development of the language. & Because the language becomes bigger as time progresses, the assignment could become too large to do in the set time. & 70\% & The analyses of MC goes on for a set period, which prevents the analyses going an for too long.
      \\ \hline
      The compiler is not finished at the time the programs need to be written. & The programs can not be tested in practice. & 80\% & The programs will be manually checked for errors by the student and the research team.
      \\ \hline
      The company goes bankrupt during the internship & The assignment will not be finished and the student will not be able to graduate. & 1\% & There is nothing the student can do to prevent this.
      \\ \hline
   \end{tabular}
\end{center}

\section{Quality expectations}
MC makes use of type theory and the programs written in it will therefor need to be in line with the type theory, as described in \cite{pierce2002types}.
Because the version of MC used in this internship has no compiler yet, the student will need to manually make sure that the program has no bugs and/or errors in them.

\chapter{Results}
\section{Intended result}
The end result should be an insight into MC, with a focus on how practical the language is in it's current development stage.
This will be done by explaining the language and writing a test program in MC.

% \chapter{Definition of graduation assignment}
% \section{The client}
% See section \ref{sec:clientmandate}.

% \section{Motive}
% See section \ref{sec:motivemandate}.

% \section{Goal}
% See section \ref{sec:goalsmandate}.

% \section{Scope}

% \section{Conditions \& restrictions}

% \section{How is the supervision organized within the company}
% See section \ref{sec:workenvmandate} and chapter \ref{ch:methodmandate}.

% \section{Correlation with other projects}
% During the internship the research team keeps developing MC.
% There are also a two other student doing an internship on MC.
% One is building the compiler for MC and the other is optimizing the code generator.
% The student will have to cooperate with the research team and with both these students.

\chapter{Stakeholders}
   \begin{tabular}
      { l l }
      \textbf{The graduate} & \\
      Name & Louis van der Burg \\
      Student number & 0806963 \\
      E-mail address & louis.burg@hotmail.com \\
      Telephone number & +31 6 14 85 55 79 \\
      & \\
      \textbf{Company supervisor} & \\
      Name company & Kennis Centrum Creating 010 \\
      Name company supervisor & Sunil Choenni \\
      E-mail address & h.choenni@hr.nl \\
      Telephone number & +31 6 48 10 03 01  \\
      Function & Lector \\
      Visitors address & Wijnhaven 103 verdieping 6 \\
      Website company & www.creating010.com \\
      & \\
      \textbf{Client} & \\
      Name company & Kennis Centrum Creating 010 \\
      Name company supervisor & Sunil Choenni \\
      E-mail address & h.choenni@hr.nl \\
      Telephone number & +31 6 48 10 03 01  \\
      Function & Lector \\
      Visitors address & Wijnhaven 103 verdieping 6 \\
      Website company & www.creating010.com \\
      & \\
      Name graduate coördinator INF/TI & Aad van Raamt \\
      Telephone number & 010 7944993 \\
      E-mail address & A.van.Raamt@HRO.NL \\
      & \\
      Name examinator (first teacher) & Giuseppe Magiorre \\
      E-mail address & +31 6 41 78 12 23 \\
      Telephone number & g.maggiore@hr.nl \\
      & \\
      Name assessor (second teacher) & Hans Manni \\
      E-mail address & j.p.manni@hr.nl \\
      Telephone number & \\
   \end{tabular}
