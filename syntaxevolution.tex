\chapter{Syntax evolution}\label{chap:mcevolution}
Here we will look into the overall syntax changes of the language.

\section{Generic type identifiers}
Generic type identifiers are indicated by \verb\'\ together with a name, \verb\'a\.\footnote{As described in section~\ref{sec:basictypeidentifiers}}
The \verb\'\ notation is derived from \emph{ML language}.

In \emph{ML language} the \verb\'\ notation is used to indicate a letter from the Greek alphabet~\cite{harper1986standard}.
\verb\'a\ stands for \emph{alpha}, \verb\'b\ for \emph{beta} and so on.

By using this notation the readability of the language is improved.
It is now clear to the user when generic types are the same or different from each other.


\section{Generic kind identifiers}
Generic kind identifiers are currently unused, but it was thought to be necessary after the implementation of \verb\TypeAlias\.

When working with kinds they are always generic, as they indicate an unknown type.
That was why they were indicated by an \emph{asteriks}, \verb\*\.

When \verb\TypeAlias\ was created it became apparent that the type would be the same in some cases.
This resulted in generic kind identifiers.

The notation devised was similar to the generic type identifiers.
Instead of the \emph{prime} the \emph{hash} was used.

\begin{lstlisting}
TypeFunc "unknownKinds" -> #a -> #b -> #a -> #c
\end{lstlisting}

Here we can see that the first and third arguments are of the same kind and the second argument and the return value are different kinds.

However this was an error in reasoning.

The fact that they are kinds means that they can be any type.
The official notation used for kinds is either \verb\Type\ or \verb\*\.

The syntax of kinds was changed to \verb\Type\.
This way the \verb\*\ was free to be used for other things, like the tuple in section~\ref{sec:basictypefunc}.

This also improved the readability of the language.
It is now clear when a kind is used.

Because of the use of \verb\Type\ it is also instantly clear what a kind is.
This reduces the complexity of the language.


% \section{TypeAlias}
% Before there was \verb|TypeAlias|, \verb|TypeFunc| was used for the same functionality.
% However \verb\TypeFunc\ cannot deconstruct.

% When constructing a new kind with \verb\TypeFunc\ it can not be deconstructed to its original kind.

% This error was noticed when updating the monadic part of the standard library, see section~\ref{sec:standardmonad}.

% \verb\Data\ could not be used as it cannot manipulate types: it can only construct and deconstruct them.
% That is why \verb|TypeAlias| was created.
% It provides contructing and deconstructing functionality with type manipulation.

% This adds expressivity and functionality to the language.
% Expressivity and functionality is especially useful for the user when creating complex programs.


\section{Modules}
Modules have had two major changes, both will be described here.

\subsection{Signature}
Earlier in development \verb|Module| was called \verb|Signature|.
It performed the same functionality as \verb|Module|.

From the language creators perspective \verb|Module| creates a specific signature at compile time.
For the user it looks more like a container or class, when coming from object-oriented programming.

Because the user will be the one actually using the language, the name was changed to what it resembles to the user.
It clarifies the syntax and makes it more predictable for the user.


\subsection{Expanding}
During development there were two ways of inheriting a module, via \texttt{inherit} and via expanding an existing module.
Expanding a module is removed from MC, because of inconsistency issues.

A \verb\Module\ could be expanded via a function.
The function takes a module as argument and expands this module by adding declarations and, optionally, definitions.

\begin{lstlisting}
TypeFunc "expandModule" => Module => Module
expandModule M => M{
  Func "modulo" -> Float -> Float

  a % 10 -> res
  ---------------
  modulo a -> res
}
\end{lstlisting}

The module \verb\M\ is opened up again and \verb\modulo\ is added.
Module \verb\M\ is then returned and contains the new function \verb\modulo\.

This would have been syntactic sugar for creating a new module which inherits from \verb\M\.
In this new module \verb\modulo\ is then added.

\begin{lstlisting}
TypeFunc "expandModule" => Module => Module
expandModule M => Module{
  inherit M

  Func "modulo" -> Float -> Float

  a % 10 -> res
  ---------------
  modulo a -> res
}
\end{lstlisting}

The expand-syntax makes it look like modules are mutable, which they are not.
This makes it inconsistent with the rest of MC and the expand-syntax was removed.

By not using the expanding functionality the language becomes more coherent and in line with itself.
This makes the language friendlier to the user, because there is no two ways of doing things.
It is clear how inherit works and there is no confusion possible.


% \section{priority}
% \textbf{DO THIS!!!!!!!!}
% During the development of the language priorities were needed.
% They are necessary to implement the order of operators being executed.

% For this the

% why the \verb\#>\
% there was a need for priorities
% first only implemented without associativity
% then with right, left being the default


\section{.NET libraries}
When importing and calling .NET libraries the current syntax is using the original .NET syntax:

\begin{lstlisting}
Func "dotNetTest" -> String
dotNetTest -> System.DateTime.Now.ToString()
\end{lstlisting}

This was not always used.
The syntax for .NET imports went through several stages.

The first implementation was the MC inherit order with .NET elements, while using the \verb\^\:

\begin{lstlisting}
dotNetTest -> ToString()^Now^DateTime^System
\end{lstlisting}

This seemed strange considering the method calls .NET has.
The order of the elements was then switched.

\begin{lstlisting}
dotNetTest -> System^DateTime^Now^ToString()
\end{lstlisting}

This syntax seemed out of sync with the rest of the language.
That is why the \verb\^\ was changed back to the \verb\.\.

That is how we arrived at the current syntax used.
The current syntax shows the user explicitly when a .NET call is used.

Using these different syntaxes for .NET and MC creates a clear distinction between the two.


\section{Builtin}
The keyword \verb\builtin\ was first called \verb\primitives\.

For developers this was a good choice, as they indicate the primitives that needed to be built in the compiler.

For the user however this seemed confusing.
The user sees the \texttt{primitives} as types that are built into the language.

The name was changed to \verb\builtin\, to better reflect the way how the user sees this functionality.


% \section{Data}
% why it is not called alias.


\section{Conditionals}
Two sorts of conditionals exist in MC and here we will describe why they are this way.

\subsection{Inside the conclusion}
During development we tried implementing conditionals inside the conclusion of a rule.
This could make the code more compact.

We will take the example of \verb\Func\ from section~\ref{sec:basicfunc}, and implement this.
The code is then compacted to this:

\begin{lstlisting}
add b c -> res
------------------------------------
computeNumber True^buitin b c -> res
\end{lstlisting}

This is just as easy to read as the original.

It does make the conclusion less conclusive as there is now something happening inside the conclusion.
The conclusion is meant to show the input and output of the rule, it is not meant to tell what the function does.

That is where the premises are for.

The compiler also needed to be modified heavily.
It now needed to do computations inside the conclusion.

To keep the language and the compiler modular, a comprise was done.
Types created with \verb\Data\ and \verb\TypeAlias\ can be used as conditionals inside the conclusion.
This is possible because they act as aliases and need no computation to be checked.

Conditionals requiring computation, a comparison of values, will be done in the premises.

This is also syntactically predictable.
Especially because the \verb\Data\ and \verb\TypeAlias\ used in conditionals, are aliases.
They could be replaced with that from which they are constructed.


\subsection{Equals}
When comparing values the \verb\==\ operator is used.
This is done to keep the single equals sign free to be used by the programmer.

The single \verb\=\ could be mistaken for an alias creation or value assignment, when first learning MC.
With the double \verb\==\ there is no such confusion.
