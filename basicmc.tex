\chapter{MC in detail}

\section{What is MC}
MC stands for MetaCasanova and is MC is a declaritive functional language.
\section{Why MC}


\section{Goal}
The goal of MC is to use higher abstractions with type safety.

\section{Basics}
We will now go through the basics of MC.
What systems MC uses, learn how the syntax works and use small code samples to explain the workings.

The changes and reasons why it works the way it currently works will be explained in part~\ref{part:mcexpanded}.
This way we will first have a basic understanding of how MC works and we are more able to understand the reasons why it works this way.

\subsection{Terms, types \& kinds}\label{sec:basiclevels}
With MC you program on three different levels:
\begin{enumerate}[noitemsep]
   \item Terms
   \item Types
   \item Kinds
\end{enumerate}
Terms are the values of variables, like 5, 'c' or "Hello".
Types are the types of variables, like Integer, String or Boolean.
Kinds are for types, what types are to terms.
You could say the kinds are the types of the types.

Types and kinds only exist on compile time and are either resolved or inlined during code compilation.
This will make the executed code much faster when doing complex operations.
The compilation time will suffer from it, but because compilation only happens once, it is a good compromise for faster runtime code.

\subsubsection{Identifiers}
Because there kinds are on such a high level there can be very little said about them, other then the fact that they are the same or different from each other.
This is why a syntax was divised to show this, the \verb|#| notation.
The \verb|#| is supplemented by a identifier, which is usually a letter or word.

For example you can use \verb\#a\ or \verb\#foo\.

On type level the same thing is done when the specific type is not yet known.
Instead of the \verb\#\ the \verb\'\ is used.
So instead of \verb\#a\ and \verb\#foo\, it would be \verb\'a\ and \verb\'foo\.

This is usefull when declaring generic functions.

\subsection{Runtime and compile time}
In MC there is a clear difference between compile time.
Everything on the term level is on runtime and everything on type and kind level is on compiletime.

\subsection{Func}
First we need to declare a function.
For this we use the keyword \verb|Func|.

\begin{lstlisting}
Func "foo" -> Bool -> Int -> Int -> Int
\end{lstlisting}

Here we see that the function \emph{foo} is declared by a String, which is the name of the function, a Boolean and three Integers.
The parameters are seperated by the \verb|->|.
The last parameter is the return type of the function and all the parameters between the name and the return type, are the arguments of the function.

Now let's define the function just declared.
\begin{lstlisting}
add b c -> res
---------------
foo True b c -> res
\end{lstlisting}

A function is defined by a \emph{rule} which has one or more \emph{premises}.
They are separated by the \emph{bar}.
The rule stands underneath the bar and the premises above.
This syntax is similiar to the one used in natural deduction~\cite{}.

In this case we have one premise above the bar.

We can also have multiple function definitions, like so:
\begin{lstlisting}
add b c -> res
---------------
foo True b c -> res

mul b c -> res
---------------
foo False b c -> res
\end{lstlisting}

When this happens the rules get executed in order.
If the above rule failes and the rule beneath it is matched.

Let's take the following function call:
\begin{lstlisting}
foo true 5 3 -> newValue
\end{lstlisting}

The first rule to be tested is the one written first in the source code, so:
\begin{lstlisting}
foo True b c -> res
\end{lstlisting}
This rule will fail, because the first argument is false and not true.
So this function definition will not be executed.

The next rule to be matched on, is:
\begin{lstlisting}
foo False b c -> res
\end{lstlisting}
And this rule will get executed because the first argument matches.
The rule will be executed and the result will be put in \verb|newValue|.

Ofcourse the other arguments passed will also need to match.

When no parameters are given, the function acts like a variable.
\begin{lstlisting}
Func "bar" -> Int
bar -> 5
\end{lstlisting}

As we have seen \verb|Func| uses types in it's declaration and terms in it's definition.
The types enables to make sure the function will always be able to execute, when the argumenttypes match those declared.
This also means that function declared by \verb|func| are on runtime.

As we will see in section~\ref{subsec:typefunc} MC also has the ability to declare and define functions on a level higher.

\subsection{Data}
With \verb|Data| we can create constructor and deconstructors for a new type.
\begin{lstlisting}
Data "Left"  -> String -> String | Float
Data "Right" -> Float  -> String | Float
\end{lstlisting}
Here we create two constructors whom both go to the same type, \verb_string | float_.
And when we have a term with type \verb_string | float_, so either of type \verb|Left| or \verb|Right|, we can match to get the original term back.
In this case a \verb|String|, when it matches on \verb|Left|, or a \verb|Float|, when it matches on \verb|Right|.

Because \verb|Data| can work two ways, constructing and deconstructing, it can be seen as an alias for types.

\subsection{TypeAlias}
\verb|TypeAlias| is the same as \verb|Data| only on a level higher.
It constructs and deconstructs kinds.

\begin{lstlisting}
TypeAlias #a => "*" => #b => #c
'a * 'b => tuple<'a 'b>
\end{lstlisting}

Here we see how a tuple gets created with the help of \verb|TypeAlias|.
We also see how we can declare that an argument will be in front of the the name of the \verb|TypeAlias| function.
This can also be done with \verb|Func|, \verb|Data| and \verb|TypeFunc|.

There is one limitation to the use of argument before the name of the declaration, there may only be one.
This is because of the parser used.
The parser would become overly complex when adding multiple arguments in before the name of the declaration.


The syntax reflects the difference between the levels used, to easily know on which level the function operates.

The declaration uses the \verb|=>| to indicate it operates on the type and kind level.
Just like \verb|->| indicates the operation on term and type level.

\subsection{TypeFunc}
\verb|TypeFunc| creates a function on type level.

\begin{lstlisting}
TypeFunc "bar" => #a => #b
\end{lstlisting}

It works much like \verb|Func| only instead of terms it uses types and instead of types it uses terms.

This means that \verb|TypeFunc| works with kinds in its declaration to check the types used in the definition.
Which is the same principle as \verb|Func|, which uses types in its declaration to check the terms used in the definition.

\begin{lstlisting}
a => (c * d)
(d * c) => res
---------------
bar a => res
\end{lstlisting}

As we can see function \emph{bar} takes a type, deconstructs the type, switches the arguments of this type and returns that as the result.

As with \verb|TypeAlias| it uses the \verb|=>| to indicate it works on type and kind level.

\subsection{Module}
\verb|Module| is used to create an interface on compiletime.
It is a collection of functions and can be inherited.

It does not fit within the levels defined in section~\ref{sec:basiclevels}.


\begin{lstlisting}
TypeFunc "MonoidAdd" => #a => Module
MonoidAdd 'a => Module {
  Func 'a -> "+" -> 'a -> 'a
  Func "identityAdd" -> 'a
}
\end{lstlisting}

Here we see a basic inteface for a generic add functionality.
As shown it can contain \verb|Func|s, but can also contain \verb|Data|, \verb|TypeFunc|, \verb|TypeAlias| and even another \verb|module|.

The \verb|Func|s within the Module do not have to be defined, they can just be the declarations.
This way we can define different behavior for every instance of the module.

But it is possible to have a function declared and defined within a module, this creates a property-like behavior within the module.
When this is done, the user must be certain that the function will always be executable.
If this is not the case the typechecker will fail.

When we want to define the above declared \verb|MonoidAdd| for an integer, it will look like this:

\begin{lstlisting}
MonoidAdd Int = {
  Int -> "+" -> Int -> Int
  identityAdd -> 0
}
\end{lstlisting}

The \verb|Func|s are defined as the normally are, only now they are wrapped within a \verb|Module|.
The \verb|Module| is defined in much the same way other functions are defined.
They simply contain function definitions within itself.

\subsubsection{The caret}
When we want to call a function within a module from outside the module the \verb\^\ is used.

\begin{lstlisting}
Func "caretTest" -> Int
caretTest -> identityAdd^Int
\end{lstlisting}

This creates a constant named \verb\caretTest\ with the same value as the \verb\identityAdd\ from the module \verb\Int\.
The \verb\^\ can be seen as the `\verb\.\` (the dot) used in C\#.
The main difference with most languages implementing such a feature, is the hierarchy of the elements.
In C\# the first element is the uppermost class.
The \verb\^\ puts the elements in the oposite order.

In the example we first say what we want and then say where to find it.
This has the advantage of knowing immediately what you are using and if you want to know where it actually comes from you can read on.

\section{Import}
In section~\ref{sec:basicmodule}
\verb\Import\ is used when importing or including other files in the current file.
It is called \verb\import\ because it literaly imports all the code of the imported file into the current file.
The programmer cannot directly use the declarations and definitions used in the imported file.
There is a syntax element that


\section{Comments}



\bigskip

Now that we know the basics of MC lets take a look at how the syntax has come to this point.
