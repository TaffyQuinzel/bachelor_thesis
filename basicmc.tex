\chapter{MC in detail}\label{chap:basicmc}
We will now look at MC.
%... in the state after the project.
After we have a basic understanding of MC, we will look at the improvements made.

Because the latest version of MC is not documented, the following information comes from interviews with the research team~\cite{researchteaminterview}.
This chapter also serves as the documentation for this version of MC.


\section{What is MC}
MC is a declarative functional language.
It tries to be a completely pure language, which means no side-effects are allowed directly.


\section{Goal}
The goal of MC is to use higher order types with type safety, in a natural way of programming.
These higher order types give the ability to resolve certain computations at compile time, which would normally be performed at runtime.
This can substantially increase performance at runtime.
The higher order types also add expressive power by being able to capture abstractions.

MC also supports the .NET library natively.
Because .NET has mutables~\cite{dotnetdescription}, MC can have side-effects through the use of .NET.

MC aims to be as flexible as possible, with the safety of a strong type system.


\section{Basics}
We will now go through the basics of MC.
We will look at how the syntax works and use small code samples to explain them.

The improvements made to these basics will be given in part~\ref{part:mcexpanded}.
This will give us a better understanding of why and how MC evolved.

Inside the code blocks a line wrap is indicated by \ensuremath{\color{red}\hookrightarrow\space}.


\subsection{Terms, types \& kinds}\label{sec:basiclevels}
With MC you program on three different levels:
\begin{enumerate}[noitemsep]
   \item Terms
   \item Types
   \item Kinds
\end{enumerate}
Terms are the values of variables, like \verb\5\, \verb\'c'\ or \verb\''Hello''\.
Types are the types of variables, like \verb\Integer\, \verb\String\ or \verb\Boolean\.
Kinds are to types, what types are to terms.
You could say that kinds are the types of the types.


\subsection{Func}\label{sec:basicfunc}
First we need to declare a function before defining it.
For this we use the keyword \verb|Func|.

\begin{lstlisting}
Func "computeNumber" -> Boolean -> Int -> Int -> Int
\end{lstlisting}

Here we see that the function \texttt{computeNumber} is a function which goes from a \verb\Boolean\ and two \verb\Integers\ to an \verb\Integer\.

The parameters are separated by the ASCII arrow, \verb|->|.
The last parameter is the return type of the function and all parameters between the name and the return type are the arguments of the function.

A function is defined by one or multiple \emph{rules}.
A \emph{rule} consists of a \emph{conclusion} and can contain multiple \emph{premises}.
They are separated by the \emph{bar}.
The conclusion is placed underneath the bar and the premise(s) above.

Now we define a rule for \verb\computeNumber\.

\begin{lstlisting}[xleftmargin=5.0ex,numberstyle=\tiny\color{gray},numbers=left]
a == True
add b c -> res
--------------------------
computeNumber a b c -> res
\end{lstlisting}

The conclusion is on the fourth line, the bar on the third and the premise on the first and second.
The conclusion has the name of the function and the input arguments on the left of the arrow and the output argument on the right.
This syntax is similar to that of natural deduction~\cite{prawitz1965natural}.

\subsubsection{Locally defined identifiers}
The identifiers named in the conclusion of a function are locally defined.
Within the rule of \verb\computeNumber\ the identifiers \verb\a\, \verb\b\ and \verb\c\ are examples of these local identifiers.


\subsection{Multiple rules}
There can also be multiple rules which form the function definition:

\begin{lstlisting}
a == True
add b c -> res
---------------
computeNumber a b c -> res

a == False
mul b c -> res
---------------
computeNumber a b c -> res
\end{lstlisting}

When this happens the rules are activated in order of definition.
A rule fails when the arguments do not match.
If the rule fails, the next rule is activated.
If none of the rules match, the program crashes.

Let us go through this process with an example.

First we will try the following function call, from inside a premise:
\begin{lstlisting}
computeNumber False 5 3 -> newValue
-----------------------------------
\end{lstlisting}

The first rule to be attempted is the one written first in the source code, so:
\begin{lstlisting}
a == True
add b c -> res
--------------------------
computeNumber a b c -> res
\end{lstlisting}

This rule will fail, because argument \verb\a\ has the value \verb\False\ and not \verb\True\.
So this rule will not be executed.

The next rule to be attempted is:
\begin{lstlisting}
a == False
mul b c -> res
--------------------------
computeNumber a b c -> res
\end{lstlisting}
Argument \verb\a\ is still \verb\False\ and \verb\b\ and \verb\c\ are both \verb\Integers\, so the rule matches.
The rule will be executed and the result will be put in \verb|newValue|.


\subsection{Constants}
Constants can be created by using \verb\Func\ without any parameters.
The return type still has to be declared.

\begin{lstlisting}
Func "bar" -> Int
bar -> 5
\end{lstlisting}

Here the constant \verb\bar\ is set to the value \verb\5\.

We also see that not every rule has a \emph{bar}.
This is possible only when no premise is needed in the rule.

\bigskip

As we have seen \verb|Func| uses types in its declaration and terms in its definition.
The types enable the detection of wrong function calls during compile time.

Important to note is that functions declared by \verb\Func\ are runtime constants.
MC also has the ability to declare and define functions that run at compile time.\footnote{As will be explained in section~\ref{sec:runcompiletime} and shown in section~\ref{sec:basictypealias}.}

\subsection{Generic type identifiers}\label{sec:basictypeidentifiers}
To indicate generic types the \verb\'\ notation is used.

This is enhanced with the use of \emph{identifiers} after \verb\'\, for example: \verb\'a\ or \verb\'b\.

These give the generic type a name and can be used to express if one generic type is the same as another generic type.


\subsection{Functions as parameters}
Aside from variables, functions can also be given as parameters.
This needs to be specified in the declaration of the function.

\begin{lstlisting}
Func "isThisEven" -> ('a -> 'b) -> 'a -> 'd
\end{lstlisting}

Here we see that the first argument \verb\('a -> 'b)\ is a function.
This is clear because the entire argument goes from \verb\'a\ to \verb\'b\, \verb\'a\ is the input and \verb\'b\ the output of the function \verb\('a -> 'b)\.


\subsection{Data}\label{sec:basicdata}
With \verb|Data| we can declare a constructor and deconstructor for a new type.
Because \verb\Data\ can work two ways, by constructing and deconstructing types, it effectively creates an alias for types.

Say that we want to create a union, a type which can be any of the predefined types within a set~\cite{igarashi2006union}.
For this we need a way to construct the union type and a way to deconstruct the union back to its original type.
The deconstructor is needed to get to know which of the constructors inside the union is actually used.
Only then can we know which of the types inside the union is used.

As an example we will create a union of a String and a Float.
\begin{lstlisting}
Data "Left"  -> String -> String | Float
Data "Right" -> Float  -> String | Float
\end{lstlisting}

Here we create two constructors whom both go to the same type, \verb\String | Float\.
\verb\Left\ and \verb\Right\ can be seen as an alias for the \verb\String | Float\ union.

Let us see how this is used.
We need a function which takes a union as argument.

\begin{lstlisting}
Func "isFloat" -> ( String | Float ) -> Boolean
\end{lstlisting}

The parentheses in the declaration can be left out, but are used to clarify that the union is one argument.

In the definition we want to check which type the union is.
This can be done directly in the conclusion of a rule.

\begin{lstlisting}
isFloat (Right x) -> True
\end{lstlisting}

The deconstructor of the argument given automatically checks if the types match.
This can be done because they are aliases created by \verb\Data\.
Aliases are automatically resolved to their basic types by the compiler.

Conditionals using aliases can be put in the conclusion.
Other conditionals have to be put in the premises, because they need to be computed.\footnote{As is shown in the example of section~\ref{sec:basicfunc}}

To make the definition of \verb\IsFloat\ complete we also need a rule that checks for \verb\String\.

\begin{lstlisting}
isFloat (Left x) -> False
\end{lstlisting}

Now we will test the function \verb\isFloat\ with the help of constant \verb\iAmFloat\.

\begin{lstlisting}
Right 9.28 -> iAmFloat
\end{lstlisting}

\verb\iAmFloat\ has type \verb\String | Float\ and can now be passed to \verb\isFloat\.

\begin{lstlisting}
isFloat iAmFloat -> output
\end{lstlisting}

When \verb\iAmFloat\ is deconstructed to its original type it will match with the first rule of \verb\isFloat\.
The identifier \verb\output\ now contains the value \verb\True\.


\subsection{Runtime and compile time}\label{sec:runcompiletime}
So far we have seen the \verb\->\ arrow, which indicates a function is executed on runtime.

To indicate that a function is resolved or inlined at compile time, the big ascii arrow, \verb\=>\, is used.

Types are resolved or inlined at compile time, except when they create data structures.
These data structures keep existing on runtime.
Kinds are also resolved or inlined at compile time.

We will see this starting from the next section onwards.


\subsection{TypeAlias}\label{sec:basictypealias}
Because the \verb\|\ (\emph{pipe}) operator\footnote{Used in section~\ref{sec:basicdata}} is not built in the language we need to create it.
To do this we need to manipulate types.

To make \emph{pipe} work, we both need a constructor and deconstructor that works with kinds.

\verb\TypeAlias\ is the same as \verb\Data\ only on a higher level of abstraction.
It constructs and deconstructs kinds.

With \verb\TypeAlias\ we can create a generic \emph{pipe} with which we can create unions.

\begin{lstlisting}
TypeAlias Type => "|" => Type => Type
\end{lstlisting}

Here we see how the generic \emph{pipe} is declared.
It takes two parameters of any type and returns a type.

The \verb\Type\ is a kind and indicates an unknown type is used.

Parameters may also be infix, as shown in the example above.


\subsubsection{Infix parameters}
There is a limitation to the use infix arguments, namely there may only be one.
This is because of the parser\footnote{A parser analyses the source code and puts it in a structure easily used by the rest of the compiler.} used for the bootstrap compiler of MC.
The parser would become overly complex when adding multiple arguments on the left.

Because the benefits of having multiple arguments in front of the function name does not outweigh the time it consumes to expand the parser, the choice was made to only allow one infix argument.


\subsection{TypeFunc}\label{sec:basictypefunc}
\verb|TypeFunc| creates a function on type level.
It can perform computations with both types and terms.

We will demonstrate the power of \verb\TypeFunc\ with a function that manipulates a type.
The function will take a tuple and return the same tuple, but with switched types.

First we need to declare a tuple with \verb\TypeAlias\.

\begin{lstlisting}
TypeAlias Type => "*" => Type => Type
\end{lstlisting}

Now that we have a tuple we can use it in the function declaration.
The tuple argument will have generic types so it can work with any tuple.

\begin{lstlisting}
TypeFunc "switch" => Type => Type
\end{lstlisting}

Next we have the definition.
Here we need to deconstruct the tuple to its original arguments.
These arguments can then be used to create the return tuple.

\begin{lstlisting}
a => (c * d)
(d * c) => res
---------------
switch a => res
\end{lstlisting}

Here we can see how the types inside the tuple are switched and returned as a new tuple.

{
   \begin{wraptable}{r}{0.3\textwidth}
      \begin{tabular}
         { | l | l | l | }
         \hline
         \textbf{Runtime} & \textbf{Compile time} \\
         \hline
         \verb\->\ & \verb\=>\ \\
         \hline
         \verb\Data\ & \verb\TypeAlias\  \\
         \hline
         \verb\Func\ & \verb\TypeFunc\ \\
         \hline
      \end{tabular}
      \caption{Relations between the keywords and arrows in MC}
      \label{table:relationskeywords}
   \end{wraptable}

   \subsubsection{Parameters}
   Important to note is that both \verb\TypeAlias\ and \verb\TypeFunc\ can use kinds, types and terms.
   Because they work on compile time only, they cannot replace \verb\Data\ and \verb\Func\.

\verb\Data\ and \verb\Func\ exist for explicit runtime functions.
   The relations between the keywords can be seen in table~\ref{table:relationskeywords}.


   \subsection{Modules}\label{sec:basicmodule}
}
\verb|Module| is used to create a container at compile time.
It is a collection of function declarations and definitions.
Having a collection of functions is useful when they serve a combined or general purpose.

It does not fit within the levels defined in section~\ref{sec:basiclevels}.
Modules can only be declared by \verb\TypeFunc\.

Say we want to create a container for every type which does an addition of two identifiers of these types.
The addition could be declared when we need them or we can put them in a \verb\Module\.

When we create an addition container using a module, we do not have to type out the addition functions for every type.
We can just call the module to create the functions for us.

A \verb\Module\ is declared with a \verb\TypeFunc\.
\begin{lstlisting}
TypeFunc "Add" => Type => Module
\end{lstlisting}

Here we declare \verb\Add\ to take a type and return a \verb\Module\.

We then want to define the module \verb\Add\ with the \verb\+\ operator and the \verb\identityAdd\ constant.

\begin{lstlisting}
Add 'a => Module {
  Func 'a -> "+" -> 'a -> 'a
  Func "identityAdd" -> 'a
}
\end{lstlisting}

This creates a basic container with generic addition functionality.

As shown above \verb\Module\ can contain \verb|Func|, but can also contain \verb|Data|, \verb|TypeFunc|, \verb|TypeAlias| and another \verb|Module|.

The declarations within the Module do not have to be defined.
This way we can define different behavior for every instance of the module.

It is possible to have a function declared and defined within a module.

When we want to define the above declared \verb|Add| with an \verb\integer\, we instantiate the module over \verb\Int\.
We then define the functions which are declared within \verb\Add\ to finish the module.

\begin{lstlisting}
TypeFunc "IntAdd" => Module
IntAdd => Add Int {
  Int -> "+" -> Int -> Int
  identityAdd -> 0
}
\end{lstlisting}

\verb\Func\s are defined as they normally are, only wrapped inside a \verb\Module\.


\subsubsection{The caret}
When we want to call a function within a module from outside the module, the \verb\^\ (\emph{caret} operator is used.

\begin{lstlisting}
Func "caretTest" -> Int
caretTest -> identityAdd^IntAdd
\end{lstlisting}

This creates a constant named \verb\caretTest\ with the same value as \verb\identityAdd\ from the module \verb\IntAdd\.
Important to note is that we cannot call a function outside the module from within the module.
The \emph{caret} notation is derived from \LaTeX.
In \LaTeX\space the \emph{caret} is used to indicate superscript~\cite{beebe1992latex}.
This is used is MC to show that one element is part of another element like so: \verb~identityAdd~\textsuperscript{\texttt{IntAdd}}.


\subsubsection{Inherit}
Modules can also inherit from other modules.
This is done with \verb\inherit\.

We will create a module which inherits from the \verb\Add\ module.
The operator \verb\-\ will then be added as well.

\begin{lstlisting}
TypeFunc "GroupAdd" => Type => Module
GroupAdd 'a => Module {
  inherit Add 'a
  Func 'a -> "-" -> 'a -> 'a
}
\end{lstlisting}

The module \verb\Add\\footnote{As used in section~\ref{sec:basicmodule}} is instantiated with \verb\'a\.
The created \verb\Add\ module is then inherited into \verb\GroupAdd\.

The function declarations and definitions of the inherited module are directly usable within the new module.

If a module takes another module as an argument, the module can also directly inherit.

\begin{lstlisting}
TypeFunc "GroupAdd" => Add => Module
GroupAdd M => Module {
  inherit M
}
\end{lstlisting}

Now everything from module \verb\M\ is inherited into \verb\GroupAdd\.

When inheriting a module which contains an inherit, these are also inherited in to the current module.
This works recursively.


\subsection{Priority}
We can also give a priority together with a declaration.
The priority indicates the order of function execution.

The priority is indicated by the \verb\#>\ (\emph{hash arrow}) and is placed after the declaration.
When we look at \verb\Data\ it will look like this:

\begin{lstlisting}
Data "Left"  -> String -> String | Float  #> 7
Data "Right" -> Float  -> String | Float  #> 5
\end{lstlisting}

And with \verb\Func\ and \verb\TypeFunc\ it is used like this:
\begin{lstlisting}
Func "bar" -> Int  #> 12
TypeFunc "foo" => Float => 'a => 'b  #> 9
\end{lstlisting}

The \emph{hash arrow} is also used to indicate associativity.
Everything is left associative by default, but if we want to make it right associative we can do that by placing an \verb\R\ after the \emph{hash arrow}.

\begin{lstlisting}
Func Int -> "\" -> Int -> Int  #> R
\end{lstlisting}

Priority and associativity can also be used in combination with each other.

\begin{lstlisting}
Func Int -> "\" -> Int -> Int  #> 25 R
\end{lstlisting}


\subsection{Import}
\verb\import\ is used when importing other files in the current file.

If the file imported has any imports, those are ignored.
Unlike \verb\inherit\, \verb\import\ is not recursive.

The programmer can directly use the declarations and definitions used in the imported file.
When the programmer wants to explicitly specify the use of a function from the imported file, the \emph{caret} is used.

For example we can import the file \emph{vector} and use the \verb\Vector2\ type from it, as seen below:

\begin{lstlisting}
import vector

Func "location" -> Vector2
location -> createVector2^vector 8.9 19.0
\end{lstlisting}

Importing of MC files always works with non capitalized import statements, even when the actual files does contains capital letters.
This is done to differentiate between MC imports and .NET imports.

The order of elements is the same as with modules, only when using .NET imports there is a difference in syntax.
When calling a .NET function the .NET syntax is used to differentiate between MC and .NET functions.

\begin{lstlisting}
import System

Func "dotNetTest" -> String
dotNetTest -> DateTime.Now.ToString()
\end{lstlisting}

Here \verb\dotNetTest\ becomes a String which contains the current date and time.

.NET functions can be used on both run- and compile time

\subsection{Builtin}
Some things cannot be created with MC and need to be buildin in the compiler.
An example of this is the boolean data type.

When using such a builtin literal the keyword \verb\builtin\ is used.
We can now correctly implement the function \verb\computeNumber\\footnote{As used in section~\ref{sec:basicfunc}} using the \verb\builtin\ keyword:

\begin{lstlisting}
a == True^builtin
add b c -> res
---------------
computeNumber a b c -> res

a == False^builtin
mul b c -> res
---------------
computeNumber a b c -> res
\end{lstlisting}

It can be seen as if \verb\False\ and \verb\True\ are imported from the language itself.


\subsection{Lambdas}
MC also has anonymous functions to implement lambdas~\cite{barendregt1984lambda}.

The basic syntax is as one would expect of a lambda, it has arguments and returns what the function body of the lambda returns

\begin{lstlisting}
(\ arguments -> functionBody )
\end{lstlisting}

In practice this would look like:

\begin{lstlisting}
(\ a -> divide 10 a )
\end{lstlisting}

Lambdas do not need to be declared because they always are generic functions.
The typechecker will check lambdas the same way as any other generic function.
From the context of what the function body does together with the types of the arguments, it can deduce what the output type should be.


\subsection{ArrowFunc}\label{sec:basicmcarrowfunc}
\verb\ArrowFunc\ always needs an argument on the left of the name and at least one function as argument on the right.
A unique feature of \verb\ArrowFunc\ is the two syntax options it has when the function gets called.

The first syntax option is a regular function and the second converts the function to a lambda.

As with \verb\Func\ it needs a declaration and a definition.

\begin{lstlisting}
ArrowFunc Int -> ":~>" -> (Int -> Int) -> Int
f a -> res
-------------
a :~> f -> res
\end{lstlisting}

Here \verb\:~>\ takes an integer and a function that takes an integer and returns an integer.

The declaration and definition are the same as the rest of MC.

When we call \verb\:~>\, there is the regular way:

\begin{lstlisting}
a :~> f -> res
\end{lstlisting}

And there is the second way:

\begin{lstlisting}
{a :~> out
  f out} -> res
\end{lstlisting}

The brackets are used to indicate that everything inside them belongs together.
When an \verb\ArrowFunc\ is called like this it gets converted to a lambda.
The first argument is then used as the argument for that lambda.

\begin{lstlisting}
(\ out -> f out ) a -> res
\end{lstlisting}

The result from this is then returned as the output in \verb\res\.

This syntax exists for when the nesting of lambdas occur.
Lambdas become unreadable when nested.

\begin{lstlisting}
(\ b -> (\ c -> f c ) b ) a -> res
\end{lstlisting}

When using the \verb\ArrowFunc\ syntax it nests neatly.

\begin{lstlisting}
{a :~> b
  {b :~> c
    f c}} -> res
\end{lstlisting}

This syntax stems from the use of the \emph{bind} function of monads, which will be explained in further detail in section~\ref{sec:standardmonad}.


\subsection{Partial application}
Lastly MC supports partial application.
This means that you do not have to give all the arguments to a function.
When this is done a closure is created and given as output.

We will create a closure with the function \verb\add\.
\verb\add\ takes two \verb\Integers\ and returns an \verb\Integer\.

In the definition we will say the two arguments will be added together.
\begin{lstlisting}
Func "add" -> Int -> Int -> Int

a + b -> res
--------------
add a b -> res
\end{lstlisting}

Now we will give only one argument to \verb\add\ and create a new function \verb\addThree\.

\begin{lstlisting}
add 3 -> addThree
\end{lstlisting}

With this construct we have created the partially applied function \verb\addThree\.
When we look at what it contains we see that it has a closure with a partially implemented \verb\add\.

\begin{lstlisting}
3 + b -> res
--------------
add 3 b -> res
\end{lstlisting}

We can now give \verb\addThree\ its second argument to complete the closure.
This can be done multiple times.

\begin{lstlisting}
addThree 6 -> foo

addThree 7 -> bar
\end{lstlisting}

The identifier \verb\foo\ has a value of \verb\9\ and \verb\bar\ has a value of \verb\10\.


\subsection{The main function}
While MC has no specific main function it does specify which function acts the main function.
The final function in the file compiled acts as the main function and is executed when the program is run.



% \textbf{MAYBE REMOVE THIS???? CHECK WITH REST OF DOCUMENT}
% \subsection{Type annotations}\label{sec:basictypeannotations}
% The angle brackets are used as type annotations for \verb\pipe\.
% They indicate that \verb\'a\ and \verb\'b\ are part of the type \verb\pipe\.
% This is only used with generic types, because it makes it easier for the parser to resolve the types.\footnote{More on this in section~\ref{sec:syntaxtypeannotations}}

% When type annotations are used there can be no partial application.
% The type annotations need to know what types it gets or they will fail.
% This is further explained in section~\label{sec:syntaxtypeannotations}.

\bigskip

Now that we know the basics of MC lets take a look at how the syntax has evolved and improved.
