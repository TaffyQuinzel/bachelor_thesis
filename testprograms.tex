\chapter{Test programs}\label{chap:testprograms}
There have been two test programs written to test the language.
The first is a physics simulation to test the updatable records.

Because the final goal of the language is to make creating games easier, the second test program is a game.

Neither of the programs are implemented with the monads, because of time constraints.

Both of the programs do not have their graphical parts implemented, because it would remove the focus from MC
The graphical parts of the programs would utilise the XNA framework and the focus would then be on how XNA is implemented within MC.
We want to test how MC works and not how it handles a specific library.
For this reason the graphical parts of the programs are left out.


\section{Bouncing ball}
Bouncing ball is the physics simulation used as a preliminary test of updatable records.
It will show how the updatable \verb\RecordEntry\ are used in practice.

This program bounces a ball up and down.

The ball will have two properties, velocity and position, which will implemented as records.
Both of these will then be put inside the updatable \verb\RecordEntry\ of the ball.

The \verb\update\ function will then add implement how the ball bounces.

First we import prelude, record and the XNA framework.
The Microsoft XNA library will be used for the \verb\Vector2\ it has and the Thread library will be used to call the \verb\Sleep\ function.

\begin{lstlisting}
import Microsoft^Xna^Framework
import System^Threading^Thread
import record
import prelude
\end{lstlisting}

The recordentry of the ball will be build bottom-up.
This is done because every recordentry needs to have a \verb\rest\ as an argument.

\verb\Velocity\ is created first, because it needs to be updated less.
It keeps deep searches of the recordentry to a minimum.

\begin{lstlisting}
Record "Velocity" Vector2(0.0f, 98.1f) Empty
\end{lstlisting}

Then \verb\Position\ will be created with \verb\Velocity\ as its \verb\rest\ argument.

\begin{lstlisting}
Record "Position" Vector2(100.0f, 0.0f) Velocity
\end{lstlisting}

Lastly we create the updatable recordentry \verb\Ball\.
This is done in one line by creating a recordentry entry and passing it directly as argument to \verb\updatableRecord\.
\verb\Position\ is used as \verb\rest\ argument to complete the record.

\begin{lstlisting}
UpdatableRecord (Record "Ball" unit Position) => Ball {
\end{lstlisting}

We then need to define the \verb\update\ function.

The \verb\Position\ and \verb\Velocity\ RecordEntry are put in identifiers, so we can use them directly.

\begin{lstlisting}
  get^e "Position" Rest^e => position
  get^e "Velocity" Rest^e => velocity
\end{lstlisting}

Then the height of the ball is checked and the according calculations executed.

\begin{lstlisting}
  (if (Y^Field^position <= 500.0f) then
    ((set^e "Position"
           (Field^position Vector2.+ (Field^velocity Vector2.* dt))
           Rest^position)
     (set^e "Velocity"
            (Field^velocity Vector2.+ (Vector2(0.0f, 98.1f) Vector2.* dt))
            Rest^velocity))
  else
    (set^e "Position"
           Vector2(X^Field^position, 500.0f) Vector2.- Field^velocity)) -> res
  --------------------------------------------
  update e dt -> res
}
\end{lstlisting}

The final step is to create the \verb\run\ function to start the program.
It will call the update function of \verb\Ball\, sleep for a set amount of time and then call itself.

\begin{lstlisting}
Func "run" -> RecordEntry -> String

update^ball ball time -> ball'
Sleep(1000)
run ball' (time +^Int 1000) -> res
---------------
run ball time -> res
\end{lstlisting}

This creates an infinite loop, so the ball will keep on bouncing.
A return value will never be set as it will keep calling itself until the program stops.

% Finally we start the program by calling
% \textbf{ask guys if the last function will be automatically executed}

\section{Tron}
Tron is the game and utilises user input.

Tron~\cite{troncitation} is played with two or more players.
Each controlling a bike which leaves a trail.
The bikes can move freely within the playing field, but cannot stop, slow down or speed up.
The trails the bikes leave behind become walls.

A bike may not hit a wall or trail, when it does it is destroyed.
The player who remains driving is the winner.

The game tron has been chosen because of the minimalistic graphics.

Tron has been split up in two three parts, bike, playfield and main.
The main file calls all the needed parts and starts the game.
The playfield contains all the bikes.

Splitting the code will keep all the parts clean and specific to their function.

The bikes and playfield will be records.
Some of the records will be made updatable records when necessary.

We start by looking at the bike.


\subsection{Bike}
First we import a few things.
The XNA framework is needed for the \verb\vector2\.
The \verb\System.Windows.Input\ will be needed for the controls of the bike.
Lastly we will import recordentry and prelude.

\begin{lstlisting}
import Framework^Xna^Microsoft
import System^Windows^Input
import record
import prelude
\end{lstlisting}

The bike needs several things to work in the game.

A bike needs a record which keeps track of its current position.
The position needs to be updated so we use an updatable record.
The position record is created seperately and included in the bike later.

This is done to make the definition of the bike smaller.

The \verb\RecordEntry\ is created in the premise and then passed to \verb\UpdatableRecord\.

\begin{lstlisting}
TypeFunc "PositionRecord" => String => Vector2 => Record => Record
RecordEntry label pos rest => r
--------------------------
PositionRecord label pos rest => UpdatableRecord r => Module{
\end{lstlisting}

The created record \verb\r\ will not be explicitly inherited as that happens with the \verb\UpdatableRecord\ function.
We only need to define the update function.

The update function takes a record and a tuple.
The tuple consists of delta time and speed.
This is then used to calculate the new position.

A new record containing the new position is then created and returned as the result.

\begin{lstlisting}
  dts -> (dt,speed)
  Field^p + (dt * speed) -> newPos
  set^p label^p newPos Rest^p -> res
  -----------------------------------
  update p dts -> res
}
\end{lstlisting}

Note that the \verb\+\ and \verb\*\ operators are inferenced to be the \verb\Vector2\ operators.

Now that we have an updatable position we also need an updatable trail.
This trail changes when the bike moves.

As with \verb\PositionRecord\ an we first create a record which is then passed to \verb\UpdatableRecord\.

\begin{lstlisting}
TypeFunc "TrailRecord" => String => Vector2 => Record => Record
RecordEntry label field rest => r
----------------------------
TrailRecord label field rest => UpdatableRecord r => Module{
\end{lstlisting}

The update function only needs to create a new trail record with the original record as its \verb\Rest\.
This way the trail is a record of points.

\begin{lstlisting}
  RecordEntry Label^t pos t => res
  ----------------------------------
  update t pos => res
}
\end{lstlisting}

Next we need the controls for each bike.
As the bike is always moving forward we only need to steer left or right.

The two buttons used for this are placed in a record.

\begin{lstlisting}
TypeFunc "Keys" => Key => Key => Record
RecordEntry "Left" left Empty
RecordEntry "Right" right Left
------------------------------
Keys left right => res
\end{lstlisting}

Then we come to the bike itself.
It will use the \verb\Keys\, \verb\TrailRecord\ and \verb\PositionRecord\.
And with those it also uses records to store the colour, speed and a boolean to check if it has not crashed.

The bike will also be given a unique identifier, its name.

In the definition we will see how the bike is built with multiple records.

\begin{lstlisting}
TypeFunc "Bike" => String => Int => Int => Keys => Vector2 => TrailRecordEntry => Record => Record
RecordEntry "Colour" color Empty
RecordEntry "IsAlive" True^builtin Colour
RecordEntry "Controls" keys IsAlive
RecordEntry "Speed" speed Controls
PositionRecord "Position" position Speed
RecordEntry "Trail" trail Position => field
RecordEntry label field rest => e
---------------------------------------------------------------------
Bike label color keys speed position trail rest => UpdatableRecord e{
\end{lstlisting}

The record entries are ordered according to how often they are needed, this speeds up compile time.

\verb\color\ is an \emph{ARGB} value and \verb\speed\ is the amount of pixels per second.

The update function takes a bike and the delta time as argument.
It first checks if the bike is still active.

\begin{lstlisting}
  (if Field^IsAlive then
\end{lstlisting}

Next the trail, position are put in variables.

\begin{lstlisting}
    get^b "Trail" Field^b => trail
    get^trail "Position" Rest^trail => position
    get^position "Speed" Rest^position => speed
\end{lstlisting}

Then the trail and position get updated.

\begin{lstlisting}
    update^TrailRecord Field^trail Field^position => newTrailRecord
    update^PositionRecord position (dt,Field^speed) -> newPos
\end{lstlisting}

A new trail record is created and lastly a new bike record is created and returned.

\begin{lstlisting}
    set^b label^trail newTrailRecord newPos => newTrail
    set^b Label^b newTrail Rest^b
\end{lstlisting}

The bike is returned if it is dead.

\begin{lstlisting}
  else
    b) -> res
  -------------------------------------
  update b dt -> res
}
\end{lstlisting}

We now have a complete bike that can be updated.


% \subsubsection{Evolution}
% \textbf{ADD STUFF}
% During development of tron the \verb\apply\ function was used to call all the update functions.
% After its removal the update functions are directly called.


\subsection{Playfield}
The playfield contains a record with the bikes, the filed size and a winner.
When the winner is set it will be returned to the main function and the game will end.

First we import the XNA framework, System, record, prelude and bike.
\begin{lstlisting}
import Framework^Xna^Microsoft
import record
import prelude
import bike
\end{lstlisting}

For the bikes a separate record will be created.
This enables a dynamic number of bikes without interfering with the playfield.

\begin{lstlisting}
TypeFunc "BikeRecord" => String => Record => Record => Record
BikeRecordEntry label bikes rest => RecordEntry label bikes rest{
\end{lstlisting}

When defining the update function we need a few extra functions to make it all work.
These functions are necessary because they do recursive operations.
This would be impossible without a separate function.

The first function is \verb\checkTrails\.
It goes through the trail of a bike and check for any collisions.
It takes a position and trail and returns a boolean.
If any collisions are detected the boolean returns false, otherwise it returns true.

First we check for any collisions with the current point in the trail.

\begin{lstlisting}
  Func "checkTrails" -> Vector2 -> Record -> Boolean
  (if ((X^Field^pos == X^Field^trail) ||
       (Y^Field^position == Y^fieldsize)) then
    False^builtin
  else
    True^builtin) -> newIsAlive
\end{lstlisting}

If there are no collisions and there are more points in the trail, \verb\checkTrails\ is recursively called.

\begin{lstlisting}
  (if (newIsALive && (Rest^trail != Unit)) then
    checkTrails pos Rest^trail
  else
    False^builtin) -> res
  -----------------------------
  checkTrails pos trail -> res
\end{lstlisting}

Next we have \verb\checkCollisions\, which calls \verb\checkTrials\ for the trail of every bike.
It takes a bike and a record of bikes.
The position of the current bike and the trail of the first bike in the record, will be passed to \verb\checkTrails\.

\begin{lstlisting}
  Func "checkCollisions" -> Record -> Record -> Record
  get^bs "Position" Field^bs -> pos
  get^Rest "Trail" Field^rst -> trail
  checktTrails pos trail -> newIsAlive
  (if newIsAlive && (Rest^trail != Unit))hen
    checkCollisions bs Rest^rst
  else
    set^bs "IsAlive" False^builtin) -> newBs
  ------------------------------------------
  checkCollisions bs rst -> newBs
\end{lstlisting}

Then we have the function which will update the \verb\IsAlive\ status of the current bike.
First it checks if the position of the bike is within the playfield.

\begin{lstlisting}
  Func "updateIsAlive" -> Record -> Vector2 -> Record
  get^bs "Position" Field^bs -> position
  (if ((X^Field^position <= 0.0) ||
       (Y^Field^position <= 0.0) ||
       (X^Field^position >= X^fieldsize) ||
       (Y^Field^position >= Y^fieldsize))
          (set^bs "IsAlive" False^builtin Rest^bs)
        else
\end{lstlisting}

If the bike is within the playing field, any collisions with its own trail and the other bikes is tested.

\begin{lstlisting}
          ((checkTrails bs (get^bs "Trail" Field^bs)
           (checkCollisions bs Rest^bs))) -> newBs
  updateIsAlive Rest^newBs fieldsize -> newBikeRecords
  ----------------------------------------------------
  updateIsAlive bs fieldsize -> updatedBikes
\end{lstlisting}

We then come to \verb\updateBikes\.
This goes through all the bikes and calls their update function.
It then returns the updated bikes.

First we have to check if the bike record is \verb\Empty\.

\begin{lstlisting}
  Func "updateBikes" -> Record -> Float -> Record
  (if (b == Empty) then
    Empty
  else
    (update^b b dt -> newBike
     updateBikes Rest^b dt -> newRest
     RecordEntry Label^newBike Field^newBike newRest)) -> res
  ------------------------------------------------------
  updateBikes b dt -> res
\end{lstlisting}

In the update function we want to get the actual bikes.

\begin{lstlisting}
  Field^bs -> bikeRecords
\end{lstlisting}

We also need \verb\fieldsize\ and \verb\dt\.
These are given to the function as a tuple, because \verb\update\ can only take two parameters and we need three.

The \verb\update\ function first calls \verb\updateBikes\.
And when the bikes are updated they are passed to \verb\updateIsAlive\.
The new bike record is then returned.

\begin{lstlisting}
  dtfs -> (dt,fs)
  updateBikes bikeRecords dt -> newBikeRecords
  updateIsAlive newBikeRecords fs -> aliveBikeRecords
  set^bs label^bs aliveBikeRecords bs -> res
  -----------------------------------------
  update bs dtfs -> res
}
\end{lstlisting}

When defining the actual playfield we create a \verb\BikeRecordEntry\ and a record containing the winner.
Both these records are passed to \verb\Rest\ of the playfield.
The size of the playfield is put in the \verb\Field\ of the playfield record.
The playfield record is also made updatable.

\begin{lstlisting}
TypeFunc "Playfield" => String => Record => Vector2 => Record
RecordEntry "Winner" unit Empty
BikeRecordEntry "Bikes" bikes Winner -> rest
--------------------------------------------
Playfield label bikes size => updatableRecord (RecordEntry label size rest) {
\end{lstlisting}

\verb\Winner\ is emty until there is only one player left in the game.

Next we create the \verb\checkDeaths\ function.
This function is needed as an intermediate function to \verb\update\.
\verb\checkDeaths\ calls itself recursively to check the state of all the bikes in the game.

It then keeps a counter of the amount of bikes that are still alive.
When the counter is higher than one \verb\unit\ is returned.

\textbf{FIX THE CODE AND FINISH TEXT}
\begin{lstlisting}
  Func "checkDeaths" -> bike -> amountAlive -> 'a
    (if (amountAlive > 1) then
      unit
    else
      ((if ((get^bike "IsAlive" Field^bike) == True^builtin) then
        amountAlive + 1
      else
        amountAlive) -> newAmountAlive
  (if (amountAlive == 1) then
    bike
  else
      checkDeaths Rest^bike newAmountAlive))) -> res
  ------------------------------------------------
  checkDeaths bike amountAlive -> res
\end{lstlisting}

Now that we can check that there is a winner we can define \verb\update\ of \verb\Playfield\.
First \verb\update\ updates \verb\Bikes\.

\begin{lstlisting}
  get^p "Bikes" Rest^p -> bikes
  update^bikes bikes (dt,Field^p) -> newBikes
\end{lstlisting}

Then \verb\checkDeaths\ is called and the winner is set.
The new playfield is then updated with the updated bikes and this playfield is returned as result.

\begin{lstlisting}
  checkDeaths Field^newBikes 0 -> winner
  set^p "Winner" winner Rest^p -> updatedPlayfield
  set^updatedPlayfield "Bikes" newBikes Rest^updatedPlayfield -> newPlayfield
  --------------------------------------------
  update p dt -> newPlayField
}
\end{lstlisting}

This makes the playfield complete.
All that is left to do is create the main function which starts the game.

\subsection{Main}
\textbf{DO THIS!!!!!!!!!!}
The main combines the bikes and the playfield to start the actual game.

First \verb\bike\, \verb\playfield\ and the XNA framework is imported.
\begin{lstlisting}
import playfield
import bike
import xna stuff
\end{lstlisting}

Then we create a \verb\play\ function which runs the actual game until there is a winner.
When there is no winner it increments the delta time and updates the playfield.

\begin{lstlisting}
Func "play" -> Playfield -> Record
get^p "Winner" Rest^p -> winner
(if (winner == unit)
  (update^p dt -> p'
   play p' (dt+1))
else
  Winner) -> realWinner
-------------------------------
play p dt -> realWinner
\end{lstlisting}

When a winner has emerged, this winning bike is returned.

Then we declare and define \verb\start\ which creates all the bikes and the playfield and calls \verb\play\ to start the game.

\textbf{FIX CODE!!!!!!!!!!!!!!!!}
\begin{lstlisting}
Func "start" -> 'a
Bike "playerOne" -16776961 5 -> bikeOne
Bike "playerTwo" -16711681 5 keys Vector2(0.0,0.0) trail unit -> bikeTwo
RecordEntry "gamebikes" BikeOne BikeTwo -> bikeRecord
Vector2(500,500) -> fieldsize
playfield "gamefield" bikeRecord fieldsize  -> gamefield
play gamefield 0 -> res
---------------------
start -> res
\end{lstlisting}

When a winner is present the game stops.

\subsection{Discoveries}
During the programming of the game several things became clear about MC.

\subsubsection{Record}
The \verb\set\ function of \verb\Record\ was not returning the original record, just the updatable record.\footnote{See section~\ref{sec:recordevolutionsetget}}

\subsubsection{Good language}
The points described in chapter~\cite{chap:criteria} also had there conclusions found during the creation of the test programs.

% It was discovered that MC has a steep learning curve because of the higher order types and the monadic part of the standard library.

The read- and writability of the MC syntax takes some getting used to, as with most languages, but overall it is very compact and readable.

The simplicity of MC is a mixed bag.
While the syntax is simple, the concepts used by the language can become anything but simple.
These concepts will have the user do some studying to understand these concepts.

The goal and idea behind of MC is quite clear.\footnote{As described in section~\ref{sec:}.}
And because MC has a clear idea of what it wants to do, the user can easily use the language as it was intended.

By being a declarative language MC has de advantage of making itself predictable.
The declarations always state what a function does and the definition then tells how this is achieved.
Even when using monads MC keeps its predictability, because of how the monads work.\footnote{As is shown in section~\ref{sec:}.}

The higher order types handle abstractions very well, as is evident by the short implementation of monad transformers.\footnote{As is shown in section~\cite{sec:}.}
The high abstractions MC is capable of increase the expressiveness of MC.

Because all these high abstractions are computed at compile time the efficiency does not suffer.
The efficiency can even be increased when moving computations normally done at runtime to compile time.

The custom libraries have made quite a leap during its development and the standard library has a stable base.
This stable base increases easy expansion in the future, which will help MC to thrive.

The hackability of MC has two sides.
On the one hand it forces the user to be very secure by using declarations and definitions.
On the other hand the user can implement the definitions in numerous ways.
Especially the latter enables the user to bend MC to be the language they need.

The higher order types and the high abstractions they enable allow MC to be very succinct.
The user does need to have both declarations and definitions which makes MC more work, but they do not outweigh the power given to the user with the higher order types.

The part MC might be troublesome for user is redesigning existing code.
Especially when editing what functions do, the user needs to change both the declaration and definition of a function.
This double editing is not something programmers are fond of and might ruin the flow of programming is an evolutionary way.
However when the user thinks about the code before it is written, MC works like a charm.

By implementing a direct link to the .NET libraries MC has an enormous amount of functionality added at very little cost.
This does mean MC has to compete with C\# and F\#, both of which are strong languages using the .NET library.
One advantage MC has over both C\# and F\# is the higher order types and the high abstractions it can implement.
Adding to that the ability to compute anything you want at compile time instead of runtime, brings in major advantages over C\# and F\#.


