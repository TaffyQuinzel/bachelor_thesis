\chapter{Test programs}
There have been two test programs written:

\begin{enumerate}
   \item Bouncing ball
   \item Tron
\end{enumerate}

\paragraph{Bouncing ball} is used as a preliminary test of the updatable records.
It will show how the updatable records are used in practice.

\paragraph{Tron} is a more advanced program which will utilise user input.
The game tron~\cite{} has been chosen because of the minimalistic graphics.
This allows us to focus on the internal workings of the game.


\section{Bouncing ball}
This program bounces a ball up and down.
It uses the entity from the Casanova library and the Microsoft XNA library for the \verb\Vector2\.

\begin{lstlisting}
import Microsoft^Xna^Framework
import record

RecordEntry "Velocity" Vector2(0.0f, 98.1f) Empty
RecordEntry "Position" Vector2(100.0f, 0.0f) Velocity

updatableRecord (RecordEntry "Ball" unit Position) => Ball

Ball{
  get^e "Position" Rest^e => position
  get^e "Velocity" Rest^e => velocity
  (if ((Field^position).Y <= 500.0f) then
    ((set^e "Position"
           (Field^position +^Vector2 (Field^velocity *^Vector2 dt))
           Rest^position)
     (set^e "Velocity"
            (Field^velocity +^Vector2 (Vector2(0.0f, 98.1f) *^Vector2 dt))
            Rest^velocity))
  else
    (set^e "Position"
           Vector2((Field^position).X, 500.0f)
           -^Vector2(Field^velocity))) -> res
  --------------------------------------------
  update e dt -> res
}
\end{lstlisting}

\section{Tron}
\subsection{Bike}
\subsection{Playfield}
\subsection{Powerups}
