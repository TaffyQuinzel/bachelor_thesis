\chapter{Introduction}
\section{Company}
The graduation assignment is carried out at Kenniscentrum Creating 010.
The company is located in Rotterdam.
\textit{Kenniscentrum Creating 010 is a transdisciplinary design-inclusive Research Center enabling citizens, students and creative industry making the future of Rotterdam}~\cite{creating2016home}.

The assignment is carried out within a research group, who is building a new programming language.
The new programming language is called \emph{Casanova}.


\section{The assignment}
We will now describe the assignment.
What the motive is and how it will be executed.

\subsection{Motive}
Video game development is difficult~\cite{blow2004game}.
Video games are getting bigger and, with it, more complex.
However the time between releases of the video games does not lengthen~\cite{blow2004game}.

One of the major challenges video game developers face is a short time to market.
When the developers are able to produce games faster, they have an advantage over their competitors.

This is where \emph{Casanova} comes in.
\emph{Casanova} is a programming language to make the development of games easier~\cite{maggiore2011designing}.

Casanova uses higher order types to make the programming of complex constructs easier.
These constructs are often used in game development.
A few of these constructs will be explained in detail from chapter~\ref{chap:standardlibrary} onward.

Because of the higher order types the compiler for Casanova became complex.
The compiler for Casanova worked, but was still in development.
So when a bug was found in the compiler, fixing it became more and more time consuming.
This frustrated the developers so much a new language was created~\cite{giuseppe2015mc}.

Meta Casanova (from now on called \emph{MC}) is this new language.
MC serves as the language in which Casanova will be written as a library

At the beginning of the assignment MC was still in development.
The syntax was nearly complete, but untested, and the standard library of MC was just beginning to take shape.
The assignment was created to debug and expand MC and further refine the standard library of MC.


\subsection{Research question}
As stated above the main goal of the assignment is to debug and expand MC and refine the standard library of MC.
This translates to the following research question:

\emph{How can the programming language Meta Casanova be improved for the user within the timeframe of the internship?}

Here the user is a programmer using the language to create applications.

This research question is quite broad and several sub-questions have been created to define a clearer scope for the assignment.

\begin{enumerate}[noitemsep]
   \item What is a good programming language to the user?
   \item What is MC and how does it work?
   \item How can the current syntax be improved to serve the user?
   \item How can the standard library be improved to serve the user?
\end{enumerate}

After we have a clear image of what a good programming language is, we can look at MC.
MC can then be checked and improved where necessary.

To further test MC from a users perspective a few test programs will be written in the language.

The definitions and scope of the improvements are defined by the preliminary research.


\subsection{Correlation with other projects}
During the internship the research group keeps developing Casanova.
The research group consists of
  Francesco di Giacomo\footnote{\label{venice}Universita' Ca' Foscari, Venezia},
  Mohamed Abbadi\footnoteref{venice},
  Agostino Cortesi\footnoteref{venice},
  Giuseppe Maggiore\footnote{Hogeschool Rotterdam} and
  Pieter Spronck\footnote{Tilburg University}.

Within the research group the research team developing MC consists of Douwe van Gijn, responsible for developing the back-end of the bootstrap MC compiler, Jarno Holstein, responsible for developing the front-end of the bootstrap MC compiler, and Louis van der Burg, responsible for developing the MC language.
The research team are all students and are supervised by Giuseppe Maggiore.

The student will work closely with the research team and cooperate with the research group.
