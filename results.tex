\chapter{Conclusion}
To summerize the main research question was:
\emph{How can the programming language MetaCasanova be improved from its current state within the timeframe of the internship?}

The answer to this research question is:
\emph{MetaCasanova can be improved by bringing in someone with a fresh view on the language.
This person will use be taught the basics of the language and use it to build test programs.
He or she will also extend the standard library with what is needed for these programs to work.}


\chapter{Recommendations}
These recommendations are for people who continue with the project.

\section{Improving a programming language}
When first learning about a language, keep on asking until you have specifics.
Most of the developers have general ideas of a language.
These ideas have very little specifics, but will come to the surface when really digging.

By having these ideas before hand, bugs will become apparent when first looking at the actual language.


\section{Further development}
When developing MC further I would recommend to expand the standard library even more.
Especially when the focus of the language is put more on monads.

\subsection{Monads}
First we would need a proper IO monad for interacting with the unpure functions and get real output.

Coroutines would be next.
Both as a separate entity and as a monad.
They would give the ability of multi threading directly in the language.
This would improve performance.

\subsection{The compiler}
I would also recommend to keep syntax changes to a minimum.
This will give the compiler developers a fighting chance.
During the development the compiler developers have thrown away alot of work, because the syntax became too different.

When creating a new itteration of the language, first get a clear picture of the language before starting work on the compiler.
This creates less work for the compiler developers as they do not have to change the compiler mid build.


\chapter{Evaluation}
Here I will provide proof of having mastered the dublin descriptors and the extra competentions as set in the graduation module guidelines~\cite{}.

\section{Dublin descriptors}
\subsection{Knowledge and understanding}
During the project I have used type-theory as described in~\cite{} to check the programs I was writing.
This can be explicitly seen in section~\ref{sec:standardtryablemonad} and appendix~\ref{app:typeproofs}.

Further more I have learned the monadic theory to be able to implement them in the standard library.

Both of these build on the knowledge gained during my four years of studying at the University of Rotterdam.
Here I have learned the basis which was needed during this project.

The programming skills learned were needed to expand these skills with the type theory.
The mathematical skills gained were needed when learning about the monadic constructs.


\subsection{Applying knowledge and understanding}
The knowledge gained in the research phase of this project combined with the knowledge gained in my study years, have been put into practical use.
When building upon MC I had to use all the type theory knowledge gained to be certain no function failed.

When implementing the monadic part of the standard library, I have combined the programming- with the mathematical skills to translate mathematical ideas into a practical code base.


\subsection{Making judgements}
At the start of this project I have done research into a new language, MC, type theory and monadic mathematical concepts.
These were all needed to build upon the existing codebase of MC.

\textbf{DO CITATIONS}
I have made descisions while keeping both the practical aspects and goals in mind, as seen in sections~\ref{},\ref{} and \ref{}.


\subsection{Communication}
During this project I have worked in a research team where I have had to explain myself to fellow team members and supervisors.

This was done in the form of presentations, discussions and meetings.
I also had to understand their ideas about concepts.


\subsection{Lifelong learning skills}


\section{Extra competentions}
