\chapter{Assignment}
\section{Company}
The graduation assignment is be carried out at Kenniscentrum Creating 010.
The company is located in Rotterdam.
\textit{Kenniscentrum Creating 010 is a transdisciplinary design-inclusive Research Center enabling citizens, students and creative industry making the future of Rotterdam}\cite{creating2016home}.

The assignment is carried out within a research group, who is building a new programming language.
The new programming language is called \emph{Casanova}.

% \section{Supervision \& work methods}

% \subsection{Working environment and tasks}\label{subsec:workenvmandate}
% During the internship the student will work as a part of the existing research team, whom do research in the Casanova and MC languages.
% The student will be doing research and programming and will be supported and helped by the entire research team.
% Every two weeks the student will present his progress and receive feedback.

% \subsection{Research methods}
% During the internship there will be made use of different research methods.
% This stems from the wide scope of the project and the fact that there are multiple research questions which require different approaches.

% At the start there will be a preliminary research concerning MC.
% This will consist of describing MC and the ways it works.

% After this there will be a basic definition of the test program which will we written to further test MC.

% When this is done there will be a evaluation of the test program and MC.

% \subsection{Gathering information}
% The information needed will be gathered from the research team involved, via personal interviews, and via published papers concerning the material.

% \subsection{Validation of results}
% The validity of the results will be secured by regularly discussing with the client and the supervisor.
% Further more there will be programs will be double-checked by the supervisor and the other members of the research team.

% \subsection{Validity and trustworthiness of sources}
% The sources used will be published papers only and relative to the subject matter.
% Other sources, such as personal communications will be with people who are approved by the supervisor.

% \subsection{Project methods}
% During the internship the LEAN-software development method will be used\cite{ries2011lean}.
% Using this method gives the ability to quickly iterate through versions and gather knowledge more easily through these iterations.

% \subsection{Risk analyses}
% \begin{center}
   % \begin{tabular}
      % {| p{0.2\textwidth} | p{0.25\textwidth} | l | p{0.3\textwidth} |}
      % \hline
      % \textbf{Risk} & \textbf{Effect} & \textbf{Possibility} & \textbf{Counter measure}
      % \\ \hline
      % Quick development of the language. & Because the language becomes bigger as time progresses, the assignment could become too large to do in the set time. & 70\% & The analyses of MC goes on for a set period, which prevents the analyses going an for too long.
      % \\ \hline
      % The compiler is not finished at the time the programs need to be written. & The programs can not be tested in practice. & 80\% & The programs will be manually checked for errors by the student and the research team.
      % \\ \hline
      % The company goes bankrupt during the internship & The assignment will not be finished and the student will not be able to graduate. & 1\% & There is nothing the student can do to prevent this.
      % \\ \hline
   % \end{tabular}
% \end{center}

% \subsection{Quality expectations}
% MC makes use of type theory and the programs written in it will therefor need to be in line with the type theory, as described in \cite{pierce2002types}.

% Because the version of MC used in this internship has no compiler yet, the student will need to manually make sure that the program has no bugs and/or errors in them.
% This will happen with the help of the compiler developers.

\section{Motive}

MetaCasanova (from now on called \emph{MC}) comes forth from the language Casanova, hence the name.

Casanova is a language made for building games.
It uses higher order types to make the programming of complex constructs easier.
These constucts are often used in game development.
A few of these constructs will be explained in detail from chapter~\ref{chap:standardlibrary} onward.

Because of the higher order types the compiler for Casanova became complex.
The compiler for Casanova worked, but was still in development.
So when a bug was found in the compiler, fixing it was becoming more and more time consuming.
This frustrated the developers so much a new language was created.

MC was this new language.
MC serves as the language in which the compiler for Casanova will be written.

At the beginning of the assignment MC was still in development.
The syntax was nearly complete, but untested, and the standard library of MC was just beginning to take shape.
The assignment was created to debug and expand the language.

I took this assignment and have written my bachelor thesis on it.

\section{Goal}
As stated the main goal of the assignment is to debug and expand the language.
Which translates to the following research question:
\emph{How can the programming language MetaCasanova be improved from its current state within the timeframe of the internship?}
This goal is quite broad and I have created a few subgoals to define a scope for the assignment.

\begin{enumerate}[noitemsep]
   \item Double checking the existing design of the language.
   \item Participating in the design of the language.
   \item Extending the standard library.
   \item Creating test programs.
\end{enumerate}

I will now explain the subgoals in greater detail.

\subsection{Double checking the existing design of the language}
At the start of the assignment MC will have to be checked for any existing faults.
This way we can ensure that the language has a safe base point from which it can be developed further.

The checking of MC will consist of the existing syntax and the ideas for which the syntax was created.
This will be done by exploring the current syntax and understanding the reasons why it is this way.

\subsection{Participating in the design of the language}
As MC is still in development and my assignment consists of debugging and expanding, I will propose design changes and new designs.
These designs will be discussed and reviewed with the rest of the team.

This will be done in an exploratory way as the language grows and changes.

\subsection{Extending the standard library}
The standard library of every language is an extention of the functionality of the language.
It contains often used constructs and functions which makes the language more productive for the programmer.

MC has but a small standard library and this needs to be expanded in order to be of any practical use.
This also could help make MC a success~\cite{khedker1997makes}.

The standard library of MC will be completely written in MC itself.
This will show how powerfull the simple and effective syntax of MC is.

Extending the library also gives me the position of a user of MC.
For now the only people that have used MC are the developers themselves.
Now that I will join the research team they will get a fresh and new look at the language through me.

\subsection{Creating test programs}
Writing the standard library is one aspect of a language.
Another are actual applications.

The standard library can be seen as an application, but it is different in the sense that they create standard functions to be used by programmers.
Applications are not part of the code base and also serve users outside of the programming community.

Because the only users so far have been the developers, there have only been small test programs written.
They often test a specific function of the compiler.
The test applications will test the entire language.

Because of the timelimit I will not be able to create a huge application, but I will be able to create bigger ones than those currently used for testing.

I have chosen to create game or game-like test prog




\section{Correlation with other projects}
During the internship the research team keeps developing MC.
There are also two other students doing an internship on MC.
They are both working on the compiler, one is developing the back-end of the compiler and the other the front-end.
The student will cooperate with the research team and both these students.


% \section{Execution}
% The assignment will consist of three stages:

% \begin{enumerate}
   % \item Research on MC
   % \item Expanding the standard library
   % \item Creating test programs
% \end{enumerate}

% \subsection{Research on MC}
% This will describe MC and the way it works.

   % \subsection{Expanding the standard library}
   % \subsection{Creating test programs}
% This will consist of describing MC and the ways it works.

% After this there will be a basic definition of the test program which will we written to further test MC.

% When this is done there will be a evaluation of the test program and MC.
