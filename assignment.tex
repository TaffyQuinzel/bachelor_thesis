\chapter{Assignment}
\section{Company}
The graduation assignment will be carried out at Kenniscentrum Creating 010.
The company is located in Rotterdam.
\textit{Kenniscentrum Creating 010 is a transdisciplinary design-inclusive Research Center enabling citizens, students and creative industry making the future of Rotterdam}\cite{creating2016home}.

With Kenniscentum Creating 010 the assignment will be carried out within a research group who is building a new programming language.

EXPAND THIS ALOT: SEE  3RD YEAR INTERNSHIP BEDRIJFSORIENTATIE.

\section{Motive}
MetaCasanova (from now on called \emph{MC}) comes forth from the language Casanova, hence the name.

Casanova is a language made for building games.
It uses high abstractions to make the programming of games easier.
Because of the complexity of these abstractions the compiler became complex aswell.

The compiler worked, but was still in development.
So when a bug was found in the compiler, fixing it was becoming more and more time consuming.

To address this problem MC was created.
MC serves as the language in which the compiler for Casanova will be written.

As of now the MC is still in development and I will be contributing to it's development with this bachelor thesis.

\section{Goal}
The main goal is to debug and expand the language.

Because this is a very general goal it will be split into more concrete goals:
\begin{enumerate}
\item Double checking the existing design of the language.
\item Participating in the decision making of new design choices.
\item Extending the standard library.
\item Creating test programs.
\end{enumerate}

\subsection{Double checking the existing design of the language}
At the start of the assignment MC will have to be checked for any existing faults.
This way we can insure that the language has a safe base point from which it can be developed further.

\subsection{Participating in the decision making of new design choices}
MC is still in development, thus the language will change during the course of the assignment.
These choices will have an effect on the end result of what the language will be.
Because the assignment is to debug and expand the language, I will be actively involved with new design choices.

This way there is an extra check if the choices are within the general idea of the language and if they are correctly implemented.

\subsection{Extending the standard library}
The standard library of every language is a basis for the programmers using the language.
MC has but a small standard library and this needs to be much bigger to be of any use and to make MC a success~\cite{}whatmakesprogramminglanguagesgreat.
% Because of the powerfull abilities of MC, see section~\ref{}, the standard library will be written in MC itself.

The standard library of MC will be completely written in MC itself.
This will show how powerfull the simple and effective syntax of MC is.

\subsection{Creating test programs}
As of yet there are no users of MC language.
This gives the developers very little feedback on any design ideas and choices that are and have been made.

Creating test programs will therefor be very usefull for the developers, as they will have a huser who can give feedback.

\subsubsection{The test programs}
The

The test programs themselves should be a game or game-like.
This is because games use alot of complex structures.


% \begin{enumerate}
   % \item For a minimum passing grade of 6: \newline
      % A small interactive and graphical application.
   % \item For an 8: \newline
      % Within the application for a 6, add traversable and changeable data structures.
   % \item And for a 10: \newline
      % Within the application for an 8, add coroutines.
% \end{enumerate}

% After which the result is reflected upon.

\section{Scope}
The assignment

\section{How is the supervision organized within the company}

\subsection{Working environment and tasks}\label{subsec:workenvmandate}
During the internship the student will work as a part of the existing research team, whom do research in the Casanova and MC languages.
The student will be doing research and programming and will be supported and helped by the entire research team.
Every two weeks the student will present his progress and receive feedback.

\subsection{Research methods}
During the internship there will be made use of different research methods.
This stems from the wide scope of the project and the fact that there are multiple research questions which require different approaches.

At the start there will be a preliminary research concerning MC.
This will consist of describing and comparing MC and the ways it works.

After this there will be a basic definition of the test program which will we written to further test MC.

When this is done there will be a evaluation of the test program and MC.

\subsection{Gathering information}
The information needed will be gathered from the research team involved, via personal interviews, and via published papers concerning the material.

\subsection{Validation of results}
The validity of the results will be secured by regularly discussing with the client and the supervisor.
Further more there will be programs will be double-checked by the supervisor and the other members of the research team.

\subsection{Validity and trustworthiness of sources}
The sources used will be published papers only and relative to the subject matter.
Other sources, such as personal communications will be with people who are approved by the supervisor.

\subsection{Project methods}
During the internship the LEAN-software development method will be used\cite{ries2011lean}.
Using this method gives the ability to quickly iterate through versions and gather knowledge more easily through these iterations.

\subsection{Risk analyses}
\begin{center}
   \begin{tabular}
      {| p{0.2\textwidth} | p{0.25\textwidth} | l | p{0.3\textwidth} |}
      \hline
      \textbf{Risk} & \textbf{Effect} & \textbf{Possibility} & \textbf{Counter measure}
      \\ \hline
      Quick development of the language. & Because the language becomes bigger as time progresses, the assignment could become too large to do in the set time. & 70\% & The analyses of MC goes on for a set period, which prevents the analyses going an for too long.
      \\ \hline
      The compiler is not finished at the time the programs need to be written. & The programs can not be tested in practice. & 80\% & The programs will be manually checked for errors by the student and the research team.
      \\ \hline
      The student dies & The assignment will not be finished and the student will not be able to graduate. & 0.0001\% & The student tries to avoid deadly situations.
      \\ \hline
      The company goes bankrupt during the internship & The assignment will not be finished and the student will not be able to graduate. & 1\% & There is nothing the student can do to prevent this.
      \\ \hline
   \end{tabular}
\end{center}

\subsection{Quality expectations}
MC makes use of type theory and the programs written in it will therefor need to be in line with the type theory, as described in \cite{pierce2002types}.
Because the version of MC used in this internship has no compiler yet, the student will need to manually make sure that the program has no bugs and/or errors in them.

\section{Correlation with other projects}
During the internship the research team keeps developing MC.
There are also a two other student doing an internship on MC.
One is building the compiler for MC and the other is optimizing the code generator.
The student will have to cooperate with the research team and with both these students.

\section{Working environment}
\subsection{Project methods}
During the internship the LEAN-software development method will be used\cite{ries2011lean}.
Using this method gives the ability to quickly iterate through versions and gather knowledge more easily through these iterations.
