\section{What is MC}
MC stands for MetaCasanova and is MC is a declaritive functional language.
\section{Why MC}


\section{Goal}
The goal of MC is to use higher abstractions with type safety.

\section{Basics}
We will now go through the basics of MC.

\subsection{Func}
First we need to declare a function.
   \begin{lstlisting}
Func "foo" -> Bool -> Int -> Int -> Int
   \end{lstlisting}
   For this we use the keyword \emph{Func}.
   Here we see that the function \emph{foo} is declared by a String, which is the name of the function, a Boolean and three Integers.
   The last parameter is the return type of the function and all the parameters between the name and the return type, are the arguments of the function.

   When no parameters are given the function becomes a variable, as seen in figure \ref{}.
   \begin{lstlisting}
Func "foo" -> Int
foo -> 5
   \end{lstlisting}

   \begin{lstlisting}
add b c -> res
---------------
foo True b c -> res

mul b c -> res
---------------
foo False b c -> res
   \end{lstlisting}

   Here we see the basic syntax for declaring and defining a function.

\subsection{Data}
\subsection{TypeFunc}
\subsection{TypeAlias}
\subsection{Module}
