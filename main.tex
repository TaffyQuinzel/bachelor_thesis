% \documentclass[a4paper,twoside,openright]{report}
\documentclass[landscape,a4paper]{report}
\usepackage[landscape,driver=xetex,a4paper,margin=25mm]{geometry} % propper margins on a4 paper

\usepackage[hidelinks]{hyperref} % make contents and references clickable in pdf
\usepackage[english]{babel}
\usepackage{fancyhdr}
\usepackage{emptypage}
\usepackage{amsmath}

\usepackage{pstricks-add}
%% \usepackage[official]{eurosym}
\usepackage[nottoc,numbib]{tocbibind}

\usepackage{appendix}

\usepackage{graphicx}
\usepackage{caption}
\graphicspath{{./plaatjes/}}
\DeclareGraphicsExtensions{.png}

% used for typeproofs
\usepackage{amsfonts}
\usepackage{amssymb}
\usepackage{bussproofs}

\usepackage[utf8]{inputenc}

% font stuff
\usepackage{fontspec}
\setmainfont[
   BoldFont={Source Sans Pro Black},
   AutoFakeSlant=0.3
]{Source Serif Pro}
\setmonofont{Source Code Pro}

\usepackage{url}
\usepackage{multicol}
\usepackage[section]{placeins}

\usepackage{enumerate}
\usepackage{enumitem}

\usepackage{listings}
\usepackage{amsmath}
\usepackage{xcolor}
\lstset{%
   basicstyle=\footnotesize\ttfamily,
   frame=single,
   breaklines=true,
   postbreak=\raisebox{0ex}[0ex][0ex]{\ensuremath{\color{red}\hookrightarrow\space}},
   keepspaces=true
}

% draft marker
\usepackage{draftwatermark}

\newcommand*\writer{Louis van der Burg}
% \renewcommand{\chaptermark}[1]{%
% \markboth{#1}{}}

%header & footer
\pagestyle{fancy}
\fancyhf{}
\fancyhead[LE,RO]{\nouppercase\leftmark}
\fancyhead[RE,LO]{\partname\ \thepart}
\fancyfoot[CE,CO]{\writer}
\fancyfoot[LE,RO]{\thepage}

%title page stuff
\fancypagestyle{plain}{%
   \fancyhf{}\fancyfoot[LE,RO]{\thepage}%
   \fancyfoot[CE,CO]{\writer}
\renewcommand{\headrulewidth}{0pt}}

\newcommand\BrText[2]{%
   \par\smallskip
   \noindent\makebox[\textwidth][r]{$\text{#1}\left\{
      \begin{minipage}{\textwidth}
         \lstset{#2}
      \end{minipage}
      \right.\nulldelimiterspace=0pt$}\par\smallskip
   }

%% \newenvironment{poliabstract}[1]
%%                {\renewcommand{\abstractname}{#1}\begin{abstract}}
%%                {\end{abstract}}

% start actual doc
\begin{document}
\title{Testing the Meta-Compiler language}
\author{\writer}
% \date{25 January 2016}

% \begin{titlepage}
   % \include{voorpaginamandate}
% \end{titlepage}

%\maketitle
% state the problem
% say why it's an interesting problem
% say what your solution is
% say what follows from your solution

% \begin{abstract}
% MC has not been tested in the field yet and we want to know how it fares.
% We will put it to the test and see that it still needs some development, but has promise.
% \end{abstract}
% \newpage

% \pagenumbering{gobble}
% \setcounter{tocdepth}{2}
% \tableofcontents
% \cleardoublepage
% \pagenumbering{arabic}
% \addtocounter{page}{4}


\part{Introduction}

\chapter{People involved}

\chapter{interlude}
MC stands for MetaCasanova. The reason behind the name is explained in chapter \refchapter{MetaCasanova}.

We will first talk about what MC is and explain the basics of the language.
Then we will discuss the Standard Library of MC, which is written completely in MC.
After which we will show some practical examples.

\chapter{The company}


\part{Preliminary research}
\chapter{MC}


\part{MC expanded}

\chapter{Standard library}
\section{BasicMonads}
\subsection{Background}
\subsubsection{Monads}
\subsubsection{Monad transformers}
\subsection{Implemented monads}
\subsubsection{Id}
\subsubsection{List}
\subsubsection{Either}
\subsubsection{Option}
\subsubsection{Result}
\subsubsection{State}
\subsubsection{IO}
\section{Match}
\section{Monad}
\section{Prelude}
\section{Number}
\section{Record}
\section{TryableMonad}

\chapter{Casanova}
\section{Entity}
\section{Coroutines}

\chapter{Test programs}
\section{Bouncing ball}
\section{Tron}
\subsection{Bike}
\subsection{Playfield}
\subsection{Powerups}


\part{Conclusion}

\chapter{Conclusion}
\chapter{Reflection}



\part{appendixes}

% \bibliographystyle{ieeetr}
% \bibliography{biblio}

% \end{multicols}{3}
\end{document}
