% \documentclass[a4paper,twoside,openright]{report}
\documentclass[landscape,a4paper]{report}
\usepackage[landscape,driver=xetex,a4paper,margin=25mm]{geometry} % propper margins on a4 paper

\usepackage[hidelinks]{hyperref} % make contents and references clickable in pdf
\usepackage[english]{babel}
\usepackage{fancyhdr}
\usepackage{emptypage}
\usepackage{amsmath}

\usepackage{pstricks-add}
%% \usepackage[official]{eurosym}
\usepackage[nottoc,numbib]{tocbibind}

\usepackage{appendix}

\usepackage{graphicx}
\usepackage{caption}
\graphicspath{{./plaatjes/}}
\DeclareGraphicsExtensions{.png}

% used for typeproofs
\usepackage{amsfonts}
\usepackage{amssymb}
\usepackage{bussproofs}

\usepackage[utf8]{inputenc}

% font stuff
\usepackage{fontspec}
\setmainfont[
   BoldFont={Source Sans Pro Black},
   AutoFakeSlant=0.3
]{Source Serif Pro}
\setmonofont{Source Code Pro}

\usepackage{url}
\usepackage{multicol}
\usepackage[section]{placeins}

\usepackage{enumerate}
\usepackage{enumitem}

\usepackage{listings}
\usepackage{amsmath}
\usepackage{xcolor}
\lstset{%
   basicstyle=\footnotesize\ttfamily,
   frame=single,
   breaklines=true,
   postbreak=\raisebox{0ex}[0ex][0ex]{\ensuremath{\color{red}\hookrightarrow\space}},
   keepspaces=true
}

% draft marker
\usepackage{draftwatermark}

\newcommand*\writer{Louis van der Burg}
% \renewcommand{\chaptermark}[1]{%
% \markboth{#1}{}}

%header & footer
\pagestyle{fancy}
\fancyhf{}
\fancyhead[LE,RO]{\nouppercase\leftmark}
\fancyhead[RE,LO]{\partname\ \thepart}
\fancyfoot[CE,CO]{\writer}
\fancyfoot[LE,RO]{\thepage}

%title page stuff
\fancypagestyle{plain}{%
   \fancyhf{}\fancyfoot[LE,RO]{\thepage}%
   \fancyfoot[CE,CO]{\writer}
\renewcommand{\headrulewidth}{0pt}}

\newcommand\BrText[2]{%
   \par\smallskip
   \noindent\makebox[\textwidth][r]{$\text{#1}\left\{
      \begin{minipage}{\textwidth}
         \lstset{#2}
      \end{minipage}
      \right.\nulldelimiterspace=0pt$}\par\smallskip
   }

%% \newenvironment{poliabstract}[1]
%%                {\renewcommand{\abstractname}{#1}\begin{abstract}}
%%                {\end{abstract}}

% start actual doc
\begin{document}
\title{Testing the Meta-Compiler language}
\author{\writer}
% \date{25 January 2016}

\begin{titlepage}
   \include{voorpaginamandate}
\end{titlepage}

%\maketitle
% state the problem
% say why it's an interesting problem
% say what your solution is
% say what follows from your solution

% \begin{abstract}
% MC has not been tested in the field yet and we want to know how it fares.
% We will put it to the test and see that it still needs some development, but has promise.
% \end{abstract}
% \newpage

\pagenumbering{gobble}
\setcounter{tocdepth}{2}
\tableofcontents
\cleardoublepage
\pagenumbering{arabic}
\addtocounter{page}{4}

% \begin{multicols}{2}
% samenvatting


Emerging practical patterns within programs written in a modern, state-of-the-art language

INPUT COMPACT VERSION HERE



\chapter{Introduction}
\section{Working title}
Emerging practical patterns within programs written in a modern, state-of-the-art language.
\section{Motive}
Making compilers for programming languages is a difficult task, as every language is different and needs different features.
From this problem Meta-Compiler (from now on abreviate as MC) came into existence.
As the name suggests it is a programming language made to build compilers in.
The language aims to make the process of making compilers easier.

However MC has not yet been tested in the field yet, which brings us to the goal of the assignment.

\section{Importance}
The result of this paper will give some insight of the practicality of MC and how far it is from its endgoal.
This will be used in the further development and improvment of MC.

\section{Goal}\label{sec:goalsmandate}
The main goal is to test MC with a practical application.
This is done by creating a test application which has the following steps of progressing difficulty:

\begin{enumerate}
   \item For a minnimum passing grade of 6: \newline
      A small interactive and graphical application.
   \item For an 8: \newline
      Within the application for a 6, add traversable and changable datastructures.
   \item And for a 10: \newline
      Within the application for an 8, add coroutines.
\end{enumerate}

After which the result is reflected upon.

\section{Problem}
Currently there are several working versions of MC available, but none of these implement the entire language.
We want to have a full version of MC which works as desired.
That is to say, it matches the design specifications.
However the last documented version is implemented in the latest working version of MC.
These versions are already proven to be adequate\cite{giuseppe2015mc}.

The newest development version is the one that will be used for this assignment.
This version has no working compiler yet and is therefor easily adapted to bugs found during testing.
The new version also implements new features not presen in the previous versions.
Because this newer version is still in development we want to detect any flaws and mishaps in the design during this fase.
This makes otherwise fatal flaws fixable.

The usability of the new version is also not tested yet and we want to see how MC compares to other programming languages.
For this reason we will be testing the current development version of MC.

\section{Central research question and subquestions}
From this problemstatement follows the following research question:

\textit{How good is the programming language MC in comparison to other languages?} \newline
Which is divided into these subquestions:
\begin{enumerate}
   \item What constitutes a good programming language?
   \item What is MC?
   \item How does MC work?
   \item How does MC differ with other languages?
   \item Which purpose does MC have?
   \item How practical is MC in its use?
\end{enumerate}

This will also translate to the actual assignments and goals as described in section \ref{sec:goalsmandate}.

\section{The client}
The graduation assignment will be carried out at Kenniscentrum Creating 010.
The company is located in Rotterdam.
\textit{Kenniscentrum Creating 010 is a transdisciplinary design-inclusive Research Center enabling citizens, students and creative industry making the future of Rotterdam}\cite{creating2016home}.

\section{Working environment and tasks}
During the internship the student will work as a part of the existing research team, whom do research in the Casanova and MC languages.
The student will be doing research and programming and will be supported and helped by the entire research team.
Every two weeks the student will present his progress and receive feedback.


\chapter{Method}
\section{Research methods}
During the internship there will be made use of different research methods.
This stems from the wide scope of the project and the fact that there are multiple research questions which require different approaches.

At the start there will be a preliminary research concerning MC.
This will consist of describing and comparing MC and the ways it works.

After this there will be a basic definition of the testprogram which will we written to further test MC.

When this is done there will be a evaluation of the testprogram and MC.

\section{Gathering information}
The information needed will be gathered from the research team involved, via personal interviews, and via published papers concerning the material.

\section{Validation of results}

\section{Validity and trustwortiness of sources}

\section{Project methods}
During the internship the LEAN-software development wethod will be used\cite{ries2011lean}.
Using this method gives the ability to quickly itterate through versions and gather knowledge more easily through these itterations.

\section{Risk analyses}
\section{Quality expectations}
MC makes use of typetheory and the programs written in it will therefor need to be in line with the typetheory, as described in \cite{pierce2002types}.
Because the version of MC used in this internship has no compiler yet, the student will need to manually make sure that the program has no bugs and/or errors in them.

\chapter{Results}
\section{Intended result}

\chapter{Definition of graduation assignment}
\section{The client}
\section{Motive}
\section{Goal}
\section{Scope}
\section{Conditions \& restrictions}
\section{How is the supervision organized within the company}
\section{Correlation with other projects}

\chapter{Literature}

\chapter{Stakeholders}



\include{planning}
part{intro}
\chapter{Introduction}
\section{Company}
The graduation assignment is carried out at Kenniscentrum Creating 010.
The company is located in Rotterdam.
\textit{Kenniscentrum Creating 010 is a transdisciplinary design-inclusive Research Center enabling citizens, students and creative industry making the future of Rotterdam}~\cite{creating2016home}.

The assignment is carried out within a research group, who is building a new programming language.
The new programming language is called \emph{Casanova}.


\section{The assignment}
We will now describe the assignment.
What the motive is and how it will be executed.

\subsection{Motive}
Video game development is difficult~\cite{blow2004game}.
Video games are getting bigger and, with it, more complex.
However the time between releases of the video games does not lengthen~\cite{blow2004game}.

One of the major challenges video game developers face is a short time to market.
When the developers are able to produce games faster, they have an advantage over their competitors.

This is where \emph{Casanova} comes in.
\emph{Casanova} is a programming language to make the development of games easier~\cite{maggiore2011designing}.

Casanova uses higher order types to make the programming of complex constructs easier.
These constructs are often used in game development.
A few of these constructs will be explained in detail from chapter~\ref{chap:standardlibrary} onward.

Because of the higher order types the compiler for Casanova became complex.
The compiler for Casanova worked, but was still in development.
So when a bug was found in the compiler, fixing it became more and more time consuming.
This frustrated the developers so much a new language was created~\cite{giuseppe2015mc}.

Meta Casanova (from now on called \emph{MC}) is this new language.
MC serves as the language in which Casanova will be written as a library

At the beginning of the assignment MC was still in development.
The syntax was nearly complete, but untested, and the standard library of MC was just beginning to take shape.
The assignment was created to debug and expand MC and further refine the standard library of MC.


\subsection{Research question}
As stated above the main goal of the assignment is to debug and expand MC and refine the standard library of MC.
This translates to the following research question:

\emph{How can the programming language Meta Casanova be improved for the user within the timeframe of the internship?}

Here the user is a programmer using the language to create applications.

This research question is quite broad and several sub-questions have been created to define a clearer scope for the assignment.

\begin{enumerate}[noitemsep]
   \item What is a good programming language to the user?
   \item What is MC and how does it work?
   \item How can the current syntax be improved to serve the user?
   \item How can the standard library be improved to serve the user?
\end{enumerate}

After we have a clear image of what a good programming language is, we can look at MC.
MC can then be checked and improved where necessary.

To further test MC from a users perspective a few test programs will be written in the language.

The definitions and scope of the improvements are defined by the preliminary research.


\subsection{Correlation with other projects}
During the internship the research group keeps developing Casanova.
The research group consists of
  Francesco di Giacomo\footnote{\label{venice}Universita' Ca' Foscari, Venezia},
  Mohamed Abbadi\footnoteref{venice},
  Agostino Cortesi\footnoteref{venice},
  Giuseppe Maggiore\footnote{Hogeschool Rotterdam} and
  Pieter Spronck\footnote{Tilburg University}.

Within the research group the research team developing MC consists of Douwe van Gijn, responsible for developing the back-end of the bootstrap MC compiler, Jarno Holstein, responsible for developing the front-end of the bootstrap MC compiler, and Louis van der Burg, responsible for developing the MC language.
The research team are all students and are supervised by Giuseppe Maggiore.

The student will work closely with the research team and cooperate with the research group.


part{background}
\include{problem}
\include{idea}

\chapter{interlude}
MC stands for MetaCasanova. The reason behind the name is explained in chapter \refchapter{MetaCasanova}.

We will first talk about what MC is and explain the basics of the language.
Then we will discuss the Standard Library of MC, which is written completely in MC.
After which we will show some practical examples.


\chapter{MetaCasanova}

\section{What is MC}
MC stands for MetaCasanova and is MC is a declaritive functional language.
\section{Why MC}


\section{Goal}
The goal of MC is to use higher abstractions with type safety.

\section{Basics}
We will now go through the basics of MC.

\subsection{Func}
First we need to declare a function.
   \begin{lstlisting}
Func "foo" -> Bool -> Int -> Int -> Int
   \end{lstlisting}
   For this we use the keyword \emph{Func}.
   Here we see that the function \emph{foo} is declared by a String, which is the name of the function, a Boolean and three Integers.
   The last parameter is the return type of the function and all the parameters between the name and the return type, are the arguments of the function.

   When no parameters are given the function becomes a variable, as seen in figure \ref{}.
   \begin{lstlisting}
Func "foo" -> Int
foo -> 5
   \end{lstlisting}

   \begin{lstlisting}
add b c -> res
---------------
foo True b c -> res

mul b c -> res
---------------
foo False b c -> res
   \end{lstlisting}

   Here we see the basic syntax for declaring and defining a function.

\subsection{Data}
\subsection{TypeFunc}
\subsection{TypeAlias}
\subsection{Module}


\chapter{standard library}
\section{BasicMonads}
\subsection{Monads}
\subsection{Monad transformers}
\subsubsection{Id}
\subsubsection{List}
\subsubsection{Either}
\subsubsection{Option}
\subsubsection{Result}
\subsubsection{State}
\subsubsection{IO}

\section{Standard Library}
\subsection{match}
\subsection{monad}
\subsection{prelude}
\subsection{number}
\subsection{record}
\subsection{tryableMonad}

\section{Casanova}
\subsection{Entity}
\subsection{Coroutines}


\chapter{test programs}
\section{bouncing ball}
\section{tron}
\subsection{bike}
\subsection{playfield}
\subsection{powerups}


\chapter{Conclusion}
\chapter{Reflection}

\part{work done}
\include{details}

part{conclusion/reflection}
\include{conclusion}

part{appendixes}
\include{related}

% \bibliographystyle{ieeetr}
% \bibliography{biblio}

% \end{multicols}{3}
\end{document}
