% \documentclass[a4paper,twoside,openright]{report}
\documentclass[a4paper]{article}
% \documentclass[landscape,a4paper]{report}
\usepackage[driver=xetex,a4paper]{geometry} % propper margins on a4 paper
% \usepackage[driver=xetex,a4paper,margin=25mm]{geometry} % propper margins on a4 paper
% \usepackage[landscape,driver=xetex,a4paper,margin=25mm]{geometry} % propper margins on a4 paper

\usepackage[hidelinks]{hyperref} % make contents and references clickable in pdf
\usepackage[english]{babel,isodate}\isodate
\usepackage{fancyhdr}
\usepackage{emptypage}
\usepackage{amsmath}

\usepackage{pstricks-add}
%% \usepackage[official]{eurosym}
\usepackage[nottoc,numbib]{tocbibind}

\usepackage{appendix}

\usepackage{graphicx}
\usepackage{caption}
\graphicspath{{./plaatjes/}}
\DeclareGraphicsExtensions{.png}

% used for typeproofs
\usepackage{amsfonts}
\usepackage{amssymb}
\usepackage{bussproofs}

\usepackage[utf8]{inputenc}

% font stuff
\usepackage{fontspec}
\setmainfont[
   BoldFont={Source Sans Pro Black},
   AutoFakeSlant=0.3
]{Source Serif Pro}
\setmonofont{Source Code Pro}

\usepackage{url}
\usepackage{multicol}
\usepackage[section]{placeins}

\usepackage{enumerate}
\usepackage{enumitem}

\usepackage{listings}
\usepackage{amsmath}
\usepackage{xcolor}
\lstset{%
   basicstyle=\footnotesize\ttfamily,
   frame=single,
   breaklines=true,
   postbreak=\raisebox{0ex}[0ex][0ex]{\ensuremath{\color{red}\hookrightarrow\space}},
   keepspaces=true
}

% draft marker
\usepackage{draftwatermark}

\newcommand*\writer{Louis van der Burg}
% \renewcommand{\chaptermark}[1]{%
% \markboth{#1}{}}

%header & footer
\pagestyle{fancy}
\fancyhf{}
\fancyhead[LE,RO]{\nouppercase\leftmark}
% \fancyhead[RE,LO]{\partname\ \thepart}
\fancyhead[RE,LO]{\writer}
\fancyfoot[CE,CO]{\thepage}
% \fancyfoot[LE,RO]{\thepage}

%title page stuff
\fancypagestyle{plain}{%
   \fancyhf{}\fancyfoot[LE,RO]{\thepage}%
   \fancyfoot[CE,CO]{\writer}
\renewcommand{\headrulewidth}{0pt}}

\newcommand\BrText[2]{%
   \par\smallskip
   \noindent\makebox[\textwidth][r]{$\text{#1}\left\{
      \begin{minipage}{\textwidth}
         \lstset{#2}
      \end{minipage}
      \right.\nulldelimiterspace=0pt$}\par\smallskip
   }

%% \newenvironment{poliabstract}[1]
%%                {\renewcommand{\abstractname}{#1}\begin{abstract}}
%%                {\end{abstract}}

% start actual doc
\begin{document}
\title{Debugging and expanding MetaCasanova}
\author{\writer}
% \date{25 January 2016}

\begin{titlepage}
   \include{voorpagina}
\end{titlepage}

% \maketitle
\begin{multicols}{3}
   % state the problem
   % say why it's an interesting problem
   % say what your solution is
   % say what follows from your solution

   \begin{abstract}
      \textbf{Write shit here as very short summary}
      mc needs testing and expanding
      standard library has been expanded
      test programs have been created
      conclusion has been made
   \end{abstract}
   \newpage

   %%%%%%%%%%%%%%%%%%%%%%%%%%%%%%%%%%%%%%%%%%%
   %%%%%%%%%%%%%%% real paper %%%%%%%%%%%%%%%%
   %%%%%%%%%%%%%%%%%%%%%%%%%%%%%%%%%%%%%%%%%%%

   \pagenumbering{gobble}
   \setcounter{tocdepth}{1}
   \tableofcontents
   \cleardoublepage
   \pagenumbering{arabic}
   \addtocounter{page}{4}


\end{multicols}
\part{Preface}

\section{Stakeholders}
% \renewcommand{\abstractname}{Stakeholders}
% \begin{abstract}
   % \hfill
   % \vfill
   % \columnbreak
% \begin{multicols}{2}

   \begin{tabular}
      { l l }
      \textbf{The graduate} & \\
      Name & Louis van der Burg \\
      Student number & 0806963 \\
      E-mail address & louis.burg@hotmail.com \\
      Telephone number & +31 6 14 85 55 79 \\
      & \\
      \textbf{Client} & \\
      Name company & Kennis Centrum Creating 010 \\
      Name company supervisor & Sunil Choenni \\
      E-mail address & h.choenni@hr.nl \\
      Telephone number & +31 6 48 10 03 01  \\
      Function & Lector \\
      Visitors address & Wijnhaven 103 verdieping 6 \\
      Website company & www.creating010.com \\
   % \end{tabular}
   % \begin{tabular}
      % { l l }
      & \\
      \textbf{Company supervisor} & \\
      Name company & Kennis Centrum Creating 010 \\
      Name company supervisor & Sunil Choenni \\
      E-mail address & h.choenni@hr.nl \\
      Telephone number & +31 6 48 10 03 01  \\
      Function & Lector \\
      Visitors address & Wijnhaven 103 verdieping 6 \\
      Website company & www.creating010.com \\
      & \\
      \textbf{School supervisors} & \\
      Name examinator (first teacher) & Giuseppe Magiorre \\
      E-mail address & +31 6 41 78 12 23 \\
      Telephone number & g.maggiore@hr.nl \\
      & \\
      Name assessor (second teacher) & Hans Manni \\
      E-mail address & j.p.manni@hr.nl \\
      Telephone number & \\
      & \\
      \textbf{School co\"ordinator} & \\
      Name graduate coördinator INF/TI & Aad van Raamt \\
      Telephone number & 010 7944993 \\
      E-mail address & A.van.Raamt@HRO.NL \\
   \end{tabular}
% \end{multicols}
% \end{abstract}


\begin{multicols}{3}
   \chapter{Introduction}
   % MC stands for MetaCasanova. The reason behind the name is explained in chapter \refchapter{MetaCasanova}.

   % We will first talk about what MC is and explain the basics of the language.
   % Then we will discuss the Standard Library of MC, which is written completely in MC.
   % After which we will show some practical examples.

   % explain why the layout landscape and three collumns

   \chapter{Assignment}
\section{Company}
The graduation assignment is be carried out at Kenniscentrum Creating 010.
The company is located in Rotterdam.
\textit{Kenniscentrum Creating 010 is a transdisciplinary design-inclusive Research Center enabling citizens, students and creative industry making the future of Rotterdam}\cite{creating2016home}.

The assignment is carried out within a research group, who is building a new programming language.
The new programming language is called \emph{Casanova}.

% \section{Supervision \& work methods}

% \subsection{Working environment and tasks}\label{subsec:workenvmandate}
% During the internship the student will work as a part of the existing research team, whom do research in the Casanova and MC languages.
% The student will be doing research and programming and will be supported and helped by the entire research team.
% Every two weeks the student will present his progress and receive feedback.

% \subsection{Research methods}
% During the internship there will be made use of different research methods.
% This stems from the wide scope of the project and the fact that there are multiple research questions which require different approaches.

% At the start there will be a preliminary research concerning MC.
% This will consist of describing MC and the ways it works.

% After this there will be a basic definition of the test program which will we written to further test MC.

% When this is done there will be a evaluation of the test program and MC.

% \subsection{Gathering information}
% The information needed will be gathered from the research team involved, via personal interviews, and via published papers concerning the material.

% \subsection{Validation of results}
% The validity of the results will be secured by regularly discussing with the client and the supervisor.
% Further more there will be programs will be double-checked by the supervisor and the other members of the research team.

% \subsection{Validity and trustworthiness of sources}
% The sources used will be published papers only and relative to the subject matter.
% Other sources, such as personal communications will be with people who are approved by the supervisor.

% \subsection{Project methods}
% During the internship the LEAN-software development method will be used\cite{ries2011lean}.
% Using this method gives the ability to quickly iterate through versions and gather knowledge more easily through these iterations.

% \subsection{Risk analyses}
% \begin{center}
   % \begin{tabular}
      % {| p{0.2\textwidth} | p{0.25\textwidth} | l | p{0.3\textwidth} |}
      % \hline
      % \textbf{Risk} & \textbf{Effect} & \textbf{Possibility} & \textbf{Counter measure}
      % \\ \hline
      % Quick development of the language. & Because the language becomes bigger as time progresses, the assignment could become too large to do in the set time. & 70\% & The analyses of MC goes on for a set period, which prevents the analyses going an for too long.
      % \\ \hline
      % The compiler is not finished at the time the programs need to be written. & The programs can not be tested in practice. & 80\% & The programs will be manually checked for errors by the student and the research team.
      % \\ \hline
      % The company goes bankrupt during the internship & The assignment will not be finished and the student will not be able to graduate. & 1\% & There is nothing the student can do to prevent this.
      % \\ \hline
   % \end{tabular}
% \end{center}

% \subsection{Quality expectations}
% MC makes use of type theory and the programs written in it will therefor need to be in line with the type theory, as described in \cite{pierce2002types}.

% Because the version of MC used in this internship has no compiler yet, the student will need to manually make sure that the program has no bugs and/or errors in them.
% This will happen with the help of the compiler developers.

\section{Motive}

MetaCasanova (from now on called \emph{MC}) comes forth from the language Casanova, hence the name.

Casanova is a language made for building games.
It uses higher order types to make the programming of complex constructs easier.
These constucts are often used in game development.
A few of these constructs will be explained in detail from chapter~\ref{chap:standardlibrary} onward.

Because of the higher order types the compiler for Casanova became complex.
The compiler for Casanova worked, but was still in development.
So when a bug was found in the compiler, fixing it was becoming more and more time consuming.
This frustrated the developers so much a new language was created.

MC was this new language.
MC serves as the language in which the compiler for Casanova will be written.

At the beginning of the assignment MC was still in development.
The syntax was nearly complete, but untested, and the standard library of MC was just beginning to take shape.
The assignment was created to debug and expand the language.

I took this assignment and have written my bachelor thesis on it.

\section{Goal}
As stated the main goal of the assignment is to debug and expand the language.
Which translates to the following research question:
\emph{How can the programming language MetaCasanova be improved from its current state within the timeframe of the internship?}
This goal is quite broad and I have created a few subgoals to define a scope for the assignment.

\begin{enumerate}[noitemsep]
   \item Double checking the existing design of the language.
   \item Participating in the design of the language.
   \item Extending the standard library.
   \item Creating test programs.
\end{enumerate}

I will now explain the subgoals in greater detail.

\subsection{Double checking the existing design of the language}
At the start of the assignment MC will have to be checked for any existing faults.
This way we can ensure that the language has a safe base point from which it can be developed further.

The checking of MC will consist of the existing syntax and the ideas for which the syntax was created.
This will be done by exploring the current syntax and understanding the reasons why it is this way.

\subsection{Participating in the design of the language}
As MC is still in development and my assignment consists of debugging and expanding, I will propose design changes and new designs.
These designs will be discussed and reviewed with the rest of the team.

This will be done in an exploratory way as the language grows and changes.

\subsection{Extending the standard library}
The standard library of every language is an extention of the functionality of the language.
It contains often used constructs and functions which makes the language more productive for the programmer.

MC has but a small standard library and this needs to be expanded in order to be of any practical use.
This also could help make MC a success~\cite{khedker1997makes}.

The standard library of MC will be completely written in MC itself.
This will show how powerfull the simple and effective syntax of MC is.

Extending the library also gives me the position of a user of MC.
For now the only people that have used MC are the developers themselves.
Now that I will join the research team they will get a fresh and new look at the language through me.

\subsection{Creating test programs}
Writing the standard library is one aspect of a language.
Another are actual applications.

The standard library can be seen as an application, but it is different in the sense that they create standard functions to be used by programmers.
Applications are not part of the code base and also serve users outside of the programming community.

Because the only users so far have been the developers, there have only been small test programs written.
They often test a specific function of the compiler.
The test applications will test the entire language.

Because of the timelimit I will not be able to create a huge application, but I will be able to create bigger ones than those currently used for testing.

I have chosen to create game or game-like test prog




\section{Correlation with other projects}
During the internship the research team keeps developing MC.
There are also two other students doing an internship on MC.
They are both working on the compiler, one is developing the back-end of the compiler and the other the front-end.
The student will cooperate with the research team and both these students.


% \section{Execution}
% The assignment will consist of three stages:

% \begin{enumerate}
   % \item Research on MC
   % \item Expanding the standard library
   % \item Creating test programs
% \end{enumerate}

% \subsection{Research on MC}
% This will describe MC and the way it works.

   % \subsection{Expanding the standard library}
   % \subsection{Creating test programs}
% This will consist of describing MC and the ways it works.

% After this there will be a basic definition of the test program which will we written to further test MC.

% When this is done there will be a evaluation of the test program and MC.


\end{multicols}
\part{Preliminary research}
\begin{multicols}{3}

   \chapter{MC in detail}\label{chap:basicmc}
We will now look at MC.
%... in the state after the project.
After we have a basic understanding of MC, we will look at the improvements made.

Because the latest version of MC is not documented, the following information comes from interviews with the research team~\cite{researchteaminterview}.
This chapter also serves as the documentation for this version of MC.


\section{What is MC}
MC is a declarative functional language.
It tries to be a completely pure language, which means no side-effects are allowed directly.


\section{Goal}
The goal of MC is to use higher order types with type safety, in a natural way of programming.
These higher order types give the ability to resolve certain computations at compile time, which would normally be performed at runtime.
This can substantially increase performance at runtime.
The higher order types also add expressive power by being able to capture abstractions.

MC also supports the .NET library natively.
Because .NET has mutables~\cite{dotnetdescription}, MC can have side-effects through the use of .NET.

MC aims to be as flexible as possible, with the safety of a strong type system.


\section{Basics}
We will now go through the basics of MC.
We will look at how the syntax works and use small code samples to explain them.

The improvements made to these basics will be given in part~\ref{part:mcexpanded}.
This will give us a better understanding of why and how MC evolved.

Inside the code blocks a line wrap is indicated by \ensuremath{\color{red}\hookrightarrow\space}.


\subsection{Terms, types \& kinds}\label{sec:basiclevels}
With MC you program on three different levels:
\begin{enumerate}[noitemsep]
   \item Terms
   \item Types
   \item Kinds
\end{enumerate}
Terms are the values of variables, like \verb\5\, \verb\'c'\ or \verb\''Hello''\.
Types are the types of variables, like \verb\Integer\, \verb\String\ or \verb\Boolean\.
Kinds are to types, what types are to terms.
You could say that kinds are the types of the types.


\subsection{Func}\label{sec:basicfunc}
First we need to declare a function before defining it.
For this we use the keyword \verb|Func|.

\begin{lstlisting}
Func "computeNumber" -> Boolean -> Int -> Int -> Int
\end{lstlisting}

Here we see that the function \texttt{computeNumber} is a function which goes from a \verb\Boolean\ and two \verb\Integers\ to an \verb\Integer\.

The parameters are separated by the ASCII arrow, \verb|->|.
The last parameter is the return type of the function and all parameters between the name and the return type are the arguments of the function.

A function is defined by one or multiple \emph{rules}.
A \emph{rule} consists of a \emph{conclusion} and can contain multiple \emph{premises}.
They are separated by the \emph{bar}.
The conclusion is placed underneath the bar and the premise(s) above.

Now we define a rule for \verb\computeNumber\.

\begin{lstlisting}[xleftmargin=5.0ex,numberstyle=\tiny\color{gray},numbers=left]
a == True
add b c -> res
--------------------------
computeNumber a b c -> res
\end{lstlisting}

The conclusion is on the fourth line, the bar on the third and the premise on the first and second.
The conclusion has the name of the function and the input arguments on the left of the arrow and the output argument on the right.
This syntax is similar to that of natural deduction~\cite{prawitz1965natural}.

\subsubsection{Locally defined identifiers}
The identifiers named in the conclusion of a function are locally defined.
Within the rule of \verb\computeNumber\ the identifiers \verb\a\, \verb\b\ and \verb\c\ are examples of these local identifiers.


\subsection{Multiple rules}
There can also be multiple rules which form the function definition:

\begin{lstlisting}
a == True
add b c -> res
---------------
computeNumber a b c -> res

a == False
mul b c -> res
---------------
computeNumber a b c -> res
\end{lstlisting}

When this happens the rules are activated in order of definition.
A rule fails when the arguments do not match.
If the rule fails, the next rule is activated.
If none of the rules match, the program crashes.

Let us go through this process with an example.

First we will try the following function call, from inside a premise:
\begin{lstlisting}
computeNumber False 5 3 -> newValue
-----------------------------------
\end{lstlisting}

The first rule to be attempted is the one written first in the source code, so:
\begin{lstlisting}
a == True
add b c -> res
--------------------------
computeNumber a b c -> res
\end{lstlisting}

This rule will fail, because argument \verb\a\ has the value \verb\False\ and not \verb\True\.
So this rule will not be executed.

The next rule to be attempted is:
\begin{lstlisting}
a == False
mul b c -> res
--------------------------
computeNumber a b c -> res
\end{lstlisting}
Argument \verb\a\ is still \verb\False\ and \verb\b\ and \verb\c\ are both \verb\Integers\, so the rule matches.
The rule will be executed and the result will be put in \verb|newValue|.


\subsection{Constants}
Constants can be created by using \verb\Func\ without any parameters.
The return type still has to be declared.

\begin{lstlisting}
Func "bar" -> Int
bar -> 5
\end{lstlisting}

Here the constant \verb\bar\ is set to the value \verb\5\.

We also see that not every rule has a \emph{bar}.
This is possible only when no premise is needed in the rule.

\bigskip

As we have seen \verb|Func| uses types in its declaration and terms in its definition.
The types enable the detection of wrong function calls during compile time.

Important to note is that functions declared by \verb\Func\ are runtime constants.
MC also has the ability to declare and define functions that run at compile time.\footnote{As will be explained in section~\ref{sec:runcompiletime} and shown in section~\ref{sec:basictypealias}.}

\subsection{Generic type identifiers}\label{sec:basictypeidentifiers}
To indicate generic types the \verb\'\ notation is used.

This is enhanced with the use of \emph{identifiers} after \verb\'\, for example: \verb\'a\ or \verb\'b\.

These give the generic type a name and can be used to express if one generic type is the same as another generic type.


\subsection{Functions as parameters}
Aside from variables, functions can also be given as parameters.
This needs to be specified in the declaration of the function.

\begin{lstlisting}
Func "isThisEven" -> ('a -> 'b) -> 'a -> 'd
\end{lstlisting}

Here we see that the first argument \verb\('a -> 'b)\ is a function.
This is clear because the entire argument goes from \verb\'a\ to \verb\'b\, \verb\'a\ is the input and \verb\'b\ the output of the function \verb\('a -> 'b)\.


\subsection{Data}\label{sec:basicdata}
With \verb|Data| we can declare a constructor and deconstructor for a new type.
Because \verb\Data\ can work two ways, by constructing and deconstructing types, it effectively creates an alias for types.

Say that we want to create a union, a type which can be any of the predefined types within a set~\cite{igarashi2006union}.
For this we need a way to construct the union type and a way to deconstruct the union back to its original type.
The deconstructor is needed to get to know which of the constructors inside the union is actually used.
Only then can we know which of the types inside the union is used.

As an example we will create a union of a String and a Float.
\begin{lstlisting}
Data "Left"  -> String -> String | Float
Data "Right" -> Float  -> String | Float
\end{lstlisting}

Here we create two constructors whom both go to the same type, \verb\String | Float\.
\verb\Left\ and \verb\Right\ can be seen as an alias for the \verb\String | Float\ union.

Let us see how this is used.
We need a function which takes a union as argument.

\begin{lstlisting}
Func "isFloat" -> ( String | Float ) -> Boolean
\end{lstlisting}

The parentheses in the declaration can be left out, but are used to clarify that the union is one argument.

In the definition we want to check which type the union is.
This can be done directly in the conclusion of a rule.

\begin{lstlisting}
isFloat (Right x) -> True
\end{lstlisting}

The deconstructor of the argument given automatically checks if the types match.
This can be done because they are aliases created by \verb\Data\.
Aliases are automatically resolved to their basic types by the compiler.

Conditionals using aliases can be put in the conclusion.
Other conditionals have to be put in the premises, because they need to be computed.\footnote{As is shown in the example of section~\ref{sec:basicfunc}}

To make the definition of \verb\IsFloat\ complete we also need a rule that checks for \verb\String\.

\begin{lstlisting}
isFloat (Left x) -> False
\end{lstlisting}

Now we will test the function \verb\isFloat\ with the help of constant \verb\iAmFloat\.

\begin{lstlisting}
Right 9.28 -> iAmFloat
\end{lstlisting}

\verb\iAmFloat\ has type \verb\String | Float\ and can now be passed to \verb\isFloat\.

\begin{lstlisting}
isFloat iAmFloat -> output
\end{lstlisting}

When \verb\iAmFloat\ is deconstructed to its original type it will match with the first rule of \verb\isFloat\.
The identifier \verb\output\ now contains the value \verb\True\.


\subsection{Runtime and compile time}\label{sec:runcompiletime}
So far we have seen the \verb\->\ arrow, which indicates a function is executed on runtime.

To indicate that a function is resolved or inlined at compile time, the big ascii arrow, \verb\=>\, is used.

Types are resolved or inlined at compile time, except when they create data structures.
These data structures keep existing on runtime.
Kinds are also resolved or inlined at compile time.

We will see this starting from the next section onwards.


\subsection{TypeAlias}\label{sec:basictypealias}
Because the \verb\|\ (\emph{pipe}) operator\footnote{Used in section~\ref{sec:basicdata}} is not built in the language we need to create it.
To do this we need to manipulate types.

To make \emph{pipe} work, we both need a constructor and deconstructor that works with kinds.

\verb\TypeAlias\ is the same as \verb\Data\ only on a higher level of abstraction.
It constructs and deconstructs kinds.

With \verb\TypeAlias\ we can create a generic \emph{pipe} with which we can create unions.

\begin{lstlisting}
TypeAlias Type => "|" => Type => Type
\end{lstlisting}

Here we see how the generic \emph{pipe} is declared.
It takes two parameters of any type and returns a type.

The \verb\Type\ is a kind and indicates an unknown type is used.

Parameters may also be infix, as shown in the example above.


\subsubsection{Infix parameters}
There is a limitation to the use infix arguments, namely there may only be one.
This is because of the parser\footnote{A parser analyses the source code and puts it in a structure easily used by the rest of the compiler.} used for the bootstrap compiler of MC.
The parser would become overly complex when adding multiple arguments on the left.

Because the benefits of having multiple arguments in front of the function name does not outweigh the time it consumes to expand the parser, the choice was made to only allow one infix argument.


\subsection{TypeFunc}\label{sec:basictypefunc}
\verb|TypeFunc| creates a function on type level.
It can perform computations with both types and terms.

We will demonstrate the power of \verb\TypeFunc\ with a function that manipulates a type.
The function will take a tuple and return the same tuple, but with switched types.

First we need to declare a tuple with \verb\TypeAlias\.

\begin{lstlisting}
TypeAlias Type => "*" => Type => Type
\end{lstlisting}

Now that we have a tuple we can use it in the function declaration.
The tuple argument will have generic types so it can work with any tuple.

\begin{lstlisting}
TypeFunc "switch" => Type => Type
\end{lstlisting}

Next we have the definition.
Here we need to deconstruct the tuple to its original arguments.
These arguments can then be used to create the return tuple.

\begin{lstlisting}
a => (c * d)
(d * c) => res
---------------
switch a => res
\end{lstlisting}

Here we can see how the types inside the tuple are switched and returned as a new tuple.

{
   \begin{wraptable}{r}{0.3\textwidth}
      \begin{tabular}
         { | l | l | l | }
         \hline
         \textbf{Runtime} & \textbf{Compile time} \\
         \hline
         \verb\->\ & \verb\=>\ \\
         \hline
         \verb\Data\ & \verb\TypeAlias\  \\
         \hline
         \verb\Func\ & \verb\TypeFunc\ \\
         \hline
      \end{tabular}
      \caption{Relations between the keywords and arrows in MC}
      \label{table:relationskeywords}
   \end{wraptable}

   \subsubsection{Parameters}
   Important to note is that both \verb\TypeAlias\ and \verb\TypeFunc\ can use kinds, types and terms.
   Because they work on compile time only, they cannot replace \verb\Data\ and \verb\Func\.

\verb\Data\ and \verb\Func\ exist for explicit runtime functions.
   The relations between the keywords can be seen in table~\ref{table:relationskeywords}.


   \subsection{Modules}\label{sec:basicmodule}
}
\verb|Module| is used to create a container at compile time.
It is a collection of function declarations and definitions.
Having a collection of functions is useful when they serve a combined or general purpose.

It does not fit within the levels defined in section~\ref{sec:basiclevels}.
Modules can only be declared by \verb\TypeFunc\.

Say we want to create a container for every type which does an addition of two identifiers of these types.
The addition could be declared when we need them or we can put them in a \verb\Module\.

When we create an addition container using a module, we do not have to type out the addition functions for every type.
We can just call the module to create the functions for us.

A \verb\Module\ is declared with a \verb\TypeFunc\.
\begin{lstlisting}
TypeFunc "Add" => Type => Module
\end{lstlisting}

Here we declare \verb\Add\ to take a type and return a \verb\Module\.

We then want to define the module \verb\Add\ with the \verb\+\ operator and the \verb\identityAdd\ constant.

\begin{lstlisting}
Add 'a => Module {
  Func 'a -> "+" -> 'a -> 'a
  Func "identityAdd" -> 'a
}
\end{lstlisting}

This creates a basic container with generic addition functionality.

As shown above \verb\Module\ can contain \verb|Func|, but can also contain \verb|Data|, \verb|TypeFunc|, \verb|TypeAlias| and another \verb|Module|.

The declarations within the Module do not have to be defined.
This way we can define different behavior for every instance of the module.

It is possible to have a function declared and defined within a module.

When we want to define the above declared \verb|Add| with an \verb\integer\, we instantiate the module over \verb\Int\.
We then define the functions which are declared within \verb\Add\ to finish the module.

\begin{lstlisting}
TypeFunc "IntAdd" => Module
IntAdd => Add Int {
  Int -> "+" -> Int -> Int
  identityAdd -> 0
}
\end{lstlisting}

\verb\Func\s are defined as they normally are, only wrapped inside a \verb\Module\.


\subsubsection{The caret}
When we want to call a function within a module from outside the module, the \verb\^\ (\emph{caret} operator is used.

\begin{lstlisting}
Func "caretTest" -> Int
caretTest -> identityAdd^IntAdd
\end{lstlisting}

This creates a constant named \verb\caretTest\ with the same value as \verb\identityAdd\ from the module \verb\IntAdd\.
Important to note is that we cannot call a function outside the module from within the module.
The \emph{caret} notation is derived from \LaTeX.
In \LaTeX\space the \emph{caret} is used to indicate superscript~\cite{beebe1992latex}.
This is used is MC to show that one element is part of another element like so: \verb~identityAdd~\textsuperscript{\texttt{IntAdd}}.


\subsubsection{Inherit}
Modules can also inherit from other modules.
This is done with \verb\inherit\.

We will create a module which inherits from the \verb\Add\ module.
The operator \verb\-\ will then be added as well.

\begin{lstlisting}
TypeFunc "GroupAdd" => Type => Module
GroupAdd 'a => Module {
  inherit Add 'a
  Func 'a -> "-" -> 'a -> 'a
}
\end{lstlisting}

The module \verb\Add\\footnote{As used in section~\ref{sec:basicmodule}} is instantiated with \verb\'a\.
The created \verb\Add\ module is then inherited into \verb\GroupAdd\.

The function declarations and definitions of the inherited module are directly usable within the new module.

If a module takes another module as an argument, the module can also directly inherit.

\begin{lstlisting}
TypeFunc "GroupAdd" => Add => Module
GroupAdd M => Module {
  inherit M
}
\end{lstlisting}

Now everything from module \verb\M\ is inherited into \verb\GroupAdd\.

When inheriting a module which contains an inherit, these are also inherited in to the current module.
This works recursively.


\subsection{Priority}
We can also give a priority together with a declaration.
The priority indicates the order of function execution.

The priority is indicated by the \verb\#>\ (\emph{hash arrow}) and is placed after the declaration.
When we look at \verb\Data\ it will look like this:

\begin{lstlisting}
Data "Left"  -> String -> String | Float  #> 7
Data "Right" -> Float  -> String | Float  #> 5
\end{lstlisting}

And with \verb\Func\ and \verb\TypeFunc\ it is used like this:
\begin{lstlisting}
Func "bar" -> Int  #> 12
TypeFunc "foo" => Float => 'a => 'b  #> 9
\end{lstlisting}

The \emph{hash arrow} is also used to indicate associativity.
Everything is left associative by default, but if we want to make it right associative we can do that by placing an \verb\R\ after the \emph{hash arrow}.

\begin{lstlisting}
Func Int -> "\" -> Int -> Int  #> R
\end{lstlisting}

Priority and associativity can also be used in combination with each other.

\begin{lstlisting}
Func Int -> "\" -> Int -> Int  #> 25 R
\end{lstlisting}


\subsection{Import}
\verb\import\ is used when importing other files in the current file.

If the file imported has any imports, those are ignored.
Unlike \verb\inherit\, \verb\import\ is not recursive.

The programmer can directly use the declarations and definitions used in the imported file.
When the programmer wants to explicitly specify the use of a function from the imported file, the \emph{caret} is used.

For example we can import the file \emph{vector} and use the \verb\Vector2\ type from it, as seen below:

\begin{lstlisting}
import vector

Func "location" -> Vector2
location -> createVector2^vector 8.9 19.0
\end{lstlisting}

Importing of MC files always works with non capitalized import statements, even when the actual files does contains capital letters.
This is done to differentiate between MC imports and .NET imports.

The order of elements is the same as with modules, only when using .NET imports there is a difference in syntax.
When calling a .NET function the .NET syntax is used to differentiate between MC and .NET functions.

\begin{lstlisting}
import System

Func "dotNetTest" -> String
dotNetTest -> DateTime.Now.ToString()
\end{lstlisting}

Here \verb\dotNetTest\ becomes a String which contains the current date and time.

.NET functions can be used on both run- and compile time

\subsection{Builtin}
Some things cannot be created with MC and need to be buildin in the compiler.
An example of this is the boolean data type.

When using such a builtin literal the keyword \verb\builtin\ is used.
We can now correctly implement the function \verb\computeNumber\\footnote{As used in section~\ref{sec:basicfunc}} using the \verb\builtin\ keyword:

\begin{lstlisting}
a == True^builtin
add b c -> res
---------------
computeNumber a b c -> res

a == False^builtin
mul b c -> res
---------------
computeNumber a b c -> res
\end{lstlisting}

It can be seen as if \verb\False\ and \verb\True\ are imported from the language itself.


\subsection{Lambdas}
MC also has anonymous functions to implement lambdas~\cite{barendregt1984lambda}.

The basic syntax is as one would expect of a lambda, it has arguments and returns what the function body of the lambda returns

\begin{lstlisting}
(\ arguments -> functionBody )
\end{lstlisting}

In practice this would look like:

\begin{lstlisting}
(\ a -> divide 10 a )
\end{lstlisting}

Lambdas do not need to be declared because they always are generic functions.
The typechecker will check lambdas the same way as any other generic function.
From the context of what the function body does together with the types of the arguments, it can deduce what the output type should be.


\subsection{ArrowFunc}\label{sec:basicmcarrowfunc}
\verb\ArrowFunc\ always needs an argument on the left of the name and at least one function as argument on the right.
A unique feature of \verb\ArrowFunc\ is the two syntax options it has when the function gets called.

The first syntax option is a regular function and the second converts the function to a lambda.

As with \verb\Func\ it needs a declaration and a definition.

\begin{lstlisting}
ArrowFunc Int -> ":~>" -> (Int -> Int) -> Int
f a -> res
-------------
a :~> f -> res
\end{lstlisting}

Here \verb\:~>\ takes an integer and a function that takes an integer and returns an integer.

The declaration and definition are the same as the rest of MC.

When we call \verb\:~>\, there is the regular way:

\begin{lstlisting}
a :~> f -> res
\end{lstlisting}

And there is the second way:

\begin{lstlisting}
{a :~> out
  f out} -> res
\end{lstlisting}

The brackets are used to indicate that everything inside them belongs together.
When an \verb\ArrowFunc\ is called like this it gets converted to a lambda.
The first argument is then used as the argument for that lambda.

\begin{lstlisting}
(\ out -> f out ) a -> res
\end{lstlisting}

The result from this is then returned as the output in \verb\res\.

This syntax exists for when the nesting of lambdas occur.
Lambdas become unreadable when nested.

\begin{lstlisting}
(\ b -> (\ c -> f c ) b ) a -> res
\end{lstlisting}

When using the \verb\ArrowFunc\ syntax it nests neatly.

\begin{lstlisting}
{a :~> b
  {b :~> c
    f c}} -> res
\end{lstlisting}

This syntax stems from the use of the \emph{bind} function of monads, which will be explained in further detail in section~\ref{sec:standardmonad}.


\subsection{Partial application}
Lastly MC supports partial application.
This means that you do not have to give all the arguments to a function.
When this is done a closure is created and given as output.

We will create a closure with the function \verb\add\.
\verb\add\ takes two \verb\Integers\ and returns an \verb\Integer\.

In the definition we will say the two arguments will be added together.
\begin{lstlisting}
Func "add" -> Int -> Int -> Int

a + b -> res
--------------
add a b -> res
\end{lstlisting}

Now we will give only one argument to \verb\add\ and create a new function \verb\addThree\.

\begin{lstlisting}
add 3 -> addThree
\end{lstlisting}

With this construct we have created the partially applied function \verb\addThree\.
When we look at what it contains we see that it has a closure with a partially implemented \verb\add\.

\begin{lstlisting}
3 + b -> res
--------------
add 3 b -> res
\end{lstlisting}

We can now give \verb\addThree\ its second argument to complete the closure.
This can be done multiple times.

\begin{lstlisting}
addThree 6 -> foo

addThree 7 -> bar
\end{lstlisting}

The identifier \verb\foo\ has a value of \verb\9\ and \verb\bar\ has a value of \verb\10\.


\subsection{The main function}
While MC has no specific main function it does specify which function acts the main function.
The final function in the file compiled acts as the main function and is executed when the program is run.



% \textbf{MAYBE REMOVE THIS???? CHECK WITH REST OF DOCUMENT}
% \subsection{Type annotations}\label{sec:basictypeannotations}
% The angle brackets are used as type annotations for \verb\pipe\.
% They indicate that \verb\'a\ and \verb\'b\ are part of the type \verb\pipe\.
% This is only used with generic types, because it makes it easier for the parser to resolve the types.\footnote{More on this in section~\ref{sec:syntaxtypeannotations}}

% When type annotations are used there can be no partial application.
% The type annotations need to know what types it gets or they will fail.
% This is further explained in section~\label{sec:syntaxtypeannotations}.

\bigskip

Now that we know the basics of MC lets take a look at how the syntax has evolved and improved.



\end{multicols}
\part{MC expanded}\label{part:mcexpanded}
\begin{multicols}{3}

   \chapter{Syntax evolution}
Here we will look into the overal syntax changes of the language.
The changes concerning the standard library, will be discussed in chapter~\ref{chap:standardlibrary}.

\section{Generic type identifiers}
As described in section~\ref{sec:basictypeidentifiers} generic type identifiers are indicated by \verb\'\ together with a name, \verb\'a\.
The \verb\'\ is derived from the \emph{prime} notation, which indicates a derivative of the original~\cite{primesomething}.

For example when we have identifier \verb\i\, this identifier might get changed and is then indicated by \verb\i'\.
This shows that \verb\i'\ is derived from \verb\i\.

Instead of placing the \emph{prime} at the end, it is placed before the identifier name to indicate it is a generic type.

This improved the readability of the language.
It is now clear to the user when generic types are the same of different from each other.


\section{Generic kind identifiers}
Generic kind identifiers are currently unused, but it was thought to be necessary after the implementation of \verb\TypeAlias\.

When working with kinds they are always generic, as they indicate an unknown type.
That was why they were indicated by an \emph{asteriks}, \verb\*\.

When \verb\TypeAlias\ was created it became apparent that the type would be the same in some cases.
This resulted in generic kind identifiers.

The notation devised was similiar to the generic type identifiers.
Instead of the \emph{prime} the \emph{hash} was used.

\begin{lstlisting}
TypeFunc "unknownKinds" -> #a -> #b -> #a -> #c
\end{lstlisting}

Here we can see that the first and third arguments are of the same kind and the second argument and the return value are different kinds.

However this was an error in reasoning.

The fact that they are kinds means that they can be any type.
The official notation used for kinds is either \verb\Type\ or \verb\*\.

The syntax of kinds was changed to \verb\Type\.
This way the the \verb\*\ was free to be used for other things, like the tuple in section~\ref{sec:basictypefunc}.

This also improved the readability of the language.
It is now clear when a kind is used.

Because of the use of \verb\Type\ it is also instanly clear what a kind is.
This reduces the complexity of the language.


\section{Type annotations}\label{sec:syntaxtypeannotations}
Generic type annotations were created during development to lower the complexity of the parser.

We use a data declaration which creates an array.
It takes a generic type and a Integer.

Using the notation without type annotations it looks like this:

\begin{lstlisting}
Data "array" -> 'a -> Int -> Array
\end{lstlisting}

The Integer sets the lenght and the generic type will be the type the array consists of.

When this is called we can use parentheses to indicate which arguments are grouped together.

\begin{lstlisting}
array(Int 12) -> arrayOfInts
\end{lstlisting}

The Integer array is created with twelve elements.

But when we use a variable which contains the type, there is confusion what really happens.

\begin{lstlisting}
array(var 12) -> arrayOfInts
\end{lstlisting}

This could imply that \verb\var\ is a function which uses \verb\12\ as its argument and returns a type.
\verb\arrayOfInts\ is then a closure which still needs an argument which tells the lenght of the array.

We can only be sure what happens if we look at the declaration of \verb\var\.

To avoid this and clarify what happens the type annotations were created.
They are used in the data declaration, to specify the types used.

\begin{lstlisting}
Data "array" -> 'a -> Int -> Array<'a Int>
\end{lstlisting}

Types using the type annotations cannot be partial applied, because they need to know the types they contain to be declared correctly.

When this is not done a compile time error is generated.

So now when we call \verb\array\ we know that \verb\var\ must be a type.
When \verb\var\ is declared as a function which returns a type, we will have to call \verb\array\ like this:

\begin{lstlisting}
TypeFunc "var" => Int => Type

nr == 12
------------
var nr -> Int

array((var 12) 9) -> arrayOfInts
\end{lstlisting}

This also lowers the complexity of the parser.
With the use of type annotations, the parser does not have to look at \verb\var\ and what it is.

The use of the type annotations provides greater predictability to the language.
It is now always clear what happens when working with generic types.


\section{TypeAlias}
Before there was \verb|TypeAlias|, \verb|TypeFunc| was used for the same functionality.
However \verb\TypeFunc\ cannot deconstruct.

When constructing a new kind with \verb\TypeFunc\ it can not be deconstructed to its original kind.

This error was noticed when updating the monadic part of the standard library, see section~\ref{sec:standardmonad}.

\verb\Data\ could not be used as it cannot manipulate types it can only construct and deconstruct them.
That is why \verb|TypeAlias| was created.
It provides contructing and deconstructing functionality with type manipulation.

This adds expessivity and functionality to the language.
Which is especially usefull for the user when creating complex programs.


\section{Module}
\subsection{Signature}
Earlier in development \verb|Module| was called \verb|Signature|.
It performed the same functionality as \verb|Module|.

From the language creators perspective \verb|Module| creates a specific signature on compile time.
For the user it looks more like a container or class, when coming from object-oriented programming.

Because the user will be the one actually using the language, the name was changed to what it resembles to the user.
It clarifies the syntax and makes it more predictable for the user.


\subsection{Expanding}
During the development there were two ways of inheriting a module, via \verb\inherit\ and via expanding an existing module.
Expanding modules is removed from MC, because of inconsistency issues.

A \verb\Module\ could be expanded via a function.
The function takes a module as argument and expands this module by adding declarations and, optionally, definitions.

\begin{lstlisting}
TypeFunc "expandModule" => Module => Module
expandModule M => M{
  Func "modulo" -> Float -> Float

  a % 10 -> res
  ---------------
  modulo a -> res
}
\end{lstlisting}

The module \verb\M\ is opened up again and \verb\modulo\ is added.
Module \verb\M\ is then returned and contains the new function \verb\modulo\.

This would have been syntactic sugar for creating a new module which inherits from \verb\M\.
In this new module \verb\modulo\ is then added.

\begin{lstlisting}
TypeFunc "expandModule" => Module => Module
expandModule M => Module{
  inherit M

  Func "modulo" -> Float -> Float

  a % 10 -> res
  ---------------
  modulo a -> res
}
\end{lstlisting}

The expand syntax makes it look like modules are mutable, which they are not.
This makes it inconsistent with the rest of MC and the expand-syntax was removed.

By not using the expanding functionality the language becomes more coherent and in line with itself.
This makes the language friendly to the user, because there is no two ways of doing things.
It is clear how inherit works and there is no confusion possible.


% \section{priority}
% \textbf{DO THIS!!!!!!!!}
% During the development of the language priorities were needed.
% They are necessary to implement the order of operators being executed.

% For this the

% why the \verb\#>\
% there was a need for priorities
% first only implemented without associativity
% then with right, left being the default


\section{.NET libraries}
When importing and calling .NET libraries the current syntax is using the element order of .NET with \verb\^\ as divider.

\begin{lstlisting}
Func "dotNetTest" -> String
dotNetTest -> System^DateTime^Now^ToString()
\end{lstlisting}

This was not always used.
The syntax for .NET imports went through several stages.

The first implementation was the original .NET manner:

\begin{lstlisting}
dotNetTest -> System.DateTime.Now.ToString()
\end{lstlisting}

This seemed to strange considering the rest of the MC syntax.
The order of the elements was then switched.

\begin{lstlisting}
dotNetTest -> ToString().Now.DateTime.System
\end{lstlisting}

This made it more like MC but it was still out op place because of the dot used.
It also looked strange when calling a method.

In the next itteration the dot was replaced with the caret.

\begin{lstlisting}
dotNetTest -> ToString()^Now^DateTime^System
\end{lstlisting}

This seemed more in sync with the rest of MC.
The only problem was with the method calls.

That is how we arrived at the current syntax used.
The .NET order of elements is used and the \verb\^\ is used as seperator.

The current syntax is the best compromise from a user perspective.
It takes the most central elements of both syntaxes and combines them into a clear definite syntax.


\section{Builtin}
The keyword \verb\builtin\ was first called \verb\primitives\.

For the developers this was a good choice, as they indicate the primitives that needed to be built in the compiler.

For the user however this seemed confusing.
The user sees the \verb\primitives\ as types that are built into the language.

The name was changed to \verb\builtin\, to better reflect the way how the user sees them.


% \section{Data}
% why it is not called alias.


\section{Conditionals}
\subsection{Inside the conclusion}
During development we tried implementing conditionals inside the conclusion of a rule.
This could make the code more compact.

We will take the example of \verb\Func\ from section~\ref{sec:basicfunc}, and implement this.
The code is then compacted to this:

\begin{lstlisting}
add b c -> res
---------------
computeNumber True b c -> res
\end{lstlisting}

This is just as easy to read as the original.

It does make the conclusion less conclusive as there is now something happening inside the conclusion.
The conclusion is meant to show the input and output of the rule, it is not meant to tell what the function does.

That is where the premises are for.

The compiler also needed to be modified heavily.
It now needed to do computations inside the conclusion.

To keep the language and the compiler modular, a comprise was done.
Types created with \verb\Data\ and \verb\TypeAlias\ can be used as conditionals inside the conclusion.
This is possible because they act as aliases and need no computation to be checked.

Conditionals requiring computation, a comparison of values, will be done in the premises.

This is also syntactically predictable.
Especially so because the \verb\Data\ and \verb\TypeAlias\ used in conditionals, are aliases.
They could be replaced with that from which they are constructed.


\subsection{Equals}
When comparing values the \verb\==\ operator is used.
This is done to keep the the single equals sign free to be used by the programmer.

It is also immediately clear to the programmer a conditional is done.
The single \verb\=\ could be mistaken for an alias creation of value assignment.
With the double \verb\==\ there is no such confusion.


   \chapter{Standard library}\label{chap:standardlibrary}
We will now look at the standard library and explain the evolution it went through during the project.

When we have an idea of how it should work, the choices made during the development will be clearer.

\section{Prelude}\label{sec:standardprelude}
\emph{Prelude} is the first item in the standard library and contains a few definitions making basic programming easier.
\emph{Prelude} contains the definitions which are not large enough to grant there own item within the standard library.

The .NET \emph{System} namespace is imported and the type \verb\Unit\ is created.

\begin{lstlisting}
import System

Data "unit" -> Unit
\end{lstlisting}

The \verb\Unit\ is used as an empty or null value.

Next we have the declaration and definition of a generic tuple and union.
Because they are generic \verb\TypeAlias\ is needed for the type manipulation.

\begin{lstlisting}
TypeAlias Type => "*" => Type => Type
Data 'a -> "," -> 'b -> 'a * 'b    #> 5

TypeAlias Type => "|" => Type => Type
Data "Left" -> 'a -> 'a | 'b       #> 5
Data "Right" -> 'b -> 'a | 'b      #> 5
\end{lstlisting}

A standard if-else construct is then implemented as follows:

\begin{lstlisting}
TypeAlias "Then" => Type
Data "then" -> Then
TypeAlias "Else" => Type
Data "else" -> Else

Func "if" -> Boolean^System -> Then -> 'a -> Else -> 'a -> 'a
if True^builtin then f else g -> f
if False^builtin then f else g -> g
\end{lstlisting}

The \verb\then\ and \verb\else\ \verb\Data\s are syntactic sugar and could be left out.
We have left them in to enhance the clarity of the if-else construct.

Now that we have the if-else construct we can do basic boolean math.
It also makes the use of multiple rules an option and not a necessity, which in turn creates more freedom for the programmer.

Here we see how \emph{System} of .NET is used.
The boolean of .NET can be imported as shown, however the values \verb\True\ and \verb\False\ cannot be imported.
They are built in .NET itself.
This is why the boolean literals are built into MC as well and need to be called using \verb\builtin\.


\subsection{Match}
Next we have the \verb\match\ function.
\verb\match\ is used to check the constructor of the \emph{pipe} operator and makes computations with the \emph{pipe} easier.

It takes a variable, matches it on either \verb|Left| or \verb|Right| and executes the function specified.

\begin{lstlisting}
TypeAlias "With" => Type
Data "with" -> With

Func "match" -> ('a | 'b) -> With -> ('a -> 'c) -> ('b -> 'c) -> 'c
\end{lstlisting}

There is some syntactic sugar created to make it clearer how \verb\match\ works.
The first argument is the union which needs to get matched.
The arguments \verb\('a -> 'c)\ and \verb\('b -> 'c)\ are the functions to be executed on each branch of the match.

In this case the two functions both take an argument from the union, the \verb\'a\ and the \verb\'b\, and both return \verb\'c\.

Next we have the definition of \verb\match\.
\verb\match\ needs a definition for both cases of the match, namely the \verb\Right\ and the \verb\Left\.

\begin{lstlisting}
match (Left x) with f g -> f x
match (Right y) with f g -> g y
\end{lstlisting}

When it matches the \verb\Left\ it executes function \verb\f\ with \verb\x\ as its argument.
And when it matches the \verb\Right\ it executes function \verb\g\ with \verb\y\ as its argument.

% \textbf{THIS IS PROBABLY NOT POSSIBLE}
% However this does not take into account the possibility of nested pipes.
% For this we have written a third definition of \verb\match\:

% \begin{lstlisting}
% match y with g h -> res
% --------------------
% match (Right y) with f (g h) -> res
% \end{lstlisting}

% This definition is placed above the previous defined \verb\match\ which matches on \verb\Right\.
% Then there is first checked on nested pipes and when there are none the actual value is checked.

% In this manner there is no possibility of skipping any nested pipes.


\subsection{List}
We have a generic list implementation, which very useful considering MC has no built in arrays.
For this we need to create a list that can work with types.

The list is declared with a union.
The union is necessary because we need an end to the list.

\begin{lstlisting}
TypeAlias "List" => Unit | (Type * (List Type))
\end{lstlisting}
% List unit => Left unit
% List ('a * 'b) => Right ('a * 'b)

The list is created with a tuple.
An end to the list is created with a union of the list tuple and \verb\Unit\.

We also need the basic operators to create lists.
This is done with \verb\::\, which takes a type and a list of that same type.
And with \verb\empty\ an empty list is created.

\begin{lstlisting}
Data 'a -> "::" -> 'b -> List (Right ('a * 'b))
Data "empty" -> List (Left unit)
\end{lstlisting}

With \verb\::\ we can specify the head and tail of a list.

But we would also like to concat lists together.
Concatenation is done with the \verb\@\ operator.

\begin{lstlisting}
Func List 'a -> "@" -> List 'a -> List 'a  #> 200
empty @ l -> l
(x :: xs) @ l -> x :: (xs @ l)
\end{lstlisting}

And when we want to apply a function to the entire list we call \verb\map\.
\verb\map\ executes a function and creates a new list, which it then returns.

\begin{lstlisting}
Func "map" -> List 'a -> ('a -> 'b) -> List 'b
map empty f -> empty
map (x :: xs) f -> (f x) :: (map xs)
\end{lstlisting}

When we want to find one or more elements in the list we can call \verb\filter\.
It checks the entire list using a predicate function and returns the matching elements.

\begin{lstlisting}
Func "filter" -> List 'a -> ('a -> Boolean^System) -> List 'a
filter empty p -> empty

(if p x then
  (x :: (filter xs p))
  else
  (filter xs p)) -> res
-------------------------
filter (x :: xs) p -> res
\end{lstlisting}

The programmer can specify the predicate function \verb\p\.


\subsection{Evolution of prelude}
We will now go over the evolution of the separate parts of \emph{prelude}.

\subsubsection{Boolean}
Boolean currently does not exist anymore as a separate part of the standard library.
Here we will see why this is the case.

Boolean was created to implement boolean logic.
After many iterations it was discarded due to the evolution of the language.

When first implementing Boolean it was part of \emph{Prelude}.

\begin{lstlisting}
Data "True" -> Boolean^System
Data "False" -> Boolean^System
\end{lstlisting}

This might seem correct at first glance, but \verb\True\ and \verb\False\ have no meaning here.
There is no way to check whether a boolean value is true or false.
That is why a new notation had to be created for the boolean literals, which ended up into a module with the literals as \verb\Func\s.

\begin{lstlisting}
import System

TypeFunc "Boolean" => Module
Boolean => Module {
  Func "True" -> Boolean^System
  Func "False" -> Boolean^System
}
\end{lstlisting}

The Boolean module could then be implemented in \emph{prelude}.

\begin{lstlisting}
TypeFunc "boolean" => Boolean
boolean => Boolean {
   True -> True^builtin
   False -> False^builtin
}
\end{lstlisting}

The boolean literals could now be called using \verb\True^boolean\.
This way was noticeably the same as what happened inside the implementation of \verb\boolean\.
The choice was then made to call the boolean literals directly with \verb\True^builtin\, which made the boolean module obsolete.

This choice brings us to the current state of the boolean literals.


\subsubsection{Match}
\emph{Match} started as a separate part of the standard library containing the \verb\match\ module.
We will now look at how it was first created and how it has evolved into \emph{prelude}.

The first iteration of the \verb\match\ module started with the import of \emph{prelude} and the declaration of the \verb\match\ module.

\begin{lstlisting}
import prelude

TypeFunc "match" => Type => Module
\end{lstlisting}

So far there are no problems yet with us a module for \verb\match\.

Next we see the first part of implementation of the \verb\match\ module:
Some syntactic sugar is created and the actual function is defined which will do the matching.

\begin{lstlisting}
match ('a | 'b) => Module {
  Data "with" -> With

  TypeFunc "Left" => Type
  Left => 'a

  TypeFunc "Right" => Type
  Right => 'b

  Func "doMatch" -> 'c -> With -> (Left -> 'd) -> (Right -> 'd) -> 'd
\end{lstlisting}

Now we see why the module is needed.
It was thought that the \verb\match\ module would get a separate instance for every match that was executed.

When separate functions, like these, are needed to check the validity of another function it is better to group them together inside a module.

The definitions of \verb\doMatch\ have not changed compared to what they are currently.

  % doMatch y with g h -> res
  % --------------------
  % doMatch (Right y) with f (g h) -> res
\begin{lstlisting}
  doMatch (Left x) with f g -> f x
  doMatch (Right y) with f g -> g y
}
\end{lstlisting}

It became clear that \verb\Left\ and \verb\Right\ were not necessary, because they were simply passing type of the union.

With this removal the use of a module became obsolete, because we do not need to group a single function.
The \verb\doMatch\ function was then renamed \verb\match\ and placed inside \emph{prelude} directly.


% \subsubsection{Recursive match}
% The first itteration of match could not match nested pipes.
% It could only detect a direct match.
% As it only had the direct matches without the function that checks if \verb\Right\ is nested.

% The solution could have been left to the programmer.
% He would have to create a separate match statement for every level of nesting.
% This would be quite inconvenient for the programmer.

% For that reason the recursive functionality was added.

\subsubsection{List}
The first list was put together with the list monad.
This seemed the logical choice at the time, because the list monad needed a list implementation to use inside the monad.

Monads will be explained in section~\ref{sec:standardmonad} and the list monad in section~\ref{sec:standardmonadlist}.

When coming back to the list monad it became apparent that \verb\List\ was not specific to the list monad.
It could be used without the monad as a regular list implementation.

The choice was then made to move it to \emph{prelude}.


\section{Number}
\emph{Number} is created to give the user a generic interface to create numbers.

Because of the import system of MC, MC can directly import the integer and float types with all their functions from .NET.
\emph{Number} is therefor used for self defined number types.

\emph{Number} is build up from different modules to give the programmer the freedom to choose what their custom numbers can do.
The modules are arranged according to the mathematical groupings to built up numerical operators~\cite{bourbaki1998commutative}.

It starts with the \verb\MonoidAdd\ module.
It declares the \verb\+\ operator for the custom number.

\begin{lstlisting}
TypeFunc "MonoidAdd" => Type => Module
MonoidAdd 'a => Module {
  Func 'a -> "+" -> 'a -> 'a #> 60
  Func "identityAdd" -> 'a
}
\end{lstlisting}

\verb\MonoidAdd\ needs the identity as a base number from which the operations will work.

Next we have the \verb\GroupAdd\ module.
\verb\GroupAdd\ inherits everything from the module \verb\MonoidAdd\ and adds the \verb\-\ operator.

\begin{lstlisting}
TypeFunc "GroupAdd" => Type => Module
GroupAdd 'a => Module {
  inherit MonoidAdd 'a
  Func 'a -> "-" -> 'a -> 'a #> 60
}
\end{lstlisting}

We do the same for multiplication and dividing.

\begin{lstlisting}
TypeFunc "MonoidMul" => Type => Module
MonoidMul 'a => Module {
  Func 'a -> "*" -> 'a -> 'a #> 70
  Func "identityMul" -> 'a
}

TypeFunc "GroupMul" => Type => Module
GroupMul 'a => Module {
  inherit MonoidMul 'a
  Func 'a -> "/" -> 'a -> 'a #> 70
}
\end{lstlisting}

Now we have declared the basic number operations of addition, subtraction, multiplication and dividing.

We can combine them into a basic number like so:

\begin{lstlisting}
TypeFunc "Number" => Type => Module
Number 'a => Module {
  inherit GroupAdd 'a
  inherit GroupMul 'a
}
\end{lstlisting}

Or create a Vector without the dividing operator, because vectors cannot be divided~\cite{hazewinkel2013encyclopaedia}.

\begin{lstlisting}
TypeFunc "Vector" => Type => Module
Vector 'a => Module {
  inherit GroupAdd 'a
  inherit MonoidMul 'a
}
\end{lstlisting}

Of course all the operators still have to be defined, but that is up to the programmer.
This gives the programmer the freedom to choose what the operators do.


\subsection{Evolution of number}
During development \emph{number} was used to create integers and floats.
This was before the .NET import system was in place.
Currently it is only needed for user defined number types.

The first iteration of \emph{number} was just one module with all the operators declared.
The first iteration works for the standard number types like \verb\Integer\ and \verb\Float\, but not \verb\Vectors\, because they cannot be divided.

For this reason the current \emph{number} design is similar to the mathematical way.
When implementing \emph{number} we only define what we need, which reduces the overhead generated.


\section{Record}\label{sec:standardrecord}
With \emph{Record} we can create a list of key/value pairs which can be searched and manipulated.
The special thing about this record is that it does all the searching on compile time.
This makes the runtime code much faster.

% When using records in practice we see that not everything can will be known on compile time.
% Which makes a record that works on compile time useless.

% The way MC fixes this is to inline the code on compile time.
% In this way the code that needs to be actually executed can be optimised by the compiler after which it is executed on runtime.

Record is declared as a module and contains \verb\TypeFunc\s to make it work at compile time.

\begin{lstlisting}
TypeFunc "Record" => Module
Record => Module{
  TypeFunc "Label" => String
  TypeFunc "Field" => Type
  TypeFunc "Rest" => Record
\end{lstlisting}

Here we see the declarations which are the basis of a list of key/value pairs.
\verb\Label\ acts as the key, \verb\Field\ as the value paired with the key and \verb\Rest\ contains the rest of the record.

When there is just one pair in the record we can get that record out easily, as we already have it.
But when there are multiple record entries in a record we need a function to search through the record to find the sub record we want to have.
For this the \verb\get\ function was created.

\begin{lstlisting}
  TypeFunc "get" => String => Record => Record

  (if (l == Label^rs) then
    rs
  else
    get l Rest^rs) => res
  -----------------------
  get l rs => res
\end{lstlisting}

The \verb\get\ function takes a labels and a record.
It checks the label of the record and if it is the same it returns the current record and if not the it recursively calls itself with the rest of the record.

To manipulate these records there the \verb\set\ function was created.
The \verb\set\ function takes a label, a field and a record.
The field is the new value that needs to be paired with the label.

\begin{lstlisting}
  TypeFunc "set" => String => Type => Record => Record
  (if (l == Label^rs) then
    RecordEntry l f rs
  else
    (set l f Rest^rs -> rs'
     RecordEntry Label^rs Field^rs rs')) => res
  -------------------------
  set l f rs => res
}
\end{lstlisting}

Because MC does not allow direct manipulation of values, the set function returns a new record instead of changing the value of the specified record.
When the record does not match the current record \verb\set\ is called recursively, and the result is used as the new \verb\Rest\ for the current record.

With this the declaration of the record is complete.
Now we can look at how the definition of a record works.

First we need to create a record entry.
This is done with the \verb\TypeFunc\ \verb\RecordEntry\.

\begin{lstlisting}
TypeFunc "RecordEntry" => String => * => Record => Record
RecordEntry label field rest => Record{
  Field => field
  Label => label
  Rest => rest
}
\end{lstlisting}

We also need to create an empty record entry so we can end the record.
This is done by the \verb\TypeFunc\ \verb\EmptyRecord\.

\begin{lstlisting}
TypeFunc "EmptyRecord" => RecordEntry
EmptyRecord => RecordEntry {
  Field => Unit
  Label => Unit
  Rest => Unit
}
\end{lstlisting}

The programmer has to use \verb\EmptyRecord\ as the \verb\rest\ argument of the first record entry.
This creates an end to the record.


\subsection{Updatable record}
The current record can only be instantiated and not updated.
To expand the record to be updatable we need to add the update function.

First we declare and define a \verb\TypeFunc\ which creates a new record from the original with inherit.

\begin{lstlisting}
TypeFunc "updatableRecord" => Record => Record
updatableRecord r => Module {
   inherit r
\end{lstlisting}

The new module now contains everything of record \verb\r\.

Then we declare and define the update function.

\begin{lstlisting}
  TypeFunc "update" => Record => Type => Record
  r == Empty
  --------------------
  update r dt => Empty
}
\end{lstlisting}

Only the definition of \verb\update\ which checks for an empty record can be defined.
It is the only situation in which we can be certain what the return value will be, namely \verb\Empty\.

If the record is not \verb\Empty\ we cannot know what the programmer wants to do with \verb\update\.
That is why we leave that definition of \verb\update\ up to the programmer.



\subsection{Evolution of record}
Several parts of \verb\Record\ have changed.
We will go over them here.

\subsubsection{Set}
The original version of \verb\set\ checked the field-argument to be the same type of \verb\Field\ inside the record.
For this \verb\Fields\ was needed, which contained the type of \verb\Field\.

However the checking of \verb\Fields\ needed to happen in the declaration of the \verb\set\.
This required the declaration to have premises and this is not possible, because the declaration cannot be variable.

The field-argument also did not need to be checked, because \verb\Field\ of a record can be anything.\footnote{As described in section~\ref{sec:standardrecord}.}

\subsubsection{Set return}
The return value of \verb\Set\ was first only the record that was changed and not all the previous records in the record structure.
This was fixed by returning the recursive call to \verb\set\ as the new value of the current record.

Now the entire record structure is returned and not just the record that is changed.


\subsubsection{Cons}
During the development the idea occurred that every \verb\Record\ had to have its own type signature.
Because the field can be anything, it was thought to impact the type of the \verb\Record\.

This would be a problem when trying to declare the use of a specific \verb\Record\.

This resulted in the creation of \verb\Cons\.
\verb\Cons\ would be the type signature of that particular \verb\Record\.

It would consist of the types of the \verb\Label\, \verb\Field\ and \verb\Rest\.

\begin{lstlisting}
  TypeFunc "Cons" => Type
  Cons => (Label,Field,Rest)
\end{lstlisting}

It became apparent that every record is of type \verb\Record\ and that the type of \verb\Field\ had no impact on the type of the record.
\verb\Cons\ was then removed.

This also reduced the complexity of the records for the user.
Now the user can use the records without having to keep in mind which type belongs to which record.


   \section{Monad}\label{sec:standardmonad}
Now we come to the module \verb\Monad\.
This module takes a mathematical concept and makes into practical programming construct.

First we are going to look at what monads exactly are within programming.
Then we will see how they are implemented within MC.

\subsection{Monads}
Monads are a container like, generic interface.
They contain two functions:

\begin{itemize}
   \item Return
   \item Bind
\end{itemize}

These two functions create the basis on which the monads are built.

\subsubsection{The return}
The return function takes a value and wraps it inside a monad.

{
   \centering
   \includegraphics[width=\columnwidth]{return}\\
   \captionof{figure}{The return}\label{fig:monadreturn}
}

The box in figure~\ref{fig:monadreturn} visualizes the monad and \verb\2\ is the value that is being put inside the monad.
With the return function any value can be put inside a monad.
When it is inside the monad it cannot be seen by the outside world.

This is where the bind comes in.

\subsubsection{The bind}
Having a monad is fine, but what if you want to manupilate the value of the monad?
The bind function supplies this functionality.

{
   \centering
   \includegraphics[width=\columnwidth]{bind1}\\
   \captionof{figure}{A function and a monad}\label{fig:monadreturn}
}

When we want to apply the function \verb\(+3)\ to the monad containing \verb\2\, we call the bind.
The bind unpacks the monad.

{
   \centering
   \includegraphics[width=\columnwidth]{bind2}\\
   \captionof{figure}{The return}\label{fig:monadreturn}
}

Applies the function to the value.
And then packs the new value inside the monad.

Using these two basic functions, we can always work with monads.
The monads have then become the generic interface of every function.
This gives us greater safety when moving values between functions, because the values thrown between functions are always monads.

\subsubsection{More than a wrapper}
But just a monad offers very little besides being a wrapper for values.
That is why there have been created a few different sorts of monads.

We will discuss two of these for now.
More will be explained when looking at the implemented monads in MC, see section~\ref{sec:standardimplementedmonads}.

\paragraph{The state monad}
gives monads the ability to behave like mutables.
It takes a \emph{state} and returns a new state and a return value.

Say we want to roll a dice.
The state monad will simulate the \emph{state} of the dice being roled.
The state monad is given a \emph{state} of the dice and from this it will compute the new state of the dice and the number roled.

In MC the functionality might look like this:

\begin{lstlisting}
Func "state" -> 's -> ('a,'s)
state dice -> (numberRoled,newDice)
\end{lstlisting}
MAYBE LEAVE IT IN BUT PROBABLY BEST TO LEAVE CODE OUT OF STATE MONAD FOR NOW.


\paragraph{The maybe monad}
offers the ability of the value to be either a value or nothing.
For example we could utilize the pipe operator for this:

\begin{lstlisting}
Data "Maybe" -> 'a -> 'a | Unit

Func "test" -> Maybe -> Boolean^System
(match a with
  (\ x -> True^builtin)
  (\ unit -> False^builtin)) -> res
-----------------------------------
test a -> res
\end{lstlisting}

Here we use \verb\Unit\ as the nothing and 'a as the value.
In the example the term a is checked for a being a value, the \verb\Left\, or being nothing, the \verb\Right\.

This is a very basic example of how the maybe works.
It acts like a pipe and can contain either a value, \verb\Left\, or nothing, \verb\Right\.

\subsubsection{Combining monads}
Apart from having different sorts of monads, they can also be combined.
Since they are the same generic interface they can be combined into another generic inteface.
These new monads can utilise the functionality of the all combined monads.

In this manner the functionality of a monad can be extended.

{
   \centering
   \includegraphics{transformers}\\
   \captionof{figure}{Combining monads}\label{fig:monadcombining}
}

Combining monads can be seen as putting one monad inside another monad, as illustrated in figure~\ref{fig:monadcombinig}.

We will take the state and maybe monad and combine these into a parser monad.
A parser monad can parse through a data structure and scan for certain elements.

The state monad will be used for the scanning functionality and the maybe monad will be used to determine of the elemnt has been found.
The state monad returns the maybe monad as its result.
When the maybe monad has a value as result the parser monad stops.

Like the normal monads they have to be build manually.
This is a very error prone process when combining more than two monads or when using more complex monads.

What we actually want is to write the monads once and use combine them automatically.
This is where monad transformers come in.

\subsection{Monad transformers}
Monad transformers are a way to automatically combine monads.
Instead of simply having a monad which takes the arguments it needs, it also takes another monad transformor as one of its arguments.
This monad transformer is then used to wrap the result.

Combining monads this way is quite useful when having complex monads.
Because the combining happens naturally with monad transformers, there are no errors created in the process of combining.

MC has an implementation of the basic monad transformer with which all monad transformers can be created.

\subsection{Implementation}\label{sec:basicmonadsimplementation}
Now we will look at how the monad transformer is implemented.
With this monad transformer we will be able to create actual monads, this can be seen as the basic interface of a monad.

First we take a look at the declaration:

\begin{lstlisting}
TypeFunc "Monad" => (#a => #b) => Module
\end{lstlisting}

Monads are implemented as modules.
This way we can put the return and bind functions inside a monad.

The first argument is a function on type level.
It represents the monad transformer it uses to wrap the result in.

When looking at the implementation we see that a module is created with the return and the bind:

\begin{lstlisting}
Monad 'M => Module {
  ArrowFunc 'M 'a -> ">>=" -> ('a -> 'M 'b) -> 'M 'b   #> 10
  Func "return" -> 'a -> 'M 'a
\end{lstlisting}

The bind is declared with \verb\ArrowFunc\ and is called \verb\>>=\.
It takes a monad as its first argument and the function to create the new monad as its second argument.
It returns the monad created with the function it takes.

The return takes a value, \verb\'a\, and wraps it inside the monad, \verb\'M\.

Both the bind and the return are only declared and not defined, because they are different for every monad created.
When creating a new monad the bind and return will need to be defined.

Sometimes we want to directly pass the value inside the monad instead of wrapping it.
This is done with \verb\returnFrom\.

\begin{lstlisting}
  Func "returnFrom" -> 'a -> 'a
  returnFrom a -> a
\end{lstlisting}

This function can be defined in the monade interface, because it is the same for every monad.

Next we need to know the signature of the monad wrapped inside the monad created.

\begin{lstlisting}
  TypeFunc "MCons" -> #a
  MCons -> 'M
\end{lstlisting}

This is needed when declaring new functions which utilise the wrapped monad.

Then we have the lift functionality.

\begin{lstlisting}
  Func "lift" -> ('a -> 'b ) -> 'M 'a -> 'M 'b
  {a >>= a'
     return f a'} -> res
  ----------------------
  lift f a -> res
\end{lstlisting}

The \verb\lift\ takes a regular function, that goes from one value to another, and a monad.
This monad is then unwrapped, \verb\a >>= a'\, and the value is passed to the function, \verb\f a'\.
The result of this is then wrapped again inside the monad and returned with the monadic \verb\return\.

The \verb\Lift\ can also be declared with functions which take two arguments.

\begin{lstlisting}
  Func "lift2" -> ('a -> 'b -> 'c) -> 'M 'a -> 'M 'b -> 'M 'c
  {a >>= a'
     b >>= b'
       return f a' b'} -> res
  ---------------------------
  lift2 f a b -> res
\end{lstlisting}

This can be expanded to functions which take any number of arguments.

There are ofcourse functions which work with the wrapped monad.
For this \verb\liftM\ is created.

\begin{lstlisting}
  Func "liftM" -> (MCons^'M 'a -> MCons^M' 'b) -> 'M (MCons^'M 'a) -> 'M (Mcons^'M 'b)
  {M >>=^'M a
    f a -> b
    return^'M b} -> res
  ---------------------
  liftM f M -> res
}
\end{lstlisting}

It works the same as the other lift functions, with the specification of using the bind and return of the wrapped monad, \verb\'M\, instead of the bind and return of the current monad.

Now we have the complete interface of the monad transformer.
With this we can combine and built monads automatically.


\subsection{Evolution}


\section{TryableMonad}
Apart from the regular monadic interface there is also the tryable monadic interface.
This creates a monad with a few extra functions, namely the \emph{try}.

It uses the regular monadic interface to create a new interface.

\begin{lstlisting}
TypeFunc "TryableMonad" => (#a => #b) => Module
TryableMonad 'M => Monad(MCons^'M) {
  inherit 'M
\end{lstlisting}

As declared \verb\TryableMonad\ takes a monad as its argument and creates a new module.
It uses the type signature of \verb\'M\ to create a new monad and inherits the functions from \verb\'M\.

Here we also see the first use of \verb\MCons\.
When creating monads using other monads this function comes in very handy.

Then we declare the \verb\try\ function and initialise \verb\e\.

\begin{lstlisting}
  Func "try" ('a -> MCons^'M 'b) => ('e -> MCons^'M 'b) => MCons^'M 'a => MCons^'M 'b
\end{lstlisting}

If \verb\try\ succeeds the first function will be executed and if it fails the second function will be executed.

We also want to be able to get the original monad out of the tryable monad.
This is done with \verb\getMonad\.

\begin{lstlisting}
  $$ return the monad of the tryable monad, this way you can use the tryable monad as a the normal monad
  Func "getMonad" -> MCons^'M
  getMonad -> 'M
}
\end{lstlisting}

Here the original monad is given when \verb\getMonad\ is called.

\subsection{Evolution}

\section{Implemented monads}\label{sec:standardimplementedmonads}
\subsection{Id}\label{sec:basicmonadsimplementedid}
The main problem with using monad transformers is that the always take another monad.
There is no end to the chain.

That is why we have the \emph{id} monad transformer.
It takes no monad as argument and simply returns all values as the are.

First wee need a type signature to tell the monad interface which monad we are creating.

\begin{lstlisting}
TypeAlias "Id" => #a => #b
Id 'a => 'a
\end{lstlisting}

Here we can see what the id monad should do, simply pass the type on as it was given.

Now we will declare and define the id monad:

\begin{lstlisting}
TypeFunc "id" => Monad
id => Monad(Id) {
  bind x k -> k x
  return x -> x
}
\end{lstlisting}

Here we see how the monad transformer interface in called and instantiated with the type signature created with \verb\Id\.

\verb\Id\ also takes a \verb\'a\, but this is never used as the id monad does not need an extra type to say which type it really uses.
While the \verb\'a\ is not used, it necessary to have them in the type signature.
Else \verb\Id\ cannot be used to instantiate \verb\Monad\, as evident from section~\ref{sec:basicmonadsimplementation}.

Other monad transformers which do actually use the \verb\'a\ also do not instantiate them immediatly.
The programmer can decide for which type the monads will be created, therefor it is left open.
This effectively creates an interface for the defined monad to be instantiated with a type later specified.

The \verb\return\ simply returns the exact same value and \verb\bind\ executes the function \verb\k\ with \verb\x\ as its argument.

When using the \verb\id\ monad transformer with another transformer, it becomes that monad.
Because the values are directly returned from the \verb\id\ transformer the values are directly used in the other monad transformer, thus creating that monad.


\subsection{List}
The list monad is a monad with a list inside itself.

First we needed to create a list that can work with types.

\begin{lstlisting}
TypeAlias "List" => #a => Unit | (#a * (List #a))
List unit => Left unit
List 'a => Right ('a * (List 'a))
\end{lstlisting}

It uses the pipe operator to create \verb\List\.
This is necessary because we need an end to the list, which in this case is \verb\Unit\.
In \verb\Right\ the list is created recursively with the help of a tuple.

We also need the basic operators to create lists.

\begin{lstlisting}
TypeAlias #a -> "::" -> List -> List
'a :: 'b => List ('a * 'b)

TypeAlias "empty" -> List
empty => List unit
\end{lstlisting}

A list can now be created with the \verb\::\ operator.
We can also intatiate an empty list with \verb\empty\.

But we would also like to concat lists together.

\begin{lstlisting}
Func List 'a -> "@" -> List 'a -> List 'a  #> 200
empty @ l -> l
(x :: xs) @ l -> x :: (xs @ l)
\end{lstlisting}

This is done with the \verb\@\ operator.

And when we want to apply a function to the entire list we call \verb\map\.

\begin{lstlisting}
Func "map" -> List 'a -> ('a -> 'b) -> List 'b
map empty f -> empty
map (x :: xs) f -> (f x) :: (map xs)
\end{lstlisting}

This executes a function and creates a new list, which it then returns.

When we want to find one or more elements in the list we can call \verb\filter\.

\begin{lstlisting}
Func "filter" -> List 'a -> ('a -> Boolean^System) -> List 'a
filter empty p -> empty

(if p x then
  (x :: (filter xs p))
  else
  (filter xs p)) -> res
-------------------------
filter (x :: xs) p -> res
\end{lstlisting}

The programmer still has to create the function that actually does the checking, but \verb\filter\ checks the entire list and returns a new list.
The returned list contains the elements which are specified in function \verb\p\.

Now that we have the complete \verb\List\ type defined and its functions, we can create the actual monad.
First we have to declare and define the list monad signature.

\begin{lstlisting}
TypeAlias "ListT" => (#a => #b) => #c => #d
ListT 'M 'a => 'M(List 'a)
\end{lstlisting}

Using \verb\ListT\ together with the signature of \verb\'M\ we create the \verb\list\ monad.

\begin{lstlisting}
TypeFunc "list" => Monad => Monad
list 'M => Monad(ListT MCons^'M) {
\end{lstlisting}

\verb\ListT\ also takes a \verb\'a\, which is the type for the list created.
As stated in section~\ref{sec:basicmonadsimplementedid} it will be left up to the programmer to do this.

When we look at the return it simply puts the value in a list and packs it into the monad.

\begin{lstlisting}
  return x -> return^'M(x :: empty)
\end{lstlisting}

The bind looks a bit more complicated.

\begin{lstlisting}
  {lm >>= l
    (match^prelude l with
      (\empty -> return^'M empty)
      (\(x :: xs) ->
        {x >>=^'M y
          return^'M k y} -> z
        xs >>= k -> zs
        (z @ zs)))} -> res
  -------------------------------
  lm >>= k -> res
}
\end{lstlisting}

It starts with the \verb\ArrowFunc\ syntax and uses the match to determine if the list actually contains a list or not.
When it is \verb\empty\ the bind simply returns an empty.

When a list is present the first element of the list gets unpacked by the bind of \verb\'M\, which gets named \verb\y\.
The function \verb\k\ gets executed with \verb\y\ as its argument and the result gets repacked by the return.

Now the rest of the list, \verb\xs\, gets passed to the bind of \verb\list\ so the entire list gets passed to \verb\k\.
The results of processing \verb\x\ and \verb\xs\ are then concatenated into a single list which gets returned as a result of the bind.

If the use of the different syntaxes of the \verb\ArrowFunc\ seems confusing, take extra look at section~\ref{sec:basicmcarrowfunc}.

\subsubsection{Evolution}
\subsection{Either}\label{sec:basicmonadseither}
The either monad is an implementation of the \verb\TryableMonad\ interface.

It creates a monad which can be either a value or a list of fails.
The type signature is as follows:

\begin{lstlisting}
TypeAlias "EitherT" => (#a => #b) => #c => #d => #e
EitherT 'M ['e] 'a => 'M('a | 'e)
\end{lstlisting}

It uses the pipe to be able to be either failed or a value.

The \verb\Tryable monad\ is then called with the signature of the monadic argument.

\begin{lstlisting}
TypeFunc "either" => Monad => #a => TryableMonad
either 'M (List 'e) => TryableMonad(EitherT MCons^'M (List 'e)) {
\end{lstlisting}

The \verb\'a\ is again not used, as is complient with section~\ref{sec:basicmonadsimplementedid}.

Next we see the bind and return defined:

\begin{lstlisting}
  pm >>= k -> try pm k fail
  return x -> return^'M(Left x)
\end{lstlisting}

The bind calls \verb\try\ which is declared but not yet implemented.
The try will then be defined:

\begin{lstlisting}
  {pm >>=^'M y
    (match y with
      (\x -> k x)
      (\e -> err e)) -> z
    return^'M z} -> res
  ---------------------
  try pm k err -> res
}
\end{lstlisting}

The \verb\try\ takes monad, a function and an error function.
The error function will be executed if \verb\y\ contains an error and if it contains a value the first function will be executed.

It calls the bind if \verb\'M\ to get to the value inside.

We also have a fail function which can be called if the monad fails.

\begin{lstlisting}
  Func "fail" -> 'e -> MCons 'b
  fail e -> return^'M(Right (Right^'M :: e))
\end{lstlisting}

What these fails are is up to the programmer.
The fails get concatinated into a list and returned.

\subsubsection{Evolution}
\subsection{Option}
CAN THIS BE LEFT OUT???????????
probably yes

\begin{lstlisting}
TypeFunc "Option" => #a => #b
Option 'a => Unit | 'a

Func "Some" -> 'a -> Option 'a
Some x -> Right x

Func "None" -> Option 'a
None -> Left Unit

TypeFunc "option" => TryableMonad
option => either Option None {
  return
}
\end{lstlisting}
\subsubsection{Evolution}

\subsection{Result}
The result monad is complete implementation of the either monad from section~\ref{sec:basicmonadseither}.
It takes a monad and uses a \verb\String\ as the error type.

\begin{lstlisting}
TypeFunc "result" => Monad => TryableMonad
result 'M => either MCons^'M (List String)
\end{lstlisting}

\subsubsection{Evolution}
\subsection{State}
We have hade a basic explanation of the state monad in section~\ref{sec:standardmonad}.

Now we will see how it is really implemented.
The first thing to do is create a type signature of the state monad:

\begin{lstlisting}
TypeAlias "StateT" => (#a => #b) => #c => #d => #e
StateT 'M 's 'a => ('s -> 'M('a * 's))
\end{lstlisting}

As we can see the state monad is defined as a function.
It takes a state and returns a monad containing a tuple af the resulting value and the new state.

We use \verb\StateT\ in the call to \verb\Monad\ so it knows what the type signature will be.

\begin{lstlisting}
TypeFunc "state" => Monad => #a => Monad
state 'M 's => Monad(StateT MCons^'M 's) {
\end{lstlisting}

When looking at the return we see something new.

\begin{lstlisting}
  return x -> (\ s -> return^'M(x,s))
\end{lstlisting}

A lambda is created to match the type signature of the state monad.

We see this happening also in the bind.

\begin{lstlisting}
  (\ s ->
    {p s >>=^'M x
      x -> (x',s')
      k x' s'}) -> res
  ---------------------
  p >>= k -> res
\end{lstlisting}

The bind of \verb\'M\ is called and the result is then deconstructed to get to the tuple which it contains.
We know \verb\x\ is a tuple because it lives inside a state monad which always creates a tuple inside the resulting monad.
As is evident by the type signature of the state monad.

The state monad also contains two extra functions.
They make it possible to get and set the state.

To get the state we call \verb\getState\.

\begin{lstlisting}
  Func "getState" -> MCons 's
  getState -> (\ s -> return^'M(s,s))
\end{lstlisting}

It sets the state as the result in the return lambda.
By which you will get the state back when it is entered into the lambda.

To set the state we call \verb\setState\.

\begin{lstlisting}
  Func "setState" -> 's -> MCons Unit
  setState s -> (\ unit -> return^'M(unit,s))
}
\end{lstlisting}

It sets the result to unit and the state to the input state.

\subsubsection{Evolution}
\subsection{IO}
\subsubsection{Evolution}


   % DROP THIS
   % \chapter{Casanova}
   % \section{Entity}
   % \section{Coroutines}

   \chapter{Test programs}
There have been two test programs written to test the language.

The first of these is a simple bouncing ball.


\section{Bouncing ball}
Bouncing ball is used as a preliminary test of the updatable records.
It will show how the updatable records are used in practice.

This program bounces a ball up and down.

The ball will have two properties, velocity and position, which will implemented as records.
Both of these will then be put inside the updatable record of the ball.

The update function will then add implement how the ball bounces.

First we import prelude, record and the XNA framework
The Microsoft XNA library will be used for the \verb\Vector2\ it has and the Thread library will be used to call the \verb\Sleep\ function.

\begin{lstlisting}
import Microsoft^Xna^Framework
import System^Threading^Thread
import record
import prelude
\end{lstlisting}

The record of the ball will be build bottom-up.
This is done because every record needs to have a \verb\rest\ as an argument.

\verb\Velocity\ is created first, because it needs to be updated less.
It keeps deep searches of the record to a minimum.

\begin{lstlisting}
RecordEntry "Velocity" Vector2(0.0f, 98.1f) Empty
\end{lstlisting}

Then \verb\Position\ will be created with \verb\Velocity\ as its \verb\rest\ argument.

\begin{lstlisting}
RecordEntry "Position" Vector2(100.0f, 0.0f) Velocity
\end{lstlisting}

Lastly we create the updatable record \verb\Ball\.
This is done in one line by creating a record entry and passing it directly as argument to \verb\updatableRecord\.
\verb\Position\ is used as \verb\rest\ argument to complete the record.

\begin{lstlisting}
updatableRecord (RecordEntry "Ball" unit Position) => Ball {
\end{lstlisting}

We then need to define the \verb\update\ function.

The \verb\Position\ and \verb\Velocit\ records are put in a variable, so we can use them directly.
\begin{lstlisting}
  get^e "Position" Rest^e => position
  get^e "Velocity" Rest^e => velocity
\end{lstlisting}

\begin{lstlisting}
  (if ((Field^position).Y <= 500.0f) then
    ((set^e "Position"
           (Field^position +^Vector2 (Field^velocity *^Vector2 dt))
           Rest^position)
     (set^e "Velocity"
            (Field^velocity +^Vector2 (Vector2(0.0f, 98.1f) *^Vector2 dt))
            Rest^velocity))
  else
    (set^e "Position"
           Vector2((Field^position).X, 500.0f)
           -^Vector2(Field^velocity))) -> res
  --------------------------------------------
  update e dt -> res
}
\end{lstlisting}

The final step is to create the \verb\run\ function to start the program.
It will call the update function of \verb\Ball\, sleep for a set amount of time and then call itself.

\begin{lstlisting}
Func "run" -> Record -> String

update^ball ball time -> ball'
Sleep(1000)
run ball' (time +^Int 1000) -> res
---------------
run ball time -> res
\end{lstlisting}

This creates an infinite loop, so the ball will keep on bouncing.

The return will never be set as it will keep calling itself until the program stops.

\section{Tron}
Tron is a more advanced program which will utilise user input.
The game tron~\cite{} has been chosen because of the minimalistic graphics.
This allows us to focus on the internal workings of the game.


\subsection{Bike}

\begin{lstlisting}
import Framework^Xna^Microsoft
import Input^Windows^System
import entity^Casanova
import prelude

TypeFunc "PositionEntity" => String => Vector2 => EntityField => EntityField
Entity label pos rest => e
--------------------------
PositionEntity label pos rest => UpdatableEntity e{
  dts -> (dt,speed)
  Field^p +^Vector2 dt *^Vector2 speed -> newPos
  Entity label^p newPos Rest^p -> res
  -----------------------------------
  update p dts -> res
}

TypeFunc "PositionRule" => Rule
PositionRule => (Rule PositionEntity){
  update^e dt -> res
  ------------------
  apply e dt -> res
}

TypeFunc "TrailEntity" => String => Vector2 => EntityField => EntityField
Entity label field rest => e
----------------------------
TrailEntity label field rest => UpdatableEntity e{
  Entity label^t pos fields^t => res
  ----------------------------------
  update t pos => res
}

TypeFunc "TrailRule" => Rule
TrailRule => (Rule TrailEntity){
  update^e dt -> res
  ------------------
  apply e dt -> res
}

TypeFunc "Keys" => Key => Key => Key => Key => EntityField
Entity "Left" left Empty
Entity "Right" right Left
Entity "Up" up Right
Entity "Down" down Up => res
------------------------------
Keys left right up down => res

TypeFunc "Bike" => String => Int => Keys => Vector2 => TrialEntity => EntityField => EntityField
Entity "Colour" rgb Empty
Entity "IsAlive" True^builtin Colour
Entity "Controls" keys IsAlive
Entity "Speed" speed Controls
PositionEntity "Position" position Speed
Entity "Trail" trail Position => field
Entity label field rest => e
--------------------------------------
Bike label rgb keys speed position trail rest => UpdatableEntity e{
  $$ check if alive
  (if Field^IsAlive then
    get^b "Trial" Field^b => trial
    get^trail "Position" Rest^trail => position
    $$ update trail
    apply^TrialRule Field^trial Field^position => newTrailEntity
    $$ update position
    get^position "Speed" Rest^position => speed
    apply^PositionRule position (dt,Field^speed) -> newPos
    Entity label^trial newTrialEntity newPos => newTrial
    Entity Label^b newTrail Rest^b
  else
    b) -> res
  -------------------------------------
  update b dt -> res
}

TypeFunc "BikeRule" => Rule
BikeRule => (Rule BikeEntity){
  update^e dt -> res
  ------------------
  apply e dt -> res
}
\end{lstlisting}


\subsection{Playfield}
\begin{lstlisting}
import Framework^Xna^Microsoft
import entity^Casanova
import prelude
import bike


TypeFunc "BikeEntity" => String => EntityField => EntityField => EntityField
BikeEntity label bikes rest => Entity label bikes rest{

  Func "updateIsAliveAll" -> EntityField -> Vector2 -> EntityField
  Func "updateIsAliveAll" -> EntityField -> Vector2 -> EntityField
  Field^bs -> bikeEntities
  updateBikes bikeEntities dt -> newBikeEntities
  set^bs label^bs newBikeEntities bs -> res
  ------
  updateIsAlive bs fieldsize -> updatedBikes

  Func "updateBikes" -> EntityField -> Float -> EntityField
  update^b b dt
  updateBikes Rest^b dt
  ---------------------
  updateBikes b dt -> res

  updateBikes Empty dt -> Empty

  Field^bs -> bikeEntities
  updateBikes bikeEntities dt -> newBikeEntities
  set^bs label^bs newBikeEntities bs -> res
  -----------------------------------------
  update bs dt -> res
}

TypeFunc "Playfield" => String => EntityField => Vector2 => EntityField => EntityField
Entity "Winner" Unit Empty
BikeEntity (label + "bikes") bikes Winner
Entity "FieldSize" size BikeEntity
-> field
-----------
Playfield label bikes size rest => Entity label field rest{

  Func "checkBikeCollisions" -> EntityField -> EntityField -> EntityField

  -> newBikes
  ------
  checkBikeCollisions b bs -> newBikes

  Func "checkFieldCollisions" -> Playfield -> EntityField -> EntityField
  get "Position" b -> bikePosition
  get "FieldSize" p -> playfieldSize
  (if ((X.bikePosition > X.playfieldSize) ||^System
       (X.bikePosition < 0.0) ||^System
       (Y.bikePosition > Y.playfieldSize) ||^System
       (Y.bikePosition < 0.0))then
    set^b "IsAlive" False^builtin b
  else
    b) -> newBike
  --------
  checkFieldCollisions p b -> newBike

  Func "checkCollisions" -> Playfield -> EntityField -> Playfield
  checkFieldCollisions p b -> newBike
  (if (Rest^newBike = Empty) then
    set^p
  else
    checkBikeCollisions b Rest^b -> newNewBike
    checkCollisions p Rest^b
    ) -> newPlayfield
  ------
  checkCollisions p b -> newPlayfield

  Func "checkDeaths" -> Playfield -> Playfield

  checkDeaths p -> newP



  get^p (Label^p + "bikes") Field^p -> bikes
  update^bikes bikes dt -> newBikes
  set^p (Label^p + "bikes") newBikes Fields^p -> newPlayfield
  checkDeaths newPlayfield ->
  --------
  update p dt -> res
}
\end{lstlisting}


\subsection{Powerups}
\begin{lstlisting}
Func "dottedLine" -> Bike -> Bike
Func "ghostMode" -> Bike -> Bike
Func "thickerLine" -> Bike -> Bike
\end{lstlisting}


   % DROP THIS
   % \part{Extra curicular activities}
   % \chapter{Syntax highlighter}
During the assignment I was getting confused at times by all the plain text I was seeing on screen.
I then created a plugin for vim to have some basic syntax highlighting for MC.





\end{multicols}
\part{Results}
\begin{multicols}{3}

   \chapter{Conclusion}
To summerize the main research question was:
\emph{How can the programming language MetaCasanova be improved from its current state within the timeframe of the internship?}

The answer to this research question is:
\emph{MetaCasanova can be improved by bringing in someone with a fresh view on the language.
This person will use be taught the basics of the language and use it to build test programs.
He or she will also extend the standard library with what is needed for these programs to work.}


\chapter{Recommendations}
These recommendations are for people who continue with the project.

\section{Improving a programming language}
When first learning about a language, keep on asking until you have specifics.
Most of the developers have general ideas of a language.
These ideas have very little specifics, but will come to the surface when really digging.

By having these ideas before hand, bugs will become apparent when first looking at the actual language.


\section{Further development}
When developing MC further I would recommend to expand the standard library even more.
Especially when the focus of the language is put more on monads.

\subsection{Monads}
First we would need a proper IO monad for interacting with the unpure functions and get real output.

Coroutines would be next.
Both as a separate entity and as a monad.
They would give the ability of multi threading directly in the language.
This would improve performance.

\subsection{The compiler}
I would also recommend to keep syntax changes to a minimum.
This will give the compiler developers a fighting chance.
During the development the compiler developers have thrown away alot of work, because the syntax became too different.

When creating a new itteration of the language, first get a clear picture of the language before starting work on the compiler.
This creates less work for the compiler developers as they do not have to change the compiler mid build.


\chapter{Evaluation}
Here I will provide proof of having mastered the dublin descriptors and the extra competentions as set in the graduation module guidelines~\cite{}.

\section{Dublin descriptors}
\subsection{Knowledge and understanding}
During the project I have used type-theory as described in~\cite{} to check the programs I was writing.
This can be explicitly seen in section~\ref{sec:standardtryablemonad} and appendix~\ref{app:typeproofs}.

Further more I have learned the monadic theory to be able to implement them in the standard library.

Both of these build on the knowledge gained during my four years of studying at the University of Rotterdam.
Here I have learned the basis which was needed during this project.

The programming skills learned were needed to expand these skills with the type theory.
The mathematical skills gained were needed when learning about the monadic constructs.


\subsection{Applying knowledge and understanding}
The knowledge gained in the research phase of this project combined with the knowledge gained in my study years, have been put into practical use.
When building upon MC I had to use all the type theory knowledge gained to be certain no function failed.

When implementing the monadic part of the standard library, I have combined the programming- with the mathematical skills to translate mathematical ideas into a practical code base.


\subsection{Making judgements}
At the start of this project I have done research into a new language, MC, type theory and monadic mathematical concepts.
These were all needed to build upon the existing codebase of MC.

\textbf{DO CITATIONS}
I have made descisions while keeping both the practical aspects and goals in mind, as seen in sections~\ref{},\ref{} and \ref{}.


\subsection{Communication}
During this project I have worked in a research team where I have had to explain myself to fellow team members and supervisors.

This was done in the form of presentations, discussions and meetings.
I also had to understand their ideas about concepts.


\subsection{Lifelong learning skills}


\section{Extra competentions}


   \bibliographystyle{ieeetr}
   \bibliography{biblio}

\end{multicols}
\begin{appendices}
   \part{Appendices}

   \include{typeproofs}
\end{appendices}

% \chapter{Glossery}
\end{document}
